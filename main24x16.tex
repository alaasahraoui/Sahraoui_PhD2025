%%%%%%%%%%%%%%%%%%%%%%%%%%%%%%%%%%%%%%%%%
% Masters/Doctoral Thesis 
% LaTeX Template
% Version 2.1 (2/9/15)
%
% This template has been downloaded from:
% http://www.LaTeXTemplates.com
%
% Version 2.0 major modifications by:
% Vel (vel@latextemplates.com)
%
% Original authors:
% Steven Gunn  qaaaaaaaaaaaaaaaaaaaaaaaaaaaaaaaaaaa(http://users.ecs.soton.ac.uk/srg/softwaretools/document/templates/)
% Sunil Patel (http://www.sunilpatel.co.uk/thesis-template/)
%
% License:
% CC BY-NC-SA 3.0 (http://creativecommons.org/licenses/by-nc-sa/3.0/)
%
%%%%%%%%%%%%%%%%%%%%%%%%%%%%%%%%%%%%%%%%%

%----------------------------------------------------------------------------------------
%	PACKAGES AND OTHER DOCUMENT CONFIGURATIONS
%----------------------------------------------------------------------------------------

\documentclass[
10pt, % The default document font size, options: 10pt, 11pt, 12pt
%oneside, % Two side (alternating margins) for binding by default, uncomment to switch to one side
english, % ngerman for German
singlespacing, % Single line spacing, alternatives: onehalfspacing or doublespacing
%draft, % Uncomment to enable draft mode (no pictures, no links, overfull hboxes indicated)
%nolistspacing, % If the document is onehalfspacing or doublespacing, uncomment this to set spacing in lists to single
liststotoc, % Uncomment to add the list of figures/tables/etc to the table of contents
%toctotoc, % Uncomment to add the main table of contents to the table of contents
%parskip, % Uncomment to add space between paragraphs
]{MastersDoctoralThesis} % The class file specifying the document structure

\usepackage[utf8]{inputenc} % Required for inputting international characters
\usepackage[T1]{fontenc} % Output font encoding for international characters
\usepackage[font=footnotesize]{caption}
\usepackage{palatino} % Use the Palatino font by default % lmodern

%\usepackage[backend=bibtex,style=numeric,natbib=true]{biblatex} % User the bibtex backend with the authoryear citation style (which resembles APA)
%\usepackage[backend=bibtex,sorting=nyt,style=numeric,natbib=true]{biblatex} % User the bibtex backend with the authoryear citation style (which resembles APA)
\usepackage[backend=bibtex,style=authoryear,natbib=true,maxnames=99]{biblatex}





\usepackage{enumitem}
%\usepackage{showframe}






\usepackage{amsmath}
\usepackage{amssymb}
\usepackage{graphicx}
\usepackage{amsfonts}
\usepackage{mathtools}
\usepackage{mathrsfs}
\usepackage{algorithm}
\usepackage{algorithmic}
\usepackage{accents}
\usepackage{multirow}
\usepackage{adjustbox}
%\usepackage{pgfplots}
\usepackage{tabularx}
%\usepackage{subfigure} this is outdated 
\usepackage{subcaption}
\usepackage{makecell}
\usepackage{xcolor,colortbl}
\usepackage[most]{tcolorbox}
\usepackage{multirow, multicol}
\usepackage{rotating}
%Improved tables
\usepackage{rotating}
\usepackage{booktabs}
\newcommand{\ra}[1]{\renewcommand{\arraystretch}{#1}}
\definecolor{lightgray}{rgb}{0.83, 0.83, 0.83}
\usepackage{xspace}
\newcommand{\ie}{\textit{i.e.,}\xspace}
\newcommand{\eg}[0]{\textit{e.g.,}\xspace}
\newcommand{\vs}[0]{\textit{vs.}\xspace}
\newcommand{\etal}{\textit{et al.}\xspace}
%Improved vectors
\DeclareRobustCommand*{\ora}{\overrightarrow}

%\usepackage{csquotes}

% \usepackage{bm}
%Color
\tcbset{on line, 
        boxsep=4pt, left=0pt,right=0pt,top=0pt,bottom=0pt,
        colframe=white,colback=blue!45!gray!35,
        highlight math style={enhanced}
        }


\usepackage{epstopdf}
\usepackage{tikz}
\usetikzlibrary{arrows,backgrounds,snakes}

\DeclareUnicodeCharacter{2212}{-}

\usepackage{exscale,relsize} % allows to adapt the font size continuously

\renewcommand{\algorithmicrequire}{\textbf{Input:}}
\renewcommand{\algorithmicensure}{\textbf{Output:}}
\renewcommand{\algorithmiccomment}[1]{$\quad\triangleright$ #1}
\newcommand{\mat}[1]{\mathbf{#1}}
\newcommand*\Bell{\ensuremath{\boldsymbol\ell}}
\newcommand{\dist}{{\textstyle \mathsmaller{\mathit{\Delta}}}} % define the distance symbol
\renewcommand{\vec}[1]{\mathbf{vec}(#1)}
\newcommand{\vecnp}{\textbf{vec}}
\newcommand{\risk}[1]{\textbf{risk}(#1)}
\usepackage{array}
\newcolumntype{L}[1]{>{\raggedright\let\newline\\\arraybackslash\hspace{0pt}}m{#1}}
\newcolumntype{C}[1]{>{\centering\let\newline\\\arraybackslash\hspace{0pt}}m{#1}}
\newcolumntype{R}[1]{>{\raggedleft\let\newline\\\arraybackslash\hspace{0pt}}m{#1}}

\newcommand{\ie}{\textit{i.e.}\xspace}
\newcommand{\eg}[0]{\textit{e.g.}\xspace}
\newcommand{\vs}[0]{\textit{vs.}\xspace}
\newcommand{\etal}{\textit{et al.}\xspace}

\def\D{\mathcal{D}}
\def\expm{\mathop{\mathrm{expm}}}
\let\ab\allowbreak
\newcommand{\Diag}[1]{\mathbf{Diag}(#1)}
\newcommand{\den}[1]{\mathrm{den}(#1)}
%\newcommand{\trace}[1]{\mathrm{trace}\left(#1\right)} \mathop{\text{trace}}
\newcommand{\trace}[1]{\mathop{\mathrm{trace}}(#1)}
%\newcommand{\vol}[1]{\mathrm{vol}(#1)}
\newcommand{\vol}[1]{\textstyle \mathsmaller{\mathrm{vol}(#1)}}
%\newcommand{\vol}[1]{\textstyle \mathsmaller{\text{vol}(#1)}}
\renewcommand{\vec}[1]{\mathbf{vec}(#1)}
%\renewcommand{\vec}[1]{\ensuremath{\mbox{\boldmath$#1$}}}
\newcommand{\giv}{\ensuremath{\: \vert \:}}
\newcommand{\argmin}{\mathop{\mathrm{arg\,min}}}
\newcommand{\argmax}{\mathop{\mathrm{arg\,max}}}
\newcommand{\minimize}{\mathop{\mathrm{minimize}}}
\newcommand{\maximize}{\mathop{\mathrm{maximize}}}
\newcommand{\subjectto}{\mathop{\mathrm{subject\,to}}}
\newcommand{\numerator}{\mathop{\mathrm{numerator}}}
\newcommand{\myDelta}{{\textstyle \mathsmaller{\varDelta}}} % define the distance symbol
\newcommand{\noth}{\mathrlap{\text{h}}/}
\newcommand{\notzero}{\mathrlap{\text{0}}/}
\newcommand{\dsum}{\displaystyle\sum}
\newcommand{\smallbullet}{\,\begin{picture}(-1,1)(-1,-3)\circle*{3}\end{picture}\ \,}
%\newcommand*{\smallbullet}{\raisebox{-0.25ex}{\scalebox{1.2}{$\cdot$}}}
\renewcommand{\labelitemi}{$\blacktriangleright$} % define the item symbol
%\renewcommand{\labelitemi}{$\blacksquare$}
%\renewcommand{\labelitemii}{$\sqbullet$}

\DeclareCiteCommand{\fullcite}
  {\usebibmacro{prenote}}
  {\usedriver
     {\defcounter{minnames}{6}%
      \defcounter{maxnames}{6}}
     {\thefield{entrytype}}.}
  {\multicitedelim}
  {\usebibmacro{postnote}}

%\mleftright % avoids extra spaces in \left( \right)


% Circles
\newcommand*\emptycirc[1][0.7ex]{\tikz\draw[thick] (0,0) circle (#1);} 
\newcommand*\halfcirc[1][0.7ex]{%
  \begin{tikzpicture}
  \draw[fill] (0,0)-- (90:#1) arc (90:270:#1) -- cycle ;
  \draw[thick] (0,0) circle (#1);
  \end{tikzpicture}}
\newcommand*\fullcirc[1][0.7ex]{\tikz\fill (0,0) circle (#1);} 


\definecolor{ForestGreen}{rgb}{0.13, 0.55, 0.13}
%\setlength{\headheight}{27.06pt}
\addbibresource{Biblio.bib} % The filename of the bibliography

\usepackage[autostyle=true]{csquotes} % Required to generate language-dependent quotes in the bibliography

\renewcommand{\chaptermark}[1]{%
\markboth{\chaptername
\ \thechapter.\ #1}{}}

\usepackage{fancyhdr} 
\fancyhead[LO,RE]{\textsl{\rightmark}}
\fancyhead[LE,RO]{\textsl{\leftmark}}
\fancyfoot[C]{\thepage}

\usepackage[activate={true,nocompatibility},final,tracking=true,kerning=true,spacing=true,factor=1100,stretch=10,shrink=10]{microtype}
\microtypecontext{spacing=nonfrench}

%----------------------------------------------------------------------------------------
%	THESIS INFORMATION
%----------------------------------------------------------------------------------------

    \thesistitle{The Extra-User Interface to Support User Interface Evolution: Definition, Design Spaces, and Implementations for Adaptation}
                
    
%\thesistitle{A Method For A Comparative Testing of 3D Hand Gestures Across Multiple Context of Use }
% Your thesis title, this is used in the title and abstract, print it elsewhere with \ttitle
\supervisor{Prof. Jean \textsc{Vanderdonckt}} % Your supervisor's name, this is used in the title page, print it elsewhere with \supname
\examiner{} % Your examiner's name, this is not currently used anywhere in the template, print it elsewhere with \examname
\degree{Doctor in Engineering and Technology} % Your degree name, this is used in the title page and abstract, print it elsewhere with \degreename
\author{Alaa \textsc{Sahraoui}} % Your name, this is used in the title page and abstract, print it elsewhere with \authorname
\addresses{} % Your address, this is not currently used anywhere in the template, print it elsewhere with \addressname

\subject{THE SUBJECT} % Your subject area, this is not currently used anywhere in the template, print it elsewhere with \subjectname
\keywords{} % Keywords for your thesis, this is not currently used anywhere in the template, print it elsewhere with \keywordnames
\university{\href{http://www.uclouvain.be}{Universit\'e catholique de Louvain}} % Your university's name and URL, this is used in the title page and abstract, print it elsewhere with \univname
\department{Ecole polytechnique de Louvain} % Your department's name and URL, this is used in the title page and abstract, print it elsewhere with \deptname
\group{\href{https://uclouvain.be/en/research-institutes/lourim}{LouRIM}} % Your research group's name and URL, this is used in the title page, print it elsewhere with \groupname
\faculty{\href{https://uclouvain.be/en/faculties/lsm}{Louvain School of Management}} % Your faculty's name and URL, this is used in the title page and abstract, print it elsewhere with \facname

\hypersetup{pdftitle=\ttitle} % Set the PDF's title to your title
\hypersetup{pdfauthor=\authorname} % Set the PDF's author to your name
\hypersetup{pdfkeywords=\keywordnames} % Set the PDF's keywords to your keywords

\begin{document}

\frontmatter % Use roman page numbering style (i, ii, iii, iv...) for the pre-content pages

\pagestyle{plain} % Default to the plain heading style until the thesis style is called for the body content

%----------------------------------------------------------------------------------------
%	TITLE PAGE
%----------------------------------------------------------------------------------------
\thispagestyle{plain}
\begin{titlepage}
\begin{center}

% \textsc{\large \univname}\\[1.5cm] % University name


\begin{figure}
 
 \includegraphics[scale = 0.15]{Figures/Cover/UCLouvainLouRIMLogo.png}
 \vspace{1cm} 
 \includegraphics[scale = 0.085]{Figures/Cover/Lourim.png}
\end{figure}

%\textsc{\large Doctoral Thesis}\\[0.5cm] % Thesis type

%\HRule \\[0.4cm] % Horizontal line

{\Large \bfseries \ttitle}\\[0.5cm] % Thesis title
\vspace{1cm}

%\HRule \\[1.0cm] % Horizontal line
 
\begin{minipage}{0.4\textwidth}
\Large
\centering
\emph{by}\\
\vspace{0.25cm}
\href{http://www.uclouvain.be/mehdi.ousmer}{\authorname} % Author name - remove the \href bracket to remove the link
\end{minipage}

 
\vspace{1cm}
	
%\large 
{Thesis submitted in fulfillment \\ of the requirements for the degree of \\[0.3cm] \large{Doctor in Engineering and Technology}}\\[0.3cm] % University requirement text
{of the \univname}\\
{\deptname}\\[1.0cm]
\end{center}
\smaller
\textit{Commitee in charge:} \\[0.3cm]
\begin{minipage}{0.6\textwidth}
\begin{flushleft}
Prof. Gaëlle Calvary (Examiner)\\
Prof. Rachid Gherbi (Examiner)\\
%Prof. Suzanne Kieffer (Secretary) \\
%Prof. Charles Pecheur (President) \\
Prof. Jean Vanderdonckt (Supervisor)\\
\end{flushleft}
\end{minipage}
\hspace{-30px}
\begin{minipage}{0.6\textwidth}
\begin{flushleft}%President: Prof. Charles Pecheur \\
%\univname, Belgium \\
Université Grenoble Alpes, France  \\
Université de Paris-Saclay, France \\
\univname, Belgium \\
%\univname, Belgium \\
\end{flushleft}
\end{minipage}\\[1.0cm]
 
\vfill

\end{titlepage}

\newpage\null\thispagestyle{empty}\newpage


\thispagestyle{empty}
\begin{quote}\enquote{It is not because things are difficult that we do not dare, it is because we do not dare that they are difficult.}
  \begin{flushright}
    ---Seneca, Epistuale Morales
  \end{flushright}
  \end{quote}  

\newpage
\thispagestyle{empty} 

%----------------------------------------------------------------------------------------
%	ABSTRACT PAGE
%----------------------------------------------------------------------------------------

\thispagestyle{plain}
\begin{center}{\huge\textit{Abstract}\par}\end{center}


\vspace{2cm}
\addchaptertocentry{\abstractname} % Add the abstract to the table of contents
Context-aware Interactive applications are those interactive software that have parts or whole changing depending on the constraints imposed by a dynamically-changing context. The context of use covers the user and the associated tasks, the computing platform and devices, and the physical environment. Therefore, any significant change of any of these three dimensions, i.e., the user, the platform, and the environment, may trigger a change of the software, including the user interface. Until now, such changes have been managed by the system, thus leading to a system-controlled context-awareness. Instead, we want to pursue the goal of letting the end user control the changes, thus leading to a user-controlled context awareness. To enable the end user to have this control facility, there is a need for another user interface than the one of the original software. This is the Extra-User Interface, which is hereby referred to as the user interface to control the user interface of another interactive application to support context awareness. This concept can be applied in principle to any domain of computer science (e.g., ambient intelligence, smart rooms, ubiquitous computing, multimedia). We will instantiate this approach to the area of information visualization.

\vspace{0.5cm}%{0.33cm}
\textbf{Keywords:} Adaptation, Adaptation operation, Adaptive user interfaces, Extra User Interface, Human-Computer Interaction, Information visualization, Meta User Interface, Plasticity of user interfaces, User control 
\checktoopen
\null
\vspace{1em}%{0.33cm}





%----------------------------------------------------------------------------------------
%	ACKNOWLEDGEMENTS
%----------------------------------------------------------------------------------------


\thispagestyle{plain}
\begin{center}{\huge\textit{Acknowledgements}\par}\end{center}
\vspace{0.5cm}
\addchaptertocentry{\acknowledgementname} % Add the Acknowledgements to the table of contents

\label{chap:acknowledgement}
First of all, I would like to express my sincere gratitude to my thesis supervisor, Prof. Jean Vanderdonckt, for providing me with the opportunity to complete this thesis. He has always supported me throughout my Ph.D. and related research. His patience, motivation, and extensive knowledge have been incredibly valuable. In addition, his advice helped me throughout the research and writing of this thesis.\\

I would like to express my gratitude to my co-supervisor, Professor Paolo Roselli, for his invaluable collaboration and assistance in navigating the complex knowledge of geometric algebra and its application in gesture recognition. His guidance and encouragement have played a pivotal role in shaping the direction of my research.\\

I would like to express my gratitude to the members of my Ph.D. committee for reviewing the manuscript and providing valuable comments. To Prof. Rachid Gherbi, whom I met during my master's degree in Algeria and who was a source of inspiration for his dedication to transmitting his knowledge about the VR/AR field by giving constructive advice. I would like to thank Prof. Suzanne Kieffer and Prof. Giovanni Saggio for their expertise and contribution to the quality of this thesis. I would also like to thank Prof. Charles Pecheur for being the president of my thesis committee.\\

Special thanks to Prof. Radu Vatavu, from Stefan cel Mare University of Suceava, who shared his deep knowledge in gesture interaction~\cite{Vatavu:2023:HHCI} with me at different stages of my research.

In addition, I would like to thank my colleagues at the Louvain Research Institute in Management and Organizations (LouRIM) Institute and the assistants and researchers for their fruitful discussions, comments, and support throughout this research and teaching work. I would also like to thank the administrative team for their invaluable advice and excellent work in facilitating our tasks and creating a favorable working environment.\\

Finally, I am deeply grateful to my family and friends for their unconditional love, understanding, and support throughout this Ph.D. journey. I want to express my profound gratitude for their constant source of motivation, strength, and invaluable advice.






%----------------------------------------------------------------------------------------
%	PUBLICATIONS
%----------------------------------------------------------------------------------------
\chapter*{Publications}\label{chap:publications}
\addchaptertocentry{\publicationname}
\begin{frame}

\begin{itemize}
    \item\begin{refsection}~\fullcite{Calvary:2025}\end{refsection}
    \item\begin{refsection}~\fullcite{Sahraoui:2025a}\end{refsection}
    \item\begin{refsection}~\fullcite{Sahraoui:2025B}\end{refsection}
\end{itemize}
\end{frame}



%----------------------------------------------------------------------------------------
%	LIST OF CONTENTS/FIGURES/TABLES PAGES
%----------------------------------------------------------------------------------------
\setcounter{tocdepth}{2}
\tableofcontents % Prints the main table of contents

\listoffigures % Prints the list of figures
\listoftables % Prints the list of tables

%----------------------------------------------------------------------------------------
%	DEDICATION
%----------------------------------------------------------------------------------------

%\dedicatory{To my family.} 
%\thispagestyle{empty} 

%----------------------------------------------------------------------------------------
%	THESIS CONTENT - CHAPTERS
%----------------------------------------------------------------------------------------

\mainmatter % Begin numeric (1,2,3...) page numbering

\pagestyle{fancy} % Return the page headers back to the "thesis" style

% Include the chapters of the thesis as separate files from the Chapters folder
% Uncomment the lines as you write the chapters

\chapter{Introduction}\label{chap:Introduction}


\section{Context and Motivations}\label{sec:Introduction}

\paragraph{Software and contextual evolution.}
\textit{Software evolution}~\cite{Lehman:1980} is the continuous process of developing, maintaining, and updating software after its initial release to meet changing requirements and needs. Software evolution is required when (non-)functional requirements change, when the \textit{context of use} evolves, or when usability errors, bugs, and security vulnerabilities need to be fixed~\cite{Mens:2008:book,Zaidman:2010}. The \textit{context of use} \cite{Calvary:2003,Dey:2001} of an interactive application encompasses the end user and the related interactive tasks, the computing platform, and the environment in which interaction takes place. Consequently, any \textit{contextual evolution} of one of these three dimensions, such as the arrival of a new category of users~\cite{Pleuss:2013}, an updated task~\cite{Vandenbergh:2010}, a new version of the operating system, a new form factor for the UI, or an evolving physical and social environment~\cite{Mens:2008:intro}, is relevant to software evolution. In the presence of contextual evolution, the User Interface (UI) itself should change accordingly, thus raising challenges of \textit{contextual adaptation}~\cite{Calvary:2025}, a particular case of software evolution in which the UI needs to be adapted to the evolving context of use~\cite{Abrahao:2021}.

\paragraph{The promise of adaptive and context-aware UIs.}
``The goal is not to replace people but to empower them by making design choices that give humans control over technology,'' as stated by Shneiderman \cite{Shneiderman:2022}. This sentiment encapsulates a core challenge in modern Human–Computer Interaction (HCI): as UIs become increasingly adapted to new contexts, how can we ensure that the \textit{user} remains in control of this adaptation?

Ambient intelligent environments, such as smart homes, intelligent meeting rooms, or context-aware workspaces, automatically tailor UIs to situational factors (user preferences, devices at hand, physical environment) in order to enhance usability. This vision of ubiquitous computing promises UIs that seamlessly ``mold'' to their context of use, \ie exhibit \textit{plasticity}~\cite{Calvary:2007}. Adaptive User Interfaces (AUIs) incorporate \textit{adaptivity}: the ability of the system to autonomously adjust UI content, layout, or behavior in response to contextual changes~\cite{Browne:1990}. Typical examples include a smartphone application switching to a dark theme at night, or a car dashboard simplifying its display when the driver’s cognitive load is high. These adaptive capabilities are associated with several \textbf{anticipated advantages}~\cite{Gajos:2008,Gajos:2010}:

\begin{itemize}
  \item \textbf{Improved efficiency:} the system pre-configures the UI for the current context of use, reducing the number of manual steps required from the user.
  \item \textbf{Reduced cognitive load:} by hiding non-relevant functions and emphasizing relevant ones, the UI can become easier to scan and operate.
  \item \textbf{Better fit to the context of use:} adaptation enables the same application to function across a variety of devices, environments, and user profiles, supporting UI plasticity.
  \item \textbf{Personalization:} over time, adaptive mechanisms can tailor interaction styles and content to user preferences or habits.
\end{itemize}

However, these benefits often come with important \textit{drawbacks}~\cite{Lavie:2010} when adaptation remains opaque to the user~\cite{Eloi:2024}:
\begin{itemize}
  \item \textbf{Loss of control:} the system may make changes that the user does not understand or did not request, creating a feeling of being ``acted upon'' rather than in charge.
  \item \textbf{Lack of predictability:} if users cannot anticipate when and how the UI will adapt, they may struggle to form stable mental models of the interface.
  \item \textbf{Opacity and ``black-box'' behavior:} adaptation logic frequently operates behind the scenes; users cannot see which rules or models are responsible for changes, nor inspect alternative options.
  \item \textbf{Mismatches and errors:} when system assumptions about relevance are wrong (\eg due to noisy context sensing or biased models), critical information can be hidden or distorted without the user noticing.
  \item \textbf{Erosion of trust:} repeated unexpected changes and unexplained system decisions can reduce user confidence in the system’s reliability and fairness \cite{Bellotti:2001}.
\end{itemize}

Traditional UIs are largely static~\cite{Vanderdonckt:2020}: they require manual reconfiguration if the context changes, placing the burden on users to adapt themselves to the interface. AUIs remove part of this burden but introduce a new problem: because adaptation logic is automated and often invisible, users frequently have no clear way to observe or influence it. The UI might change suddenly without explanation and users are expected to simply accept these changes. In low-stakes scenarios this may remain a mild annoyance (``Why did the font size change on its own?''), but in high-stakes contexts the lack of control can be detrimental.

\paragraph{Motivating scenarios.}
Consider a crime investigation dashboard that uses machine learning to automatically highlights certain suspects. If the system’s AUI misjudges relevance and filters out a crucial piece of evidence, the investigating officer might not realize what has been omitted, nor have any direct means to correct or override the system’s decision. Likewise, imagine an ambient smart room where mid-air gestures are used to control IoT devices: a system that remaps a command to another gesture without informing the user, could result to frustration or even unsafe situations.

These scenarios highlight a critical gap in today’s adaptive systems: \textbf{users lack agency and insight into the adaptation process when it matters most}. Users need ways to see what the UI is doing (\textit{observability}) and to shape, configure or veto adaptation decisions (\textit{controllability}) when automatic adaptations do not align with their intentions. Recent HCI work has underscored the importance of keeping humans ``in the loop'' of adaptation decisions to maintain transparency and trust \cite{Dessart:2011}. Without explicit support for user control, adaptive interfaces risk becoming opaque ``black boxes'' that undermine their own usability advantages. This doctoral research is motivated by the imperative to reconcile the benefits of adaptivity with the principle of user empowerment.

\paragraph{Meta-User Interfaces: interfaces about interfaces.}
One approach to addressing this problem emerged in the early 2000s with the concept of the \textit{meta-user interface} (meta-UI). A meta-UI was originally defined as ``the set of functions (and their associated user interfaces) that are necessary and sufficient to \textit{monitor and control} the state of an interactive ambient space'' \cite{Coutaz:2006,Demeure:2011}. In simpler terms, a meta-UI is an interface \textit{about} the interface: an auxiliary UI that allows users to observe what an ambient system is doing and to intervene in its configuration or behavior.

This concept was introduced to cope with the complexity of context-awareness by making the ambient system’s state and possible adaptations visible and manipulable to its users. For example, a meta-UI presents a control dashboard for managing distributed displays in a smart room, or a visualization of active modalities in a multimodal system that users can toggle on/off \cite{Vanderhulst:2009}. Early research demonstrated meta-UIs for tasks such as controlling UI \textit{plasticity} on mobile devices, managing multi-user collaborative settings, or orchestrating multi-surface interactions. By providing a ``UI for the UI'', these solutions sought to restore a degree of user agency and transparency in environments that would otherwise adapt automatically and invisibly.

Meta-UIs have thus been \textit{reasonably well explored} in the literature and have attracted a significant number of design proposals and case studies. However, as we will argue, many of these implementations actually correspond to a more specific class of interfaces that later came to be called \textit{Extra-User Interfaces}.

\paragraph{From Meta-UI to Extra-UI and Mega-UI.}
Over time, the meta-UI idea has been refined and extended. Extra-User Interfaces (Extra-UIs) have emerged as a modern evolution of the meta-UI, aiming to more explicitly empower users (and sometimes designers) to intervene in adaptive behaviors at runtime. The terminology was clarified to distinguish levels of control granularity in what Coutaz \etal called the ``(meta–extra–mega)-UI'' framework \cite{Demeure:2011}, in whic:
\begin{itemize}
  \item an \textbf{Extra-UI} provides mechanisms to observe and modify \emph{UI design decisions} at runtime by exposing the underlying UI models (for instance, allowing a user to redistribute UI components across available devices, or adjust interface parameters);
  \item a \textbf{Meta-UI} (in the updated sense) goes a level deeper, letting designers alter or extend the \emph{language or model} that defines the UI (\eg introducing new interaction modalities or rules into the UI’s specification);
  \item a \textbf{Mega-UI} is a hypothetical superset that permits observation and control of \emph{everything} in the interactive system-including the base UI, any Extra-UIs and Meta-UIs, and the relationships among them (\eg a Mega-UI could manage model-to-model transformations ~\cite{Sottet:2010} or the entire adaptation infrastructure itself) \cite{Vanderdonckt:2009}.
  \^item a \textbf{Supra-UI}
\end{itemize}

In practice, however, \textbf{most implementations that were historically called ``meta-UIs'' have in fact been instances of Extra-UIs}: additional interfaces giving end-users control over an adaptive system’s behavior, typically realized as graphical overlays, configuration panels, or control dashboards. The notion of Mega-UIs remains largely conceptual, representing an ultimate extension of the idea of user-control interfaces where the entire ecosystem of UIs-including Extra- and Meta-UIs-could be observed and manipulated.

While the Meta-UI concept has been revisited in various works, the Extra-UI notion has received comparatively little in-depth treatment: its definition, design space, and systematic design methods remain under-specified. This dissertation addresses that gap.

\paragraph{Extra-User Interfaces as focal concept.}
In line with this progression, Extra-User Interfaces serve as the focal point of our research. An Extra-UI can be seen as an interface layer that sits ``above'' or ``around'' the regular user interface of an application, with the dedicated purpose of making adaptation \textit{observable} and \textit{controllable} by users in real time. Rather than the system adapting silently and unilaterally, an Extra-UI offers tools for the user to~\cite{Melchior:2010}:
\begin{itemize}
  \item \textbf{Discover} what adaptive changes are possible or ongoing (e.g., viewing which UI elements have changed, which contextual rules fired, or which resources are available);
  \item \textbf{Remodel} or adjust the UI’s configuration (e.g., rearrange layout or swap a widget for an alternative);
  \item \textbf{Redistribute} parts of the interface across multiple screens or devices;
  \item \textbf{Parameterize} adaptive behaviors (e.g., tuning the sensitivity of a gesture recognizer or the threshold of an alert);
  \item \textbf{Extend} the interface with new functionality or content on the fly.
\end{itemize}

The five aforementioned capabilities emerge as core primitives of Extra-UIs and will be formally defined and positioned in a broader design space later in this thesis. By offering such services, Extra-UIs aim to strike a balance where the system can still perform automated adaptations (leveraging context-awareness and AI as appropriate), but the user is never a helpless bystander. Instead, the user gains oversight and veto power: they can monitor adaptation processes, intervene to correct or personalize outcomes, and thus remain in control of their interaction.

This restoration of user agency and insight is not merely a convenience; in safety-critical or mission-critical applications, it can be essential. In dynamic environments and high-stakes contexts-such as mid-air gestural control systems in interactive command centers, or investigative data visualization tools aiding police work-an Extra-UI can ensure that human operators have the final say and a clear understanding of interface changes that could impact their decisions. In summary, the context for this thesis is the challenge of adaptive user interfaces in ambient intelligent systems, and the motivation is to enhance those systems with a new class of meta-level interfaces-Extra-User Interfaces-that put users back in charge of UI adaptation. This sets the stage for a deeper investigation into what Extra-UIs are, how to design them, and how they improve the user experience of adaptive systems.

%\bigbreak

\section{Research Question}

Current AUIs too often treat the user as a passive subject of adaptation rather than an active controller of it. The core \textbf{research problem} addressed in this dissertation is therefore the lack of user control and transparency in automatic user interface adaptations. As adaptive and context-aware technologies proliferate, users frequently encounter situations where the software adjusts itself (its layout, content, or behavior) without their explicit consent or understanding. The origin of this problem lies in a design paradigm that prioritizes full automation of adaptation: the system monitors context and applies predefined adaptation rules or machine-learning-driven changes, while the user’s role is reduced to observing the outcome. Important UI changes may be hidden or insufficiently explained, and if the adaptation is inappropriate or undesired, the user has no straightforward means to alter it. This can lead to mismatches between the system’s assumptions and the user’s actual needs, undermining usability and trust. In short, adaptive systems have a tendency to violate the principle of predictability and controllability that is essential for user trust. Addressing this problem requires a new approach that reintroduces user agency into the adaptation loop.

This thesis tackles this problem through the lens of Extra-User Interfaces. There are two tightly coupled facets to this endeavour:
\begin{enumerate}
  \item \textbf{A general conceptual facet:} defining, characterizing, and structuring Extra-UIs as a concept in their own right, independently of a particular application domain;
  \item \textbf{An applied facet:} studying how Extra-UIs can be instantiated in adaptive and context-aware systems to restore transparency and user control over UI adaptation.
\end{enumerate}

The overarching research question guiding this work is:
\begin{quote}
\textit{How can we define, design, develop, and validate an Extra-User Interface that enables users to understand and control UI adaptations in context-aware interactive systems?}
\end{quote}

In other words, how can we empower end-users with an ``interface for controlling other interfaces'' so that adaptive behavior becomes user-steerable rather than opaque and system-driven? To answer this broad question systematically, we adopt the Goal–Question–Metric (GQM) approach \cite{Basili:1984}, which breaks down the primary goal into specific research questions and further into measurable objectives. Following the GQM model, our main goal (empowering users in adaptive UI contexts via Extra-UIs) is decomposed into three key research questions:

\begin{description}
\item[\textbf{Q1: What is an Extra-UI?}] What are the fundamental concepts, definitions, and scope of an Extra-User Interface? This question seeks to establish a formal definition of Extra-UIs and to delineate their design space, independently of a specific application. Answering Q1 involves identifying the core properties and primitives of Extra-UIs (e.g., the capabilities to rediscover, remodel, redistribute, parameterize, and extend a UI) and ensuring that this definition comprehensively covers the range of user-driven adaptation operations needed for interactive systems. The outcome of Q1 will be a clear conceptual framework that distinguishes Extra-UIs from related concepts (such as traditional Meta-UIs) and articulates the role Extra-UIs play in an interactive system.

\item[\textbf{Q2: How to design and develop an Extra-UI?}] What methods and tools are required to create an Extra-User Interface and integrate it into an existing adaptive system? This question addresses the methodological and engineering challenges of Extra-UIs. It encompasses designing the Extra-UI’s functionality and user experience (what services it provides and how users interact with them), as well as the architecture needed to support those services at runtime. Q2 entails developing a design method or process-likely model-based-to guide practitioners in implementing Extra-UIs. It also involves ensuring that the Extra-UI’s design aligns with a continuous adaptation loop (the feedback cycle through which the system senses context, adapts the UI, and incorporates user adjustments). Concretely, answering Q2 will produce a set of design principles, an architectural model, and possibly a notation or toolkit (for example, using the Software Process Engineering Metamodel, SPEM) that together enable systematic development of Extra-UIs within adaptive systems.

\item[\textbf{Q3: How to evaluate the usefulness and usability of an Extra-UI?}] As soon as an Extra-UI is built, how can we empirically assess its impact on user performance and experience? Q3 focuses on the evaluation framework for Extra-UIs. We must determine whether introducing an Extra-UI actually improves the situation for users of adaptive interfaces: Do users achieve better task outcomes or efficiency when they have an Extra-UI available? Do they feel more satisfied, in control, and confident about the system’s behavior? Answering Q3 involves identifying appropriate evaluation criteria and metrics-for instance, usability scores (such as System Usability Scale or User Experience Questionnaire results), task success and error rates, adaptation effectiveness measures, and subjective perceptions of transparency and controllability. It also involves designing user studies or experiments that compare adaptive systems \emph{with} versus \emph{without} Extra-UI support. The goal is to gather evidence on whether Extra-UIs deliver tangible benefits (and at what cost, if any, in terms of added complexity or interaction overhead).
\end{description}

These three research questions (Q1–Q3) form a coherent inquiry path: first establishing the concept of Extra-UI, then providing a method to realize it, and finally validating its value. They are directly aligned with the dissertation’s aim of tackling the lack of user control in adaptive UIs, and they will be revisited throughout the thesis. Each question is associated with specific goals and metrics (per the GQM approach) that guide our investigation and ensure that we can measure success. In summary, the research problem is rooted in the tension between automatic UI adaptation and user agency, and our questions seek to resolve this by conceptualizing a solution (Extra-UIs), operationalizing its design, and empirically measuring its effectiveness.

\section{Objectives and Contributions}

To address the research questions above, this Ph.D.\ work pursues several specific objectives and makes corresponding scientific contributions. The overarching goal is to establish the concept of the Extra-User Interface as a viable approach for user-controllable UI adaptation, and to demonstrate its feasibility and benefits through theoretical development and practical experimentation. The main contributions of this dissertation are outlined below.

\paragraph{A theoretical framework for Extra-User Interfaces.}
We develop a formal definition and conceptual framework for Extra-UIs, solidifying what an Extra-UI is and what it entails. This includes identifying the fundamental primitives or services that an Extra-UI can provide to end-users. In particular, the framework defines a set of five core Extra-UI capabilities: \textit{rediscover}, \textit{remodel}, \textit{redistribute}, \textit{parameterize}, and \textit{extend}. These primitives encapsulate the different ways an Extra-UI can intervene in an interface-from discovering the current state or available resources, to modifying the UI’s composition or distribution, to tuning parameters of adaptive behavior, to extending the interface’s functionality. By articulating these operations, the framework creates a common language to describe and compare Extra-UIs. It also positions Extra-UIs in relation to prior concepts (clarifying distinctions with Meta-UIs and Mega-UIs) and lays out a taxonomy or design space of Extra-UIs. This theoretical contribution builds on earlier work in user interface plasticity and Meta-UI design \cite{Demeure:2011}, and extends it by explicitly focusing on end-user empowerment and real-time control.

\paragraph{A model-based design method for integrating Extra-UIs into adaptive systems.}
We propose a structured methodology to guide designers and developers in implementing an Extra-UI as part of an adaptive user interface system. This contribution is methodological and is grounded in model-based user interface engineering principles \cite{Vanderdonckt:2008}. It consists of a process model and architectural blueprint for adding an Extra-UI layer to a system’s UI. Key elements include: (a) the use of behavioral and UI models to represent both the base adaptive UI and the Extra-UI, ensuring that the Extra-UI has access to the underlying UI’s state and adaptation logic; (b) integration of the Extra-UI into a continuous adaptation loop, meaning the system continuously senses context and user inputs, adapts the UI, and incorporates any adjustments made via the Extra-UI in a closed feedback cycle; and (c) a method-engineering approach using a standard notation (such as SPEM 2.0) to formally describe the design steps and roles involved in creating the Extra-UI. In practice, this design method provides guidelines on issues such as how to decide which adaptive aspects to expose to the user and in what form, how to maintain consistency between the Extra-UI and the primary UI, and how to handle user-initiated adaptation commands alongside automatic adaptations.

\paragraph{Implementation and evaluation of two Extra-UI prototypes.}
As proof-of-concept and to validate the proposed ideas, we have designed, implemented, and empirically evaluated two distinct prototypes of Extra-User Interfaces in real-world-inspired scenarios. The first prototype is a gestural Extra-UI for ambient environments: it builds upon a mid-air gesture-controlled smart space (using a Leap Motion sensor and the QUANTUMLEAP framework) and adds an Extra-UI that allows users to inspect and modify gesture-to-command mappings on the fly. The second prototype is an adaptive data visualization dashboard for investigative analysis (in the domain of criminal investigations), augmented with an Extra-UI panel that gives investigators control over the system’s adaptive visualization behavior. Both prototypes have been evaluated with target users through user studies, comparing usage of the adaptive system \textit{with} versus \textit{without} the Extra-UI. The results of these evaluations (detailed in later chapters) provide empirical evidence regarding the usefulness and usability of Extra-UIs.

In summary, the contributions of this thesis are: (1) a clarified concept and design space for Extra-User Interfaces, complete with defined primitives and taxonomy; (2) a novel model-based methodology for designing and integrating Extra-UIs into adaptive systems; and (3) two implemented Extra-UI systems with corresponding evaluations that evidence the benefits of re-introducing user control into UI adaptation.

\section{Working Hypotheses}
% To be detailed in a later iteration; this section will state the hypotheses derived from Q1--Q3.

\section{Methodology Overview}

Investigating Extra-User Interfaces requires a multi-layered research methodology that spans from conceptual foundations to practical validation. In this dissertation, we adopt a three-tier methodological approach, with each layer addressing one of the core aspects of the research (conceptualization, design, and evaluation). The approach can be summarized as follows:

\begin{enumerate}
\item \textbf{Conceptual layer – Definition and Taxonomy of Extra-UIs:} We begin at the theoretical level by formally defining what constitutes an Extra-User Interface and constructing a taxonomy of Extra-UIs. This involves a literature analysis of existing adaptive UI systems and Meta-UI examples, in order to extract common themes and capabilities that inform the Extra-UI definition. We identify the key services an Extra-UI should provide (the primitives introduced earlier) and specify how these services relate to underlying UI models and adaptation mechanisms. We also categorize Extra-UIs along meaningful dimensions-for example, distinguishing whether an Extra-UI is embedded within the primary UI or exists as a separate application, whether it manipulates digital content or physical devices, etc. The outcome of this layer is a clear ontological description of Extra-UIs and a structured design space that maps out possible variations of the concept. This conceptual groundwork answers Q1 (What is an Extra-UI?) by delivering a rigorous definition and classification, which then guides all subsequent work.

\item \textbf{Methodological layer – Model-Based Design and Integration Process:} Building on the concept definition, the next layer develops a methodology for designing and integrating Extra-UIs into adaptive systems. We utilize a model-based engineering approach, employing modeling techniques to represent both the interactive system and the Extra-UI. In particular, we use the SPEM (Software Process Engineering Metamodel) framework to model the development process of an Extra-UI, and we incorporate the notion of a continuous adaptation loop (a cyclical process of context sensing, adaptation decision, UI update, and user feedback) into the system architecture. This layer produces a model-based design method (answering Q2: How to design/develop an Extra-UI?) that serves as a blueprint others can follow.

\item \textbf{Implementation \& Evaluation layer – Prototyping and Empirical Studies:} The final layer is highly practical: we implement concrete Extra-UI prototypes and conduct user evaluations to test our hypotheses. We instantiate the conceptual framework and design method in two case studies (the gestural interaction scenario and the investigative visualization scenario). We then perform empirical evaluations by inviting representative users to use these systems in controlled settings, gathering both quantitative data (e.g., task completion times, error rates) and qualitative feedback (e.g., SUS/UEQ scores, interviews about perceived transparency and control). Comparative study designs isolate the effect of the Extra-UI by contrasting conditions with and without Extra-UI support. This layer directly addresses Q3 (How to evaluate an Extra-UI?) and feeds back into refining our framework and method.
\end{enumerate}

Through this three-layer approach, the methodology covers the full spectrum of the research: from abstract theory to concrete design prescriptions to validated outcomes. It ensures that our work on Extra-UIs is both conceptually sound and empirically substantiated.

\section{List of Contributions}
% This section will later provide a compact, bullet-point summary of the contributions detailed above (framework, method, prototypes and evaluations).

\section{Manuscript Overview}

The remainder of this manuscript is organized as follows:

\begin{itemize}
  \item \textbf{Chapter 2 – Related Work:} introduces the state of the art on adaptive user interfaces, UI plasticity, Meta-UIs and related concepts, with a particular focus on how prior work has (or has not) addressed user control and transparency in adaptive systems.
  \item \textbf{Chapter 3 – Contextual Adaptation and Extra-UI Design Space:} presents the theoretical background on contextual adaptation and synthesizes existing work (including the encyclopedia chapter with Calvary) into a structured set of concepts that prepare the ground for Extra-UIs.
  \item \textbf{Chapter 4 – Concept and Design Space of Extra-User Interfaces:} defines Extra-UIs formally, introduces their core primitives, and elaborates a design space and taxonomy that distinguish Extra-UIs from Meta-UIs and other related notions.
  \item \textbf{Chapter 5 – Model-Based Methodology for Extra-UI Development:} details the proposed model-based process and architectural patterns for integrating Extra-UIs into adaptive systems, using SPEM and related modeling tools.
  \item \textbf{Chapter 6 – Prototypical Implementations and Evaluations:} describes the two Extra-UI prototypes (gestural Extra-UI and investigative dashboard Extra-UI), the design decisions behind them, and the empirical studies conducted to evaluate their usefulness and usability.
  \item \textbf{Chapter 7 – Discussion and Conclusion:} synthesizes the results, discusses limitations and generalizability, and outlines perspectives for future research on Extra-User Interfaces and user-controllable adaptation.
\end{itemize}

\chapter{Related Work}
\label{chap:Related}
% Insert here the chapter along with remaining entries
Adapting UIs to their contexts of use aims at meeting both evolving requirements, such as the needs, desires, and preferences of individual users or user groups, and evolving resources, such as the available interaction devices in the surroundings, that all may be dependent on the current situation. There are two categories of adaptation, whether the end user or the application is responsible for the modification~\cite{Gajos:2017}. \textit{Adaptability} refers to the end user's ability to adjust the UI, while \textit{adaptivity} or \textit{self-adaptation} denotes the application's ability to perform UI adjustments \cite{Browne:1990}. AUIs refer to user interfaces benefiting from adaptivity.
\textit{Personalisation} or \textit{customisation}, a subset of adaptivity, focuses on adapting the UI contents, presentation, and behaviour based solely on end-user data, such as personal traits~\cite{Gajos:2017}. When data comes from external sources, like other user groups, it results in \textit{recommendations} instead. \textit{Mixed-initiative adaptation} occurs when both the end user and the application collaborate to make adaptations~\cite{Horvitz:1999}.

The ultimate goal of any UI adaptation is to make the UI better from the end user perspective to satisfy him/her as much as possible, by optimising factors such as efficiency (\eg reducing the task execution time and error rates or improving the learning curve), effectiveness (\eg by ensuring full task completion), and subjective user satisfaction. This goal is commendable, but it carries some risk because adaptation is multifactorial and not all its impacts have been thoroughly researched.
Benefiting from the adaptation resulting from the improvement of certain factors comes at the price of suffering the resulting disadvantages (\eg user disruption~\cite{Hui:2009}), which may have an impact on factors other than those initially considered.

The challenge lies in suggesting and/or executing the appropriate adaptation at the right time and place to provide value to the end user~\cite{Alvarez:2009}. Otherwise, adaptation may encounter limitations that hinder expected benefits~\cite{Lavie:2010}, including the risk of misfit, user cognitive disruption, lack of prediction, lack of explanation, lack of user involvement, and privacy risks \cite{Dwork:2014}. Therefore, adapting UIs means responding adequately to the following questions of the Quintilian hexameter~\cite{Motti:2013}:

\begin{figure}
    \centering
    \includegraphics[width=\textwidth]{Images/Content.pdf}
    \caption{Example of contents adaptation: before \textcircled{1} and after \textcircled{2}.}
    \label{fig:content}
\end{figure}

\begin{itemize}
\item \textit{What to adapt?} Specifies the type of resource or aspect that is subject to adaptation: contents (\eg text, images, videos), presentation (\eg format of UI elements, position, size, arrangement), or behaviour (\eg navigational flow, activation, and deactivation of features, promotion or demotion of widgets~\cite{Bouzit:2019} - see \autoref{fig:promotion}). For example, \autoref{fig:content} shows an adaptation of a weather forecast application \textcircled{1} with more data depending on the user's location \textcircled{2}. \autoref{fig:presentation} shows an adaptation of the presentation where the background colour is adapted from daytime \textcircled{1} to nighttime \textcircled{2}. \autoref{fig:behavior} shows three examples of behaviour adaptation depending on desktop \textcircled{1}, tablet \textcircled{2}, and smartphone \textcircled{3}.

\item \textit{Why to adapt?} Specifies the main goals that the adaptation should reach and satisfy, preferably expressed as software qualities (\eg support the transition from a novice user to an expert user, maximise the learnability) and explains how the adaptation can contribute to achieving these goals (\eg this adaptation is expected to reduce the cognitive load). 

\item \textit{How to adapt?} Specifies according to which method, technique, algorithm, or strategy the adaptation will be performed (\eg a technique of changing the video quality, a decision tree to select the most preferred widget or menu type). This chapter will define an adaptation operation to specify how an adaptation action could be operationalised based on its event and condition for a certain value.

\item \textit{With regard to what?} Specifies contextual information related to the user (\eg preferences, skills, level of experience) and interactive tasks (\eg what type of task is subject to adaptation), to the computing platform (\eg device characteristics, platform features, screen resolution, interaction capabilities), and to the environment (\eg location, level of light, noise, stress) with regard to which the adaptation is justified and performed. For example, adapt the UI to colour-blind users, to a set of devices, or to a noisy environment.

\item \textit{Who is controlling the adaptation?} Specifies the actor that initiates, triggers, or is in charge of each stage of the adaptation (\eg the end user, the application, the UI, a third party such as a proxy, or any combination). This chapter addresses this question by referring to the seven stages of the life cycle.

\item \textit{When to adapt?} Specifies the temporal state in which the adaptation is performed (\eg at design time, run time, or compilation time). For example, the adaptation moment is at run-time whether it is for adaptability or adaptivity.

\item \textit{Where to adapt?} Specifies the physical location where the adaptation is computed (\eg based on the software architecture at the client, at the proxy, or on the server). For example, an adaptation can be performed on the server side when its logic is computationally demanding and cannot be operated at the UI level. 
\end{itemize}

\begin{figure}
    \centering
    \includegraphics[width=\textwidth]{Images/Presentation.pdf}
    \caption{Example of Presentation adaptation: daylight \textcircled{1} and nightlight \textcircled{2}.}
    \label{fig:presentation}
\end{figure}

\begin{figure}
    \centering
    \includegraphics[width=\textwidth]{Images/Behavior.pdf}
    \caption{Example of behaviour adaptation: desktop \textcircled{1}, tablet \textcircled{2}, and smartphone \textcircled{3}.}
    \label{fig:behavior}
\end{figure}


\section{The User Interface Adaptation Life Cycle}
\label{sec:lifecycle}
The UI adaptation life cycle can be decomposed into seven stages~\cite{Abrahao:2021,Lopez:2007} (\autoref{fig:isatine}):

\begin{figure}
    \centering
    \includegraphics[width=\textwidth]{Images/Isatine.pdf}
    \caption{The seven stages of user interface adaptation~\cite{Lopez:2007}.}
    \label{fig:isatine}
\end{figure}

\begin{enumerate}
\item \textit{Specification of the adaptation goals}: this first stage states the goals that UI adaptation should pursue, maintain, and update by any entity involved, such as the user ($U$), the application or system ($S$), a third-party ($T$), or any combination of them. Various contextual aspects are considered, such as the user profile, the task at hand, the computational platform (both hardware and software), and the entire physical and organisational environment in which the task is performed. The goals fall into three categories: self-expressed, application-expressed (locally or remotely), or according to their location. When the application is tolerant of faults depending on conditions, a machine-expressed goal is specified.


\item \textit{Specification of initiative for adaptation}: this second stage specifies which entity is responsible for initiating the adaptation: the user (explicit initiative), the application (implicit initiative that detects a change in the context of use requiring adaptation), or both jointly (as a decision taken by the entities in control). When the machine initiates an adaptation that the user subsequently cancels, a mixed initiative occurs \cite{Horvitz:1999}. This stage is further refined into an adaptation request, a detection of an adaptation need, and a notification for an adaptation request, depending on their location. 

\item \textit{Specification of adaptation}: this third stage specifies which entity is responsible for deciding that an adaptation should occur. After the adaptation is initiated, a set of adaptation proposals is created that contains zero, one, or many proposals. These proposals can be further elaborated by the user, the application, or a third party, such as a broker.  If the application is responsible for creating adaptation proposals, then adaptation rules generate proposals and even infer new rules. Once the set of proposals is obtained, a decision must be made to accept (\eg which adaptation proposal or which ones are the most appropriate), to reject (\eg if no proposal corresponds to the goals), to cancel (\eg no adaptation is finally needed), to defer (\eg by the next session), or to propose a new set.


\item \textit{Application of adaptation}: this fourth stage specifies which entity is responsible for applying the adaptation decided in the previous stage, if any. Since the adaptation will be applied to the UI, it should incorporate a mechanism that supports adaptation (\eg via an API for parameterisation \cite{Cockton:1987}). The user adapts the UI (\eg through UI options, personalisation) or the application does it on behalf of the user. For example, transformations apply various presentation adaptations of a GUI to multiple platforms with various resolutions~\cite{Furtado:2001,Aquino:2010}. 

\item \textit{Transition with adaptation}: this fifth stage specifies which entity is responsible for ensuring a smooth transition between the UIs before and after adaptation. For example, an application uses some visualisation techniques to present the intermediary steps executed (\eg by progressive rendering \cite{Rogers:1999} or by animated transition, by morphing~\cite{Dessart:2011}).


\item \textit{Interpretation of evaluation}: this sixth stage specifies which entity is responsible for providing information meaningful for understanding the adaptation. When the application performs some adaptation without any explanation, the user does not necessarily understand why this type of adaptation has been performed. Conversely, when the user has performed some adaptation, the user should tell the application how to interpret this adaptation. For example, a Machine Learning (ML) algorithm first proposes some adaptation to be applied~\cite{Eisenstein:2000}; if this adaptation does not correspond to the goals, the user produces an alternate adaptation and informs the application how to incorporate this new scheme for future use. The machine updates the knowledge base by interpreting this explanation. User-control of the adaptation is key and can be supported by unsupervised learning~\cite{Eloi:2024}.

\item \textit{Evaluation of adaptation}: this seventh stage specifies which entity is responsible for evaluating the quality of the adaptation performed to check whether the initially-specified goals are met. For example, if the application maintains some internal plan of goals, it should update this plan according to the adaptations applied so far. If the goals are in the users’ mind, they could also be evaluated with respect to what has been performed in the previous stages. In this case, the explanation of the adaptation contributes to updating the goals’ status too. For example, \autoref{fig:evaluation} shows how a previously-applied adaptation can be evaluated: the adaptation operation applied is displayed and subject to evaluation by the system itself (based on internal analysis) or by the end-user, which can be qualitative (\eg by a rating bar) or quantitative (\eg by a rating scale).
\end{enumerate}

\begin{figure}
    \centering
    \includegraphics[width=\textwidth]{Images/Evaluation.PNG}
    \caption{Example of adaptation evaluation in \href{https://symbiotik-infovis.eu/}{Symbiotik}: by the user and/or the system.}
    \label{fig:evaluation}
\end{figure}

\section{Targeted Literature Review on User Interface Adaptation}
\label{sec:TLR}
Despite the significant progress in AUIs, several challenges persist, which complicate the broader adoption of these systems and their development. Instead of performing a systematic literature review, this section prefers a targeted literature review that summarises the literature on UI adaptation, considering the most relevant and representative references in the domain. 

\subsection{Existing and Future Challenges in User Interface Adaptation}

This section therefore discusses a limited set of challenges that are still active and open today for UI adaptation.


\begin{enumerate}[
    topsep=5pt,
    itemsep=1pt,
    parsep=0pt,
    label={$C_{\arabic*}.$},
    leftmargin=0.675cm
]

 \item \textbf{Complexity in user modelling.} Building precise and thorough user models that can anticipate and adjust to each user's demands is still challenging. These models must account for the diversity of human behaviour and preferences, which are shaped by a wide range of elements such as the user's present emotional state, personal experience, and cultural background. Because of this complexity, models frequently oversimplify the user or become too complicated to be used in real-time adaptivity~\cite{Ontanon:2021}. Several factors influence the complexity in user modelling.
 Human behaviour is intrinsically highly variable and influenced by numerous external and internal factors. Capturing this variability in a user model requires sophisticated algorithms and extensive datasets~\cite{AlSeraj:2018}. Various user modelling approaches, such as GOMS models, cognitive architectures, grammar-based models, and application-specific models, have varied levels of complexity and are appropriate for different types of applications~\cite{Castillejo:2014}. The complexity increases while selecting and applying the appropriate strategy in the appropriate situation. End-users interact with AUIs in various contexts of use~\cite{Calvary:2002} that change over time, such as different tasks, devices, platforms, and environments~\cite{Calvary:2003}. Adapting to these dynamic contexts of use adds another layer of complexity~\cite{Coutaz:2006}. Cultural background, personal experiences, and personal traits~\cite{Gajos:2017} significantly influence user preferences and interactions. Models must be sensitive to these differences to provide meaningful adaptations~\cite{Castillejo:2014}. User models used for adaptation may initially be correct or not and become outdated over time due to changes in external factors. Identifying, correcting, and updating these models while the context is evolving is even more complex. Mechanisms need to be in place to detect and address inaccuracies in the user models~\cite{AlSeraj:2018}.
  Environmental factors in the context of use are often underestimated if not overlooked. The environment in which the user operates~\cite{Dubiel:2022}, including factors like traffic, noise, light, pressure, and the presence of other users, especially in collaborative configurations, significantly impacts user behaviour and the effectiveness of AUIs.

 Due to all these reasons, obtaining an accurate AUI is no longer a matter of predefined adaptation rules, heuristics, and models, but a matter of taking the evolution into play.

\item \textbf{Balancing adaptation with user control.} While AUIs aim to provide a highly adapted user experience~\cite{Abrahao:2021}, there is a fine line between helpful adaptation and perceived intrusion. Users may feel a loss of control over the UI when changes are too frequent, too disruptive~\cite{Todi:2021}, unexpected~\cite{Lavie:2010}, or disturbing~\cite{Hui:2009}. Ensuring that adaptations enhance user experience without undermining user autonomy is a crucial design consideration that remains largely unresolved. A totally automatic adaptation is certainly disrupting; a totally manual adaptation is probably even more disrupting as the adaptability cost is too expensive for the end-user. The aim of possessing a good AUI is to seem finally as if adaptation lies in mixed-initiative~\cite{Horvitz:1999} control. 

For example, \textsc{MyUI}~\cite{Peissner:2013} tested the effectiveness and acceptability of adaptation patterns in different cost-benefit situations and for different users. The patterns turn out to increase the transparency and controllability~\cite{Gajos:2017} of adaptations as soon as they are under user control. Preference and acceptance of the patterns depend on the cost-benefit condition. In the same vein, \textsc{Scaler}~\cite{Eloi:2024} supports user control by unsupervised learning by enabling end-users to personalise the form widgets based on their preferences or based on the system.

\item \textbf{Integration with existing systems.}
Integrating advanced adaptive mechanisms into existing technological ecosystems poses technical and operational challenges. Compatibility with legacy systems, scalability, and maintaining consistent performance across different platforms are significant hurdles for developers of AUIs~\cite{Abrahao:2021}. Software engineering techniques, such as reverse engineering of UIs~\cite{Bouillon:2005} are helpful, but will always require some manual code tweaking. 

\item \textbf{Privacy and security.}
To provide personalised \cite{Ontanon:2021} and context-aware services \cite{Crowley:2002}, AUIs collect a variety of user data, such as demographic data, location history, preferences, and even psychological states. Deep data collecting, however, presents privacy issues \cite{Dwork:2014,Hazard:2016} because private data may be exposed and misused. For example, an AUI can improperly determine a user's home location based on repeated visits \cite{Beresford:2003}. 
Additionally, AUIs influence user behaviour through subtly nudging them, such as encouraging in-app purchases during decision-weariness-prone moments. This is especially true when it comes to games that use subtle cues to encourage in-app purchases, which is particularly remarkable in some games \cite{Teachout:2016}. This could force end-users to buy at times when they are most likely to be vulnerable, including when they are tired from making decisions. This is an illustration of indirect manipulation of autonomy, in which the interface is made to sway the choices made by the user according to information gathered about their actions and emotional state \cite{Hazard:2016}. 
Ethical design principles should put user liberty and privacy first to reduce these hazards and protect personal data \cite{Zyskind:2015}.

Differential privacy is one promising strategy that preserves individual privacy while enabling user data analysis. Differential privacy lowers the likelihood of privacy breaches by adding "noise" to the data, which makes it harder to identify particular users \cite{Dwork:2014}. Blockchain technology can also be used to improve data usage accountability and transparency. Users can have more insight into how and by whom their data is being used by keeping track of data transactions on a decentralised ledger. This helps foster trust and prevent data misuse \cite{Zyskind:2015}.

\item \textbf{Handling wrong, outdated, and inadequate data in adaptivity.}
AUIs heavily rely on user models that are intended to anticipate and react to unique demands, preferences \cite{Eslami:2018}, and behaviours \cite{Cockton:1987}. These models frequently encounter difficulties because of errors resulting from a variety of sources \cite{Abrahao:2021}. User preferences and behaviours may not be fully captured, which might result in incorrect assumptions when building user models. For example, Netflix's recommendation engine begins with user inputs, such as preferred genres, and improves over time by analysing movie viewing patterns to provide more precise recommendations \cite{Gomez-Uribe:2015}. To avoid these models becoming out of date, they should be regularly updated with new data to preserve the relevance and responsiveness of AUIs. For this purpose, methods like data mining and categorisation ensure that the AUI adapts to suit the user's evolving needs \cite{Jalil:2021}.

For example, in a smart home setting, an AUI may adjust lighting settings in response to user behaviour. If the user's schedule changes, the model may become unaligned.
The system can determine the requirement for model upgrades by identifying more frequent manual illumination adjustments \cite{Mahdavinejad:2018}.
Furthermore, due to privacy restrictions, incomplete data collection, or difficulties in predicting specific behaviours, the data available for creating user models may occasionally be insufficient, thus resulting in user dissatisfaction, decreased efficiency, and possible disengagement \cite{Abrahao:2021}. 
\end{enumerate}

\vspace{-8pt}
\subsection{Recent Advances in User Interface Adaptation}

In response to the aforementioned challenges, recent advances in technology and methodology have been made to enhance the functionality and applicability of AUIs:

\begin{enumerate}[topsep = -3pt, itemsep = 1pt, parsep=0pt, label={A$_\arabic*$.}, leftmargin=0.675cm]
\item \textbf{Enhanced machine learning techniques.}
Recent developments in ML, Deep Learning (DL), and neural networks have provided new ways to process and analyse large datasets more effectively for AUIs. These advancements enable more nuanced understanding and prediction of user behaviour, improving the accuracy of user models. For example, techniques like reinforcement learning~\cite{Todi:2021,Gaspar:2024} dynamically adapt AUIs based on user interactions, learning from each user’s preferences to optimise the UI configuration over time. This way of supporting adaptation induces two sets of operations between the end-user and the AUI~\cite{Bouzit:2017:PDA}: a Perception-Decision-Action (PDA) cycle for the user followed by a Learning-Prediction-Action (LPA) cycle for the AUI.

\item \textbf{Multimodal interaction.}
Advancements in multimodal interfaces, which combine inputs from various sources, such as voice, touch, and even gaze, offer new ways to understand user intentions and context more holistically~\cite{Blumendorf:2010}. These AUIs can adapt more effectively to the user’s current state and preferences by integrating different types of data, leading to a more intuitive and seamless user experience. In particular, the modality can be adapted to environmental conditions \cite{Josifovska:2019} while allowing the end user to choose among the available modalities which is the most adapted. For example, polymodal menus~\cite{Bouzit:2017} enable end-users to select menu items graphically, vocally, or gesturally depending on their preferences and their context of use.

\item \textbf{Context-aware adaptation.}
Progress in sensor technology and context-aware computing allows AUIs to adapt not only to the user, but also to the entire context of use at once~\cite{Motti:2013,Yigitbas:2016,Yigitbas:2020}. AUIs can now consider factors such as location, time of day, and even the presence of other people to tailor user experiences. For example, a UI might simplify its options and enlarge buttons when it detects that the user is in a moving vehicle.

\item \textbf{User-centred design approaches.}
There is a growing emphasis on involving users directly in the adaptation process. User-centred design approaches \cite{Velsen:2008} that allow users to set preferences on how and when the UI should adapt are being explored to balance automation with user control. This approach not only addresses the challenge of maintaining user agency but also helps in fine-tuning the system according to individual preferences.
Although AUIs have made remarkable progress, they continue to evolve in response to technological advances and user feedback. Overcoming existing challenges and effectively leveraging recent advances will be key to developing more intuitive, efficient, and user-friendly AUIs in the future. These advances promise not only to enhance user interaction but also to pave the way for UIs that are truly responsive to the complex and dynamic nature of human needs and behaviours.
\end{enumerate}

\vspace{-8pt}
\subsection{Computational Approaches to Adaptive User Interfaces}

 Using modern computational techniques, AUIs dynamically adjust user experiences in real-time, adapting to users' changing choices, actions, and environmental situations. The need for these UIs is growing as systems become more complicated and as users demand more customized experiences. Incorporating developed computational methods \cite{Jiang:2022} into the comprehension, creation, and modification of UIs not only improves the adaptability and responsiveness of these UIs, but also addresses important issues in human-computer interaction (HCI), including context awareness, usability, and accessibility. In order to enable AUIs, a number of generations of computational approaches \cite{Jiang:2022} from rule-based systems to machine learning algorithms and model-based techniques \cite{Hussain:2018} have been studied over time.


\subsubsection{Model-Based Development of Adaptive User Interfaces}
Model-based development (MBD)~\cite{Eisenstein:2000,Derakhshanmanesh:2019,Hussain:2018} provides a structured approach to designing AUIs that can dynamically adjust to the user's context of use. By defining high-level abstractions that represent the core functionalities and user interactions, MBD allows for the creation of flexible and adaptable systems. This section explores how MBD facilitates the development of AUIs and examines case studies to illustrate its practical applications.
MBD focuses on creating a set of models that define the UI at various levels of abstraction, from user interactions down to the specific elements of the UI design. These models include:
\begin{enumerate}
\item \textit{Domain models}: Define the data and logic specific to the application domain, ensuring that the UI aligns with the business requirements~\cite{Stocq:2004} and needs \cite{Gomez-Uribe:2015}.
\item \textit{User models}: Capture information about the users, their preferences, and behaviours to tailor the UI accordingly. These models make it possible to modify the UI in real-time, improving usability and satisfaction by customising the experience to each user's requirements. For example, the "Mining Minds" platform adapts its UI based on user context, such as increasing the font size for visually impaired users or switching to graphical modes \cite{Hussain:2018}.
\item \textit{Interaction models}: Specify how users will interact with the system, detailing the flow and dynamic aspects of the UI.
For example, interaction models can be constantly modified according to user circumstances and preferences using a Rule-Based User Interface (RBUI) method~\cite{Akiki:2016}. RBUI, which uses a rule-based framework, allows real-time UI adaptation by considering many variables, including user experience, device capabilities, and ambient conditions. This method closely adheres to MBD principles while also improving the user experience and facilitating a more customised interaction flow.
\item \textit{Presentation models}: Describe the UI rendering (\eg the graphical components in the case of graphical UIs) while making sure that they are consistent and easy to use on various platforms \cite{Aquino:2010} and UI types \cite{Martinez:2006}. These models are made to adapt dynamically to changes in context, such as different screen sizes and resolutions, while also retaining a consistent look and feel when users switch between different devices, like desktops, tablets, and smartphones. For example, \textit{graceful degradation} exploits a presentation model to adapt an existing UI for a certain platform to a new UI for a smaller platform (\autoref{fig:graceful}) \cite{Florins:2004}. The opposite operation is called \textit{progressive enhancement}.
This flexibility offers a productive user experience, especially when the AUI instantly adapts to changing user preferences or environmental circumstances. For example, the UI of a travel planning application adapts based on the user's context~\cite{Abrahao:2021}: a business traveler in a busy airport using a smartphone might see a simplified layout with larger buttons to facilitate quick interactions, while a tourist using a laptop in a quiet hotel room might experience a more detailed UI, taking advantage of the larger screen and calmer environment. 
\end{enumerate}

\begin{figure}
    \centering
    \includegraphics[width=\textwidth]{Images/Graceful.pdf}
    \caption{Graceful degradation of a desktop UI to a mobile~\cite{Amouh:2005}.} 
    \label{fig:graceful}
\end{figure}

By abstracting these elements, MBD supports the generation of UIs that can adapt to different user requirements and operating environments without significant rework. This approach not only enhances the adaptability of the UI but also promotes reusability and scalability. The MBD advantages include:
\begin{itemize}
\item \textit{Consistency across platforms}: MBD allows developers to maintain a consistent user experience across various platforms by adapting the base model to fit different device specifications \cite{Furtado:2001}.
\item \textit{Efficiency in development}: by abstracting the UI into models, changes can be implemented more quickly and propagated across platforms automatically \cite{Montero:2006}.
\item \textit{Improved adaptability}: with models that dynamically adjust based on user data and context, the UI can respond in real-time to changes in user behaviour or environmental conditions, such as ubiquitous environments \cite{Paterno:2009}.
\item \textit{Model reusability}: once some UI or user aspects have been captured in the models, any model excerpt can be reused in other contexts of use that share some common characteristics \cite{Delgado:2016}. For example, the nomadic gestures \cite{Vatavu:2012} can be reused from one context to another.
\end{itemize}
The main MBD limitations are:
\begin{itemize}
\item \textit{Required abstraction by modelling}: the big win with MBD is that, when the model(s) change, the UI should change accordingly. This approach always requires abstracting various UI aspects and user parameters into a set of models, an approach that is not straightforward for any designer or developer \cite{Puerta:1996,Puerta:1997}.
\item \textit{Dependence on rendering engines}: UI rendering engines can produce final AUIs either by code generation or by model parsing and interpretation. In both cases, the quality of produced AUIs heavily depends on the power of such engines and on adaptation rules or operations \cite{Akiki:2014}.
\item \textit{Increased complexity}: the MBD process adds another layer of complexity on top of the adaptation process, but this is probably inevitable since adaptation cannot happen without such mechanisms \cite{Jalil:2021}.
\end{itemize}

\subsubsection{Model-Driven Engineering of Adaptive User Interfaces}
Model-Driven Engineering (MDE) is a computational approach that focuses on creating and exploiting models, which are representations of the knowledge and know-how in specific areas. In the context of AUIs, MDE allows developers to abstract and automate the generation of UIs from high-level models. These models describe the UI independent of technology, enabling the adaptive system to instantiate UIs tailored to different devices, users, or surroundings.

%remplacer le texte par la référence
For example, Yigitbas \etal \cite{Yigitbas:2020} discuss how MDE can facilitate the development of UIs that automatically adjust to user-specific data and context changes without manual intervention. By employing a model-driven approach, developers can ensure consistency across different platforms while accommodating individual user needs and preferences.

\subsubsection{Model-Free Approaches of Adaptive User Interfaces}

Model-free approaches do not assume to have any particular underlying model. For example, ACE (Adaptive CSS-oriented Engine)~\cite{Leiva:2011} automatically adapts presentation properties of widgets in a web page based on designer's input and interaction history: adapt the font size of title, labels, and contents, changing the margins,  increase the surface of push buttons to enable touch-based interaction. Adaptation of presentation properties can also be generated, either completely or partially unsupervised, according to the collective behaviour of a set of web page users~\cite{Leiva:2012,Leiva:2018}. While a representation of the UI elements subject to adaptation exists in both cases, there is no explicit user or complete UI model to support adaptation.


\subsubsection{Artificial Intelligence, Machine Learning, and Deep Learning of AUIs}
AI and ML are at the forefront of driving adaptive capabilities in UIs. These technologies enable systems to learn from data, identify patterns, and make decisions with minimal human intervention. In AUIs, machine learning algorithms can predict user behavior, adapt interfaces to new contexts, and even anticipate future needs by analyzing historical data.

The application of AI in AUIs often involves natural language processing (NLP), image recognition, and predictive analytics. For example, NLP can interpret and respond to user inputs in natural language, enhancing UI accessibility and ease of use. More recently, Reinforcement Learning (RL) of AUIs for Accessibility Context \cite{Zouhaier:2021} exploits three Knowledge Layers depending on the target adaptation: the Disability Knowledge Layer learns the behaviour of the structure of the UI depending on the disability profile, the Modality Knowledge Layer learns adaptation facilities based on the couple (UI, profile), and the Platform Knowledge Layer exploits platform knowledge to learn adaptation facilities.
RL can also train agents to learn how to adapt UIs in a specific context of use to maximise user engagement (which can be measured by electroencephalography \cite{Gaspar:2023}) by using an interaction model in a reward function \cite{Gaspar:2024}. \textsc{Marlui} \cite{Langerak:2024} engaged a user agent that mimics a real user and learns to interact via point-and-click actions with a UI agent that learns UI adaptations, to maximise efficiency by observing the user agent's behaviour.

%\textbf{Context-Aware Computing}

Context-aware computing is crucial for the development of AUIs that can intelligently \cite{Jalil:2021} adapt to the user environment \cite{Dubiel:2022}. This approach utilizes sensors, data from the device, and other sources to detect and interpret the surrounding physical and digital environment. Then it adjusts the UI accordingly, addressing factors such as location, time, ambient conditions, and even social context.

For example, an AUI might enlarge buttons on a mobile device when it detects that the user is in a shaky environment, such as riding in a vehicle. Similarly, it might switch to a dark mode in low-light conditions to reduce eye strain. 

% \textbf{User Modeling and Simulation}

% User modeling is a computational technique used to represent the behaviors, preferences, and abilities of users within a system. These models are often simulated to predict how changes to the UI will affect user performance and satisfaction. By creating detailed user models, AUIs can adapt to the specific needs and limitations of individual users, enhancing accessibility and usability.

% Simulations allow designers to test and refine adaptive interfaces under varied conditions and with virtual users before deploying them in real-world environments. This approach is particularly beneficial for assessing the usability of interfaces in scenarios that are difficult to replicate physically or where user testing might be impractical or costly.

%\textbf{Adaptive Algorithms}

Adaptive algorithms are designed to modify and control their behaviour based on feedback or changes in their operating environment. In AUIs, these algorithms adjust the UI in response to user interactions. They manage the layout, content, and functionality of the UI based on real-time data to optimise the user experience.

For example, an adaptive algorithm might alter the UI layout when it detects that the user frequently accesses certain features but ignores others~\cite{Schlee:2004}. It could bring those frequently used features to a more prominent position within the UI. For example, AUI layout and widgets constraints feed a fuzzy constraint satisfaction problem to dynamically adapt them in a graphical UI depending on the screen resolution and the window size (\autoref{fig:fwl})~\cite{Yanagida:2023}.

\begin{figure}
    \centering
    \includegraphics[width=\textwidth]{Images/FWL.pdf}
    \caption{Adaptive layout and widgets depending on screen resolution~\cite{Yanagida:2023}.}
    \label{fig:fwl}
\end{figure}

The use of computational design in AUIs represents a significant shift towards more dynamic, responsive, and personalised user experiences. By harnessing model-driven engineering, artificial intelligence, context-aware computing, user modelling, and adaptive algorithms, designers can create UIs that not only meet the diverse needs of users but also anticipate and adapt to changes in user behaviour and environmental conditions. These technologies foster a deeper integration of human factors into system design, leading to UIs that are not only more functional but also more intuitive and satisfying to use. As these computational techniques continue to evolve, the potential for creating truly intelligent and adaptive UIs will expand, offering new possibilities to improve interaction.



% \textbf{Integrated Model-Driven Development of Self-Adaptive User Interfaces} This case study discussed in "Integrated model-driven development of self-adaptive user interfaces" highlights the implementation of an MBD approach in developing a UI that automatically adjusts according to the user's current context and device capabilities. The system utilizes context-aware models to detect changes in the user's environment and adapts the UI components dynamically to provide an optimal user experience.\item 
 
\chapter{A Design Space for Extra-User Interfaces}
\label{chap:design}
%%#comment should i keep this ? 
Initially, an extra-UI covers the set of functions and their respective UI required to enable the end user to configure, control, and evaluate the state of the current UI in the domain, such as in ambient interactive~\cite{Coutaz:2007}. Like any digital service, an extra-UI:
\begin{enumerate}
\item allows objects to be manipulated,
\item offers functions whose power is intended to cover the required utility,
and
\item has qualities in response to usability requirements.
\end{enumerate}

Figure~\ref{fig:circle} sets out these three aspects with the object, power and quality dials, which we refine into six non-oriented axes:
- For the manipulated objects dial: nature of objects and type of object visualization,
- For the power quadrant: services offered and extensibility of the interaction language,
- For the qualities quadrant: level of categorization and level of user control.
In the following sections, we present each of these axes in detail.
 
\section{A Problem Space for Engineering User Interface Adaptation}
\label{sec:operation}

In the context of UI adaptation, there are many design spaces \cite{Alvarez:2009,Bouzit:2017:PDA,Vanderdonckt:2020}, taxonomies \cite{Kuhme:1992,Dieterich:1994}, and frameworks \cite{Calvary:2002,Lopez:2007,Nivethika:2013,Dubiel:2022}, each shedding light on different perspectives based on the authors' concerns. Here, we adopt a comprehensive approach for the specification of UI adaptation, while also considering the quality of adaptation. We propose a dual problem space model called \textsc{Hemispheres} \cite{Calvary:2007}, designed for the engineering of ``plastic`` interactive systems \cite{Sottet:2007,Sottet:2008,Vanderdonckt:2008,Vanderdonckt:2008:multi}, i.e. systems whose adaptation preserves quality in use over the evolution of the context of use. This problem space separates concerns into two areas:

\begin{itemize}
    \item The specification of the adaptation operation (\eg ``If the battery gets low, then migrate the UI to the nearest platform``);
    \item The specification of the adaptation implementation (\eg including when and how to adapt).
\end{itemize}

The right hemisphere (Adaptation operation) and left hemisphere (Life cycle) are responsible for these concerns.

\begin{figure}
    \centering
%    \vspace{-8pt}
    \includegraphics[width=.4\textwidth]{Images/Circle.pdf}
    \caption{The four quadrants of an adaptation operation: \textsc{condition}, \textsc{event}, \textsc{action}, and \textsc{value}.}
    \label{fig:circle}
%    \vspace{-32pt}
\end{figure}
\subsection{The Right Hemisphere for the Specification of Adaptation Operations}
The right hemisphere examines the specification of \textit{adaptation operations}, which under condition associate a response to a change in the context of use, aiming to preserve a certain set of values, such as quality factors, altogether contributing to the system's worth. The overall format of an adaptation operation is: \textsc{On Event, If Condition, Then Action for Value(s)}. For example, in the adaptation operation ``If the laptop battery gets low, when a smartphone becomes available around, then propose migrating the UI from the laptop to the smartphone``:
\begin{itemize}
    \item The \textsc{Condition} concerns the battery state of the laptop: ``If the battery gets low``;
    \item The \textsc{Event} occurs when the context of use changes with the arrival of the smartphone: ``when a smartphone becomes available around``;
    \item The \textsc{Action} proposed in response to this contextual change response is a UI migration: ``then propose migrating the UI from the laptop to the smartphone");
    \item The \textsc{Value} is implicit: the interaction continuity should be preserved.
\end{itemize}
Note that UI migration~\cite{Grolaux:2004,Frosini:2014} consists of transferring a particular UI from one platform to another while preserving its interaction state.
The key concepts of \textsc{Condition}, \textsc{Event}, \textsc{Action}, and \textsc{Value} (\autoref{fig:circle}) are classical \textsc{Event-Condition-Action} (ECA) rules enhanced with the notion of \textsc{Value}. We will now further examine, define, and develop the four quadrants of \autoref{fig:circle}.
\begin{figure}
    \centering
    \includegraphics[width=\textwidth]{Images/Condition.pdf}
    \caption{The \textsc{Condition} quadrant of our design space for adaptation.}
    \label{fig:condition}
%    \vspace{-12pt}
\end{figure}
\subsubsection{Condition}
The \textsc{Condition} relates to any observable aspect of the interactive ecosystem, which encompasses any variable of the context of use (\eg, user and task, platform/device, and environment)~\cite{Calvary:2002,Calvary:2003}. If the condition is not met, the adaptation operation cannot be triggered. The \textsc{Condition} is expressed as either \textit{atomic} when it consists of only one term or \textit{compound} when it is composed of several terms (\autoref{fig:condition}). Examples of atomic conditions include: ``if the user is color-blind'', ``if there is no ongoing task'', ``if the PC battery gets low'', ``if there is a mobile device near the user'', ``if the computing platform is a public display``, ``if the physical environment is shaky'', ``if the organisational environment is hierarchical''.


An atomic condition concerns an observable (its subject) that can belong to the interactive system in its business domain of application (\eg adapt the UI on business processes~\cite{Sousa:2008}) or adaptation parts, the context of use with its components \textsc{User}, \textsc{Platform}, and \textsc{Environment}, or the deployment of the interactive system in its context of use. The condition is expressed as a formula of the first-order logic \textsc{Predicate} evaluated with potential arguments and their names, such as \textsc{isLow(battery)} or \textsc{isNear(user)}. The expression can involve quantifiers (\eg ``For all'', ``There exists'') that may be positive or negative.

A \textsc{Composite Condition} is a combination of conditions (atomic or composite) using operators. For example, ``If the user's smartphone is switched on AND there is no ongoing task on it'', ``If the laptop battery has dropped AND the user turns on his/her mobile phone OR PDA''. These examples implement two classic types of operators in task modeling~\cite{Paterno:2009}: logical and temporal. Temporal operators allow reasoning about the interaction history. Other types of operator could be imagined.

The condition can be decorated with \textsc{properties} which can also be reasoned about (\eg by property-based reasoning~\cite{Blouin:2011}), such as ``If condition C is frequent'', ``If menu item I has a high probability of selection''. Relevant properties include: 
the actor who evaluated the condition,
the time of condition evaluation (Time),
the confidence factor in the condition's evaluation, and
the frequency of the condition being met. This detailed structure is crucial for creating effective adaptation operations that can be effectively and efficiently implemented to ensure AUIs under varying conditions.

\begin{figure}
    \centering
    \includegraphics[width=\textwidth]{Images/Event.pdf}
    \vspace{-8pt}
    \caption{The \textsc{Event} quadrant of our design space for adaptation.}
    \label{fig:event}
\end{figure}
\subsubsection{Event}
An \textsc{Event} identifies what has changed in the context of use and justifies an action in response to this change. Key elements include (\autoref{fig:event}):

\begin{itemize}
    \item The \textsc{variable} of change: the change can concern the user, the platform, or the environment. For example, if the user turns on a smartphone, the variable is the platform, which is now enriched with this device.
    \item The \textsc{nature} of the variation: in this case, it is the arrival (Addition) of a new platform.
    \item The \textsc{initiator} of the variation: here, it is the user.
\end{itemize}

The \textsc{Event} can also have properties such as the confidence level in perception, the actor of perception, or the frequency of the event. Reasoning can be applied to these properties. For example, the adaptation operation ``If the user is in front of the PC and the confidence factor for this event is 100\%, then...``.
The variables identify what has changed in the context of use. According to the ontology of Crowley \etal (Crowley:2002), four types of changes can occur depending on whether they affect \textsc{entities}, \textsc{roles}, \textsc{relations}, or \textsc{associations} between entities, roles, and relations:
\begin{figure}
        \centering
        \includegraphics[width=0.6\linewidth]{Images/HandMenu.png}
        \caption{Selection of menu item adapted to the hand~\cite{Antoniac:2002}.}
        \label{fig:hand}
%        \vspace{-16pt}
    \end{figure}
\begin{itemize}
    \item \textsc{Entities:} they pertain to the living or inanimate physical world. An entity is a grouping of observables. By measuring observables, the application can detect the presence of physical world entities, recognize them (\eg by gesture recognition through surfaces~\cite{Sluyters:2024}), track them, and determine the value of their attributes and relationships. We generalize this to all observable components of the context of use: user, platform, and environment (\textsc{Type}).
    \item \textsc{Roles:} they are the functions held by these entities. The roles of input and output devices are predominant. In the past, these roles were assigned to the traditional screen with the keyboard and mouse trio. Today, they are replaced by any physical entity that has the right properties and is observable by the system. For example, a hand, traditionally used as a pointing device, becomes a display surface \cite{Antoniac:2002}. Its proximity to the user and the elongated shape of the fingers make it suitable for menu display (\autoref{fig:hand}). A single entity can have multiple roles. In contrast, a role can be held by several entities simultaneously.
    
    \item \textsc{Relations:} they pertain to a set of entities. They can be spatial (\eg ``the user is near the platform''), temporal (\eg ``the device will be available during 5 minutes''), functional (\eg ``the sun illuminates the plant''. Identifying the variables means locating the change in terms of entities (\ie state, roles, or relations) as well as the set of roles and relations held.
        
    \item \textsc{Variation:} it characterizes the type of change. According to Crowley's ontology \cite{Crowley:2002}, the notions of context of use (or usage context) and situation are distinguished. A \textsc{context of use} is defined by a set of roles and relations held by a set of entities. A context change occurs when a role or relation appears (\textsc{Create}), is modified (\textsc{Modify}), or disappears (\textsc{Delete}). A \textsc{situation} is characterized by a mapping between entities, roles, and relations. A situation change occurs when these mappings change (\textsc{Create} or \textsc{Delete}  associations between entities and roles/relations) or when entities appear (\textsc{Create}) or disappear (\textsc{Delete}). \textsc{Modify} is planned to integrate changes in the state of entities into the reflection if necessary.
             This ontology allows the establishment of a graph of contexts and situations. An event is implicitly defined as an arc of this graph. There are two types of events depending on whether they are \textit{intra-context} (\ie between situations within the same context) or \textit{inter-contexts} (\ie between situations of different contexts).
        
\end{itemize}
  
Schmidt \cite{Schmidt:2000} defines the notion of an event: it is not necessarily the conjunction of an exit from one node (context or situation) and an entry into another node. An event can be more atomic, an exit from a node, an entry into a node, or even the presence in a node. Thus, we distinguish atomic events from compound events. Similarly to compound conditions, compound events use operators, particularly logical operators.

The \textsc{Initiator} of the change identifies the primary responsible agent for the variation. This information can be relevant to avoid conflicting with user actions. For example, if the user has moved the UI to a less visible area of the screen, it would be inappropriate to recentralize the UI in response. The initiator can be the interactive system or its usage context. In the interactive system, we distinguish between the business parts and the adaptation parts. For example, the interactive system \textsc{Diffie} initiates the content adaptation and highlights the results since the last visit of the user (\autoref{fig:diffie}). For the context of use, we refer to its three components: user, platform, and environment \cite{Calvary:2003}. By identifying these changes and their properties, the adaptation can better respond to the dynamic conditions in which an AUI operates, ensuring continued usability and satisfaction.
\begin{figure}
    \centering
    \includegraphics[width=\textwidth]{Images/Diffie.png}
    \caption{\textsc{Diffie} adapts the content and highlights the changes since the last visit~\cite{Teevan:2009}.}
    \label{fig:diffie}
\end{figure}




\begin{figure}
    \centering
    \includegraphics[width=\textwidth]{Images/Action.pdf}
    \caption{The \textsc{Action} quadrant of our design space for adaptation.}
    \label{fig:action}
\end{figure}

\subsubsection{Action}

The \textsc{Action} specifies the reaction to implement following a change in the context of use (\autoref{fig:action}). Examples of atomic actions include: ``Migrate the UI to the nearest platform'' \cite{Frosini:2014}, ``Remove infrequent tasks'' \cite{Florins:2004}, ``Execute a specific UI'' \cite{Calvary:2002}, ``Prioritize remodeling'' \cite{Vanderdonckt:2008:multi}, ``Eliminate automatic distribution'' \cite{Grolaux:2004}, ``Declare that the user is tired'', ``Add a specific adaptation operation on top of existing ones'', ``Execute the first applicable operation possible''.

Compound actions can also be imagined by combining actions (atomic or compound) through operators (\eg logical or temporal). Actions, whether atomic or compound, can be decorated with \textsc{properties}. Relevant properties include specifying the actor responsible for their execution and the optimization of their execution (\textsc{Time}). An atomic action is issued with a certain \textsc{force}: the action is either \textsc{proposed} or \textsc{imposed}. Actions are governed by control policies, such as ``Execute all imposed actions''. \textsc{Create}, \textsc{Delete}, and \textsc{Modify} are action types for control policies. These are \textsc{Statement} actions as opposed to \textsc{Strategies}.

\textsc{Strategies} manipulate \textsc{statements}: they filter or weight them. Four types of statements are defined: \textsc{policies}, \textsc{facts}, \textsc{guidelines}, and \textsc{target interventions}. \textsc{Target interventions} act on the context of use (\eg ``turning on the light'') or on the interactive system. For the interactive system, we functionally refine the target into either the business functions (i.e., related to the application domain) or the adaptation itself. In turn, the functional \textsc{coverage} needs to be specified in terms of the components being modified (\ie UI, functional core, and/or dialog control). For the \textsc{functional core}, we simply nuance whether the adaptation concerns the intrinsic functional core or its deployment in the context of use.

Interventions in the UI can be explored from the perspective of software architecture (\textsc{Scope}): which elements (\ie functions, components, processes) and allocations between elements (\ie functions to components, components to processes, processes to physical resources) are concerned. The considered elements and allocations depend on the adaptation level \cite{Vanderdonckt:2008:multi}: \textsc{remodeling} acts on a constant distribution state of the UI on interaction resources, unlike \textsc{redistribution}. Remodeling and redistribution can be specified in terms of \textsc{plan} or \textsc{goal}. The goal sets the objective to be achieved (\eg ``migrate the UI'') without imposing any conceptual or implementation solution, unlike the plan, where all degrees of freedom are fixed.
\begin{figure}
    \centering
    \includegraphics[width=\linewidth]{Images/Home.PNG}
    \caption{Example of a task model for home automation and its related final UI \cite{Sottet:2007}.}
    \label{fig:home}
\end{figure}

Research in plasticity~\cite{Calvary:2002} began studying remodeling on various abstraction levels such as the task level (user tasks \cite{Mezhoudi:2021} and concepts~\cite{Vanderdonckt:2008}), the abstract UI level, the concrete UI level, and the final UI level.

The \textsc{task level} describes the user's task and domain concepts independently of any representation and implementation. ConcurTaskTree and its language \cite{Paterno:2009}, UML class diagrams of OWL ontologies, are usually considered for task and domain modeling, respectively. They can be decorated with properties such as frequency, iteration, and optionality. For example, \autoref{fig:home} ~ illustrates task modeling for a home automation application. The user controls the home temperature (``Manage home temperature'') by iteratively (\ie decoration ``*'') handling the different rooms of the house. Handling a room involves specifying a command (``Specify command'') and optionally checking the feedback (``Check feedback'' with its optionality decoration). Specifying a command involves specifying the room of interest (``Specify room'') and the action to apply: either check its temperature (``Check temperature'') or change its temperature (``Set temperature''). User Interface Description Languages (UIDLs), such as UsiXML \cite{Limbourg:2004} or MariaXML~\cite{Paterno:2009} capture these aspects in a Domain Specific Language.
 


The \textsc{Final UI level} corresponds to the particular implementation of the UI for a given platform in any programming or markup language. For example, \autoref{fig:plasticlock} performs adaptation at the final UI level by adapting the layout and its widgets depending on the length and height of the window, but also depending on the different time zones when the user needs to fly between two areas.

\begin{figure}
    \centering
    \includegraphics[width=\linewidth]{Images/PlastiClock.png}
    \caption{\textsc{PlastiClock}: the final UI changes its layout and widgets depending on the window dimensions and time zone~\cite{Calvary:2004}.}
    \label{fig:plasticlock}
%    \vspace{-8pt}
\end{figure}


\subsubsection{Value}


In economics, \textsc{Value} measures the ratio between a benefit and a cost, which is suitable for adaptation, which always induces some benefits and drawbacks for a cost~\cite{Lavie:2010}. In this vein, we consider two elements: the application benefit (which we generalize into any positive or negative effect) and the implementation cost related to the execution of the adaptation operation. The cost is practical, as opposed to the effect, which is theoretical. The effect captures the expected outcomes of the adaptation operation by formulating properties expressed in a given reference frame:

\begin{itemize}
    \item \textsc{Ensured} by the application of the adaptation operation: they were not ensured before the application and they become so after the adaptation operation is completed. For example, task continuity is ensured after migration.
    \item \textsc{Preserved}: they were initially satisfied and remained so. For example, a domain concept is observable \cite{Cockton:1987} and is still observable after adaptation.
    \item \textsc{Improved}: they were partially satisfied and are now improved. For example, the cognitive load in terms of informational density is improved by moving to a large screen or multiple displays~\cite{Sluyters:2021}.
    \item \textsc{Degraded}: they were partially satisfied and are now deteriorated. For example, graceful degradation~\cite{Florins:2004} to a small screen increases navigation among tasks and increases workload in terms of physical actions.
    \item \textsc{Removed}: they were initially satisfied and they are no longer satisfied. For example, the removal of user guidance due to a lack of display surface will no longer satisfy this property.
\end{itemize}
\begin{figure}
    \centering
    \includegraphics[width=\textwidth]{Images/Value.pdf}
%    \vspace{-8pt}
    \caption{The \textsc{Value} quadrant of our design space for adaptation.}
    \label{fig:value}
%    \vspace{-8pt}
\end{figure}
The \textsc{effect} can be expressed more globally by applying the adaptation operation: does the operation pertain to the survival of the interactive system or the user's comfort? If it is about user comfort, is it functional, \ie related to the system's utility, or extra-functional related to the usability of the adapted UI and/or the meta-UI~\cite{Coutaz:2006} (defined as the UI governing the UI adaptation, later renamed into extra-UI~\cite{Melchior:2012}).

Whether the adaptation operation pertains to the survival of the interactive system or user comfort, it can be applied \textsc{proactively} (\ie in anticipation of a change of context), or \textsc{reactively} (\ie to face a reality). We call this dimension the \textsc{term}.

The \textsc{theoretical value} can be decorated with \textsc{properties} that are useful for reasoning about the adaptation. We identify as relevant the \textsc{validity} of the adaptation operation (\ie expiration date or duration), the \textsc{trust} placed in the operation, its \textsc{actor}, and its \textsc{date of issue} (\eg a recent form adaptation is more trusted than an old one~\cite{Eloi:2024}). Such values can also incorporate software quality factors \cite{iso25010}.

The \textsc{practical value} considers the cost of applying the operation. This cost is measured from a system perspective (\eg by estimating digital and physical resources required for calculation, communication, and interaction) but also from a human (\eg perceptual, cognitive, and motor load) for \textsc{perception} (condition and event) and \textsc{action}. Estimating the cost is a challenging aspect of UI adaptation. The practical value can also be decorated with properties that are evaluated by practice: the frequency of triggering the adaptation, its application and cancellation frequencies \cite{Eloi:2024}, its determinism, which integrates the level of abstraction of the operation. For example, icons in a toolbar (\autoref{fig:promotion}) can be promoted (made larger) or demoted (made smaller) depending on their probability of being selected \cite{Bouzit:2019}. This probability can be computed by recency, frequency, recurrence, importance, or any combination of them \cite{Vanderdonckt:2018, Vanderdonckt:2020}.
The adaptation operations being defined, the next section studies their life cycle.



\begin{figure}
    \centering
    \includegraphics[width=\textwidth]{Images/Promotion.pdf}
    \caption{Promotion or demotion of icons depending on their probability of use \cite{Bouzit:2019}.}
    \label{fig:promotion}
\end{figure}

\subsection{Categories of Adaptation Operations}
While any adaptation operation can be defined according to the terms defined in Section~\ref{sec:operation}, some frequent categories of adaptation operations emerge according to the effect types they produce~\cite{Paterno:2014,Florins:2004}.
We detail them in the next sub-sections.

 \subsubsection{Property-Changing Operations in Adaptation}

These operations adjust the properties of the interface without replacing or splitting elements, allowing for smooth transitions between different states of a UI based on real-time conditions. Property-changing operations are essential for dynamically modifying the presentation and behavior of UI elements in response to changes in context or user interactions. A common example of a property-changing operation can be observed in the calculator interface on a smartphone, which dynamically adjusts based on the device's orientation property.

\textbf{Example: smartphone calculator.}
When a smartphone is held vertically in portrait mode, the calculator default configuration displays only simple arithmetic operations (\autoref{fig:portrait}). When the smartphone is turned horizontally in landscape mode, it provides more space to display sophisticated scientific functions, such as trigonometric functions (\autoref{fig:landscape}). The underlying components remain the same, but the orientation affects how they are displayed and arranged. This modification is a representative example of an adaptation operation that modifies a property. In contrast, \textsc{MiniAba}~\cite{Schlee:2004} enables the end users to (un)select any subset of arithmetic, trigonometric, or advanced functions they want and to recompile the project to get an adapted version.

\begin{figure}[H]
    \centering
    \includegraphics[width=\textwidth]{Images/Calculator.png}
    \caption{Comparison of smartphone calculator in different modes: (a) portrait and (b) landscape (Source: \url{https://dribbble.com/shots/6851610-Calculator-light-mode}.}
    \label{fig:calculatorModes}
\end{figure}
%https://dribbble.com/shots/6851610-Calculator-light-mode

This adaptation operation can be represented as follows:

\textsc{On Change(Orientation)} 

\textsc{If Screen.Orientation=vertical}

\textsc{Then} 
\textsc{(Screen.orientation=horizontal And Display(FullCalculator))} 

\textsc{For Supporting expert mode}.

\begin{enumerate}
\item \textsc{Event:} the user rotates the smartphone from portrait to landscape mode, where the rotation sensor in the device detects that the phone has been turned.
\item \textsc{Condition:} the device orientation changes from vertical (portrait) to horizontal (landscape). The rotation must be 90° to transition the screen from portrait to landscape.
\item \textsc{Action:} the calculator UI reveals advanced functions (scientific mode), modifying the layout and visibility of the interface elements and dynamically adjusting its layout to display more advanced functions like trigonometric operators, exponential functions, and memory functions. In this case, the UI structure remains the same, but its presentation properties (such as visible elements and screen space allocation) are modified to accommodate the additional functions. 
\item \textsc{Value:} to enhance user interaction by dynamically displaying advanced functions, improving efficiency for expert users who need scientific calculations.
\end{enumerate}

%\textbf{Additional Examples of Property-Changing Operations:}

%\begin{enumerate} \item \textbf{Event:} The brightness of the environment changes (detected by a light sensor). \begin{itemize} \item \textbf{Condition:} The ambient light level is below a defined threshold (e.g., low light condition). \item \textbf{Action:} The UI increases the font size and contrast for better readability. \end{itemize} \item \textbf{Event:} The user activates a mobile app in a different time zone. \begin{itemize} \item \textbf{Condition:} The device detects a significant change in geographical location. \item \textbf{Action:} The app updates the displayed time and date based on the new time zone without modifying the core UI elements. \end{itemize} \end{enumerate}

\subsubsection{Splitting Operations}
Splitting operations divide a UI into smaller, manageable components, elements, or interaction spaces to adapt to devices with different display capabilities. These operations ensure the usability and consistency~\cite{Aquino:2010} across multiple platforms \cite{Florins:2006}:

\begin{enumerate}
    \item \textit{Class 1: Split in sequential operators}: if some tasks are arranged in a sequence (\eg Step 1 $\rightarrow$ Step 2 $\rightarrow$ Step 3), the UI before each sequential task can be split to create separate a screen or a page for each step, such as in a wizard widget.
    \item \textit{Class 2: Split before optional tasks}: if a sequence includes optional tasks, the UI can be split before the optional task to avoid users see unnecessary options unless needed.
    \item \textit{Class 3: Use concurrent operators when sequential splitting is not possible}: if sequential splitting is not adequate or possible, the UI can be split using concurrent operators, which indicate tasks that can occur in any order.
    \item \textit{Class 4: Split at the highest level}: when multiple levels of tasks can be split, choose the highest level. This keeps semantically related tasks together.
    \item \textit{Class 5: Distribution of tasks}:
    when splitting occurs under a higher-priority operator, ensure that necessary tasks, like a ``Cancel'' option, are available in all resulting interaction spaces.
\end{enumerate}

\begin{figure}
    \centering
    \begin{subfigure}[b]{0.9\textwidth} % Adjust the width as needed
        \centering
        \includegraphics[width=\textwidth]{Images/fig16a.jpg}
        \caption{A hotel room in three separate windows.}
        \label{fig:splitting_rules16a}
    \end{subfigure}
    \hfill
    \begin{subfigure}[b]{0.9\textwidth} % Adjust the width as needed
        \centering
        \includegraphics[width=\textwidth]{Images/Tabbed.jpg}
        \caption{A hotel room in a tabbed dialog box.}
        \label{fig:splitting_rules16b}
    \end{subfigure}
    \caption{Splitting rules applied to a hotel room booking form~\cite{Florins:2006}.}
    \label{fig:combined_splitting_rules}
\end{figure}
\textbf{Example: Book a hotel room.}
A form used to book a hotel room usually consists of specifying three parts: the \textit{location} with the arrival and departure dates and the number of guests, the selection of the hotel \textit{category}, and a selection of hotel \textit{preferences}. This form can be split into three independent windows (Class 3) as illustrated in  \autoref{fig:splitting_rules16a} or collected in a tabbed dialog box (Class 4)) as reproduced in \autoref{fig:splitting_rules16b}.
When the first task related to user information (\eg user identification) is completed, the subsequent task of specifying booking details, as represented above, and payment information can be represented as follows:

\textsc{On Change(Screen.Size)}

\textsc{If Screen.Size = Small And Task.Operator = Sequential} (\eg User Information $\rightarrow$ Booking Details $\rightarrow$ Payment Information)

\textsc{Then  Split Interaction.Space Before Task "Booking Details"}

\textsc{And Create Interaction.Space "Booking Details"}

\textsc{For Consistency}

The task model could also comprise an optional task where the guest can specify any special request, such as a non-smoking room with wi-fi access. This optional task is therefore subject to the following splitting rule (Class 2):

\textsc{On Change(Screen.Size)}

\textsc{If Screen.Size = Small and Task.Type ("Special Requests") = Optional}

\textsc{Then Split Interaction.Space Before Task="Special Requests"}

\textsc{Create Interaction.Space ("Special Requests")}

\textsc{For Optimization}

This adaptation operation ensures that the optional task "Special Requests" is displayed only when necessary to optimize the screen real estate when the screen resolution is constrained. This is part of the overall UI transformation shown in \autoref{fig:splitting_rules16b}, where navigation is adapted:

\textsc{On Change(Screen.Size)}

\textsc{If Screen.Size = Small}
\textsc{And Task.operator = Concurrent} (\eg User Information $||$ Payment Information)

\textsc{Then Create Interaction.Space ("User Information", "Payment Information") and Create Navigation.Link ("User Information", "Payment Information")}

\textsc{For Reducing(CognitiveLoad)}.

\subsubsection{Replacement Operations}
To address the needs of users with disabilities, replacement operations entail switching out certain types of UI elements with others. By offering alternate modes of interaction for the same task or action, the expected value concerns accessibility. For example, motor or visually impaired users who need to interact with push buttons, perform text input, and select options in a list, could experience trouble. Replacing these elements or enriching them with voice commands or audio feedback should support a better user experience for them \cite{Minon:2016}.
\autoref{fig:Fig13} shows the hierarchy of UI elements for entering personal data through such buttons, text fields, and radio buttons to support graphical interaction. \autoref{fig:Fig14} shows the adapted UI hierarchy where graphical elements have been replaced by speech recognition, vocal textual input and selection with audio prompts to support vocal interaction. This adaptation operation can be represented as follows:

\textsc{On Detect(User.Disability)=true} |
This event triggers the operation whenever a user’s disability is detected, such as when the user logs in or the context changes.

\textsc{If UI.Elements=graphical} |
The condition checks if the current UI elements are graphical. This means the user interface contains some elements like buttons, text fields, or icons that may not be accessible to users with certain disabilities.

\textsc{Then Replace(UI.Elements, graphical, vocal)}
\textsc{And Create(AudioCommands)} |
This action replaces the graphical UI elements with vocal elements and audio commands. The user can now interact with the interface using voice commands and receive audio feedback instead of visual prompts.

\textsc{For Accessibility}
 
\begin{figure}%[H]
    \centering
    \begin{subfigure}[b]{0.45\textwidth}
        \centering
        \includegraphics[width=\textwidth]{Images/replacement_rule_13.png}
        \caption{For graphical interaction.}
        \label{fig:Fig13}
    \end{subfigure}
    \hfill
    \begin{subfigure}[b]{0.47\textwidth}
        \centering
        \includegraphics[width=\textwidth]{Images/Replacement_rule_14.png}
        \caption{For vocal interaction.}
        \label{fig:Fig14}
    \end{subfigure}
    \caption{Hierarchy of UI elements to input personal data ~\cite{Paterno:2011}: (a) before adaptation, (b) after adaptation.}
    \label{fig:combinedFigure}
\end{figure}


\subsubsection{Removal Operations}
Removal operations remove UI elements at run-time for several reasons: they are irrelevant, inappropriate, unnecessary for particular user groups or use cases, or consuming too many resources, such as screen space. Removal operations could physically remove these UI elements, momentarily hide them, or disable them. In contrast to replacement operations, which replace UI elements with alternatives, removal operations simply delete them, either momentarily or permanently. For example, \textsc{Supple}~\cite{Gajos:2010} dynamically adapts UIs based on user behavior and needs. In \autoref{fig:removal_example15}, a print dialog box is adapted to prioritize frequently used options and remove less relevant components. 
\begin{figure}[t]
    \centering
    \includegraphics[width=\textwidth]{Images/supple.pdf}
    \caption{(a) Original print dialog box with multiple steps to access printing orientation. (b) Adapted dialog box with simplified access to printing orientation, removing unnecessary steps \cite{Gajos:2010}.}
    \label{fig:removal_example15}
\end{figure}

\autoref{fig:removal_example15} demonstrates how the UI is adapted to reduce complexity. In the original print dialog box (\autoref{fig:removal_example15}a), changing the print orientation from portrait to landscape requires navigating through multiple steps and options, which can be difficult for users with motor impairments or limited cognitive abilities. This complexity is not only time-consuming, but also increases the risk of errors during interaction. The adapted interface (\autoref{fig:removal_example15}b) simplifies this process by removing unnecessary steps and presenting the landscape printing option directly on the main screen, which reduces the need for complex navigation and makes the UI more intuitive and accessible.

By focusing on the most commonly used functions and removing less relevant options, \textsc{Supple} tailors the UI to the user's needs, resulting in a streamlined interaction experience. This adaptation operation can be represented as follows:

\textsc{On User.Access("LandscapeOption") And Frequency("LandscapeOption") $>$ Threshold}

\textsc{If UI.Contains("IntermediateSteps")}

\textsc{Then UI.Remove("IntermediateSteps") and            UI.Display("LandscapeOption", "MainScreen")}
       
\textsc{On User.Preference("SimplifiedUI")}

\textsc{If UI.Contains("RarelyUsedOptions")}

\textsc{Then UI.Remove("RarelyUsedOptions")}

\textsc{For Reducing(TaskTime)}
\begin{itemize}
    \item \textsc{Event:} the system detects frequent access to the landscape printing function by comparing it to a reference threshold.
    \item \textsc{Condition:} the ``Print'' dialog box includes multiple steps for accessing this function.
    \item \textsc{Action:} the system removes intermediate steps and presents the landscape printing option by default, simplifying the interaction.
    \item \textsc{Value:} the adaptation aims at simplifying the UI by reducing the task completion time.
\end{itemize}

This operation dynamically simplifies the user interface based on the user's capabilities, ensuring a more accessible and user-friendly experience.


\subsection{The Left Hemisphere for the Specification of the Adaptation Implementation}
The left hemisphere identifies four stages in the life cycle of an adaptation operation (\autoref{fig:lifecycle}): the definition of an adaptation operation, its execution, its evaluation and, finally, the consolidation of experience gained with this adaptation. For each stage, we are encouraged to answer the same questions of the Quintilian hexameter~\cite{Motti:2013}, as outlined in \autoref{sec:introduction}.

\begin{figure}
    \centering
    \includegraphics[width=.65\textwidth]{Images/LifeCycle.pdf}
    \vspace{-6pt}
    \caption{Left hemisphere: stages of adaptation life cycle.}
    \label{fig:lifecycle}
\end{figure}

\subsubsection{Definition of the Adaptation Operation}
The \textit{definition} of the adaptation operation is hereby referred to as any specification, possibly by code generation or interpretation of the specification, of the precise instructions to be performed by an adaptation operation. This specification is aimed at enabling the concrete execution of an operation in the next stage. This should not be confused with the specification of adaptation (\autoref{fig:lifecycle}), which specifies which entity is responsible for deciding an adaptation, whereas this definition is aimed at building the adaptation operation itself. This encompasses the research, the design, and the coding of an operation.
In regard to \textit{What?}, we distinguish the specification of the overall goals of the operation (\eg maximize the cognitive load) and the specification of the operation itself (\eg ``Remove optional widgets''). By overall goals, we may consider the specification of the \textsc{Value} to be guaranteed and the domains or zones of plasticity~\cite{Collignon:2008} to be guaranteed for this value \cite{Calvary:2002}. 
Specifications can be written by (\textit{Who?}) experts in human factors or adaptation, the designer, the end user, the interactive system, any third party, or any combination of them. For example, \textsc{Scaler}~\cite{Eloi:2024} enables the definition of an adaptation operation by configuring and balancing weights (\autoref{fig:configuration}) specified by a designer, a developer, and the end-user, either in isolation one by one or from a crowd, as suggested by Nichols \etal~\cite{Nichols:2013}.  

\begin{figure}
    \centering
    \includegraphics[width=\textwidth]{Images/Configuration.png}
    \caption{Configuration of parameterization for adaptivity by unsupervised learning \cite{Eloi:2024}.}
    \label{fig:configuration}
\end{figure}

The specification tool (\textit{How?}) depends, of course, on the actor in charge of specification. Adaptation operations can be defined according to a wide range of tools, spanning from hard-coded environments (\eg the graceful degradation plug-in offers three families of adaptation operations that are all predefined and coded in Java~\cite{Florins:2004}) and knowledge bases (\eg operations for adapting a graphical UI to a mobile device in \cite{Eisenstein:2000} are stored in a knowledge base that is processed by an inference engine) to specification environments (\eg operations are defined in a domain specific language) and model-driven architectures~\cite{Sottet:2007, Sottet:2008}.


Depending on the tools (\textit{How?}), the actor (\textit{Who?}), the definition can be achieved at different times (\textit{When?}): design time, linking time, compilation time, installation or deployment of the interactive system, its execution and inter-session, requiring the interactive system to be stopped and then restarted.
Operations can be stored (\textit{Where?}) within the interactive system itself
(internal) or externally, in configuration files for example.


\subsubsection{Execution of the Adaptation Operation}
Adaptation is a seven-stage process (\autoref{fig:lifecycle}) involving (\textit{What?}) detecting the change in the context of use to take the initiative for adaptation (stage 2), to specify which adaptation (stage 3) and apply it (stage 4, which corresponds to the actual \textit{execution} of the adaptation operation). These functions require
mechanisms that are (\textit{Where?}) internal or external to the interactive system and its AUI. 
These functions can be carried out by four types of actors (\textit{Who?}): the end-user,
another user (\eg a controller), the interactive system, or another system. We identify five key moments for their execution (\textit{When?}): either during execution (\eg at any time or at precise moments), the granularity of the stage can range from the physical action to the session, via the elementary task and the compound task. Depending on the support tool, past actions may be lost or retained.

Support tools (\textit{How?}) can be generic and reusable, or integrated into the interactive system, and consequently non-reusable. They can be equipped with an extra-UI~\cite{Coutaz:2006} (formerly known as meta-UI), which is itself a UI that ensures observability and control of the adaptation~\cite{Bouzit:2017:PDA}. This extra-UI may require
resources (\eg UI components, libraries, perception services, agents, inference engines) that are either prefabricated or generated at runtime~\cite{Blumendorf:2010}. Their location is internal or external to the interactive system and their availability can be perennial (\eg beyond adaptation) or temporary (\eg erased after adaptation).

\subsubsection{Evaluation of the Adaptation Operation}
The \textit{evaluation} is a process that critically examines to what extent the adaptation operation has been effective and efficient, which involves collecting and analyzing data and outcomes resulting from its execution and outcomes. Its purpose is to assess the overall quality of an adaptation operation to improve its effectiveness and to inform future executions.
We identify five actors (\textit{Who?}) likely to be involved in evaluation: an evaluation specialist, a designer, the end-user, and the interactive system or any third party, provided that they hold the computational capabilities to perform such an evaluation.
The evaluation can cover the four stages: the definition of an adaptation operation, its execution, its evaluation (which would then be a recursive process: how to evaluate the evaluation) and its consolidation of experience. For example, \autoref{fig:evaluation} shows how an adaptation operation can be evaluated by the system itself (\eg by performing some automatic evaluation) or by the end user qualitatively (\eg using a rating bar) or quantitatively (\eg using a rating scale). In the past, automatic UI evaluation has been explored, such as by guideline review~\cite{Beirekdar:2005} or agent analysis \cite{Lopez:2009}. More recently, eye tracking and emotion recognition represent attractive candidates to determine to what extent the end-user is happy or not with an adaptation~\cite{Haddad:2024}.

Evaluation can be carried out in the very true context of use or controlled in a specific environment, such as in a usability laboratory. It can be carried out on the fly or off-line (\textit{When?}). For example, captured on-the-fly, the first impression that an end user produces is indicative of the graphical UI being good or bad~\cite{Haddad:2024}. Mechanisms for showing how adaptivity has been performed (\eg \autoref{fig:transition} shows adaptivity at run-time using an animated transition that preserves the context of use) and for explaining it are welcome.

\begin{figure}
    \centering
    \includegraphics[width=\textwidth]{Images/Transition.png}
    \caption{Animated transition showing the various stages of adaptivity \cite{Dessart:2011}.}
    \label{fig:transition}
\end{figure}

\subsubsection{Consolidation of Experience}
Consolidation of the experience gained with the adaptation operation is a new dimension that is required for future use by learning what happened (\eg by keeping an appropriate operation and discarding an inappropriate one~\cite{Zouhaier:2021}). Five players are likely to be involved (\textit{Who?}): an expert in knowledge management or computational approaches \cite{Jiang:2022}, an experienced designer, the end user, the interactive system and its AUI or any other party. Consolidation can be general, covering the value of operations (weak or strong), or limited to values only (\textit{What?}).
Consolidation requires software mechanisms (\textit{How?}), \eg ML/DL/RL tools~\cite{Bouzit:2017:PDA,Gaspar:2024,Todi:2021}, which can be external or embedded in the interactive system itself. The support offered can range from simple memorization of experiences to the deduction of new adaptation operations. Results can be organized or not, justified or not, stored internally or externally to the interactive system (\textit{Where?}), and consolidated on-the-fly or off-line (\textit{When?}).




\section{Practical Lessons from Experience}
Existing methods for designing UIs of interactive systems (\eg \cite{Cockton:1987,Gulliksen:2003}) still apply to AUIs, but with the need to pay attention to the context of use and the quality of the use. For both of them, it must be understood that they may result either from user-centered requirements elicited during the \textsf{Analysis} phase (see the \textsf{Problem} part of the design process in \autoref{fig:Gulliksen}) or from design choices made by the practitioner (see the \textsf{Solution} part of the design process in \autoref{fig:Gulliksen}). Several lessons can be drawn from experience. \textbf{Lesson n°1} recommends considering explicitly both the context of use and the quality in use, together with their rationale. User-centered requirements ``must`` be satisfied, while practitioner's decisions ``should`` be preserved for interaction continuity.
\begin{figure}[b]
    \centering
    \vspace{-16pt}
    \includegraphics[width=.95\textwidth]{Images/UCSD.pdf}
    \vspace{-8pt}
    \caption{Gulliksen's method for designing UIs, adapted from \cite{Gulliksen:2003} (Illustration by courtesy of Jan Gulliksen and Bengt Göransson).}
    \label{fig:Gulliksen}
\end{figure}
Once the context of use is determined and the quality in use is decided accordingly, \textbf{lesson from experience n°2} recommends reasoning for the transitions, considering the change from one context of use to another. The \textbf{lesson from experience n°3} recommends designing a UI first for the most constrained context of use, \ie the one combining the most constrained user, platform, and environment. This ``reference UI`` could then be considered for designing UIs for other contexts of use by progressively relaxing, which is referred to as \textit{progressive enhancement}, the inverse of \textit{graceful degradation}~\cite{Florins:2004}. Serna \etal ~\cite{Serna:2010} demonstrates that this improves the quality of use of the resulting UIs. Consequently, \textbf{lesson from experience n°4} consists of using the AUIs as a means for quality, \ie to consider very constrained contexts of use, even if they are not targeted, just for quality.

\section{Conclusion and Perspectives}
\label{sec:conclusion}
This paper provides an overview of UI adaptation since its inception until recent days, by focusing mainly on adaptive UIs, where adaptation is ensured by the interactive system or application itself, as opposed to adaptable UIs (where the end-user is responsible for adapting the UI) and mixed-initiative UIs (where the system and the end-user collaborate to ensure the adaptation).

For this purpose, we discuss the traditional questions of adaptation (\ie what to adapt, why to adapt, how to adapt, with regard to what, who controls the adaptation, when to adapt, and where to adapt) that need to be answered when adaptation should be implemented in a UI. We then presented a generic UI adaptation life cycle, which can be decomposed into seven stages (\eg ranging from UI adaptation goals to evaluation of adaptation). A targeted literature review resulted in a discussion of existing and future challenges encountered in adaptation and in recent advances in the domain, such as those induced by computational approaches. We then presented a comprehensive problem space for UI adaptation made up of two hemispheres: one for the specification of the adaptation operation (which is the cornerstone of adaptivity) and one for the specification of the implementation of adaptation. The first hemisphere is further decomposed into four quadrants: condition, event, action, and value. The second hemisphere is further divided into four stages: definition, execution, evaluation, and consolidation. From experience, we believe that these structuring elements inform and guide the engineering of adaptive UIs.

The greatest challenge posed by adaptation, now and in the future, is that of the evolution of adaptation. Adaptivity is inevitably an evolutionary process for both the end-user and the interactive system. Users evolve in their experience, the way they perform interactive tasks, and in their preferences. The environment in which they evolve also evolves inexorably. The interactive system cannot remain unaffected by these changes. To date, numerous methods have been applied to compensate for this lack of evolution, mainly by adding new adaptation rules, modifying and improving their application strategies, or taking better account of user parameters. This compensation remains insufficient to draw relevant and useful conclusions about the future of the interactive system.

That is why machine learning, in general, \cite{Mahdavinejad:2018} and reinforcement learning in particular offer many suitable advantages, enabling us to take into account how the UI evolves in its context of use. The most recent advances \cite{Gaspar:2024,Langerak:2024,Todi:2021,Zouhaier:2021,Mezhoudi:2021,Eloi:2024} are heading in this direction, and are already taking constructive steps towards overcoming this lack of consideration for evolution. A second potential avenue is to determine to what extent adaptation can be fragmented in time and space (\eg Todi \etal \cite{Todi:2021} gradually transform graphical adaptive menus, Bouzit \etal \cite{Bouzit:2016} suggest step-by-step progress and adaptation in mobile menus) to deliver an adaptation experience that would reduce adverse effects~\cite{Lavie:2010}, such as disruption~\cite{Hui:2009}.

We hope that the future will allow us to determine the best methods, techniques, and algorithms to ensure the best possible adaptivity. Meanwhile, lessons from experience are reported to reconcile generic know-how in engineering HCI and specific advances in AUIs.

 
%\chapter[Use Case 1: Comparative Testing on Unipath Dynamic Gestures]{Use Case 1: Comparative Testing of Recognizers on Unipath Dynamic Gestures}\label{chap:Unipath_Comparative}

%ToDo
The work presented in this chapter was initially published in \textit{Proceedings of the ACM on Human-Computer Interaction, Volume 4, Issue ISS, Article No.: 198, pp 1–21}, published in 2020.\cite{Ousmer:2020}

In this work, we use the term \enquote{use case} in its general definition \enquote{\textbf{Use Case} \textit{n. A use to which something (such as a proposed product or service) can be put}}(Merriam-webster: Use Case)\cite{Merriam-webster:UseCase}. It is an application of the comparative testing systematic procedure defined in Chapter~\ref{chap:Concepts}. Two experiments will be conducted in the context of specific use cases focusing on \enquote{unipath dynamic} gestures (Chapter~\ref{chap:Unipath_Comparative}) and \enquote{multipath dynamic} gestures (Chapter~\ref{chap:LMCmultipathComparative}), respectively. These experiments will be evaluated under an \enquote{intra-device configuration}, where a single device will be used during different stages of the gesture recognition process, as shown in Figure~\ref{fig:UseCase_Configurations}. This configuration was determined by the datasets used for the experiments. 

In this chapter, we perform an experiment on seven template-based recognizers. In order to assess the efficiency of these recognizers for interaction using the comparative testing method. This interaction requires a low response time (\eg, 0.1 s according to Nielsen~\cite{Nielsen:1994}) and high accuracy (\eg, a recognition rate of $\geq90\%$~\cite{Marin:2016,Wang:2015}).  Under predefined conditions (\textit{\ie}, parameter values), we will use a challenging set of 3D trajectories known as \enquote{unipath dynamic} gestures which are divided into six datasets from different input device categories, including “LeapMotionController” for SHREC2019 and 3DTCGS, and “SoftKinetic DepthSense DS325” for the 3DMadLabSD.
  The experiment will be conducted to measure recognition rates and execution times for two datasets in user-independent scenarios. Additionally, we include one dataset that includes both user-dependent and user-independent scenarios.


\begin{sidewaysfigure}[!h]
    \centering
    %\hspace{-140px}
    \includegraphics[height=0.59\textheight]{Figures/Chap4/Chap3_TestingConfiguration.pdf}
    \caption{Configuration of Input Devices in Use Cases.}
    \label{fig:UseCase_Configurations}
\end{sidewaysfigure}

\clearpage

Some recognizers perform similarly to existing 3D recognizers for these gestures. We also discuss the implications of selecting a recognizer based on context-specific conditions, such as sampling and number of points, to determine the impact on recognition rate and execution time.
\section{Recognizers}
\label{sec:selection}
Based on the TLR in Subsection~\ref{subsec:2DRelated},  four recognizers were selected for comparative testing. Rubine ~\cite{Rubine:1991} pioneered 2D stroke recognition using statistical matching on weighted feature vectors. $\$P$~\cite{Vatavu:2012b} introduced the cloud matching principle, while $\$P+$~\cite{Vatavu:2017} has the best accuracy and $\$Q$~\cite{Vatavu:2018} has the fastest execution time.
    

$\$1$\cite{Wobbrock:2007} is not included because several aforementioned recognizers outperform it~\cite{Taranta:2015,Vanderdonckt:2018}. Similarly, $\$N$ is also not included due to a combinatory explosion in both memory and execution time despite two-speed optimizations.

$\$P$ addresses this limitation by treating gestures as unordered sets of points. Other extensions and local optimizations discussed in Subsection~\ref{subsec:2DRelated} are excluded. $\$3$ and similar algorithms are potential candidates but are not included because they rely on additional data from a wearable device, such as an accelerometer, which we want to avoid.
Vector-based recognizers~\cite{Taranta:2015,Vanderdonckt:2018} are not retained as they may require recomputation of vectors between points or vectors between vectors for each plane, leading to increased computational cost. Other 3D recognizers belonging to classifier classes with different computational complexities to cover a broader range of 3D gestures were not included.

The selected recognizers undergo a three-dimensionalization process, adding the $Z$ dimension as an independent dimension to the existing $X$ and $Y$ dimensions~\cite{Ngan:2004}.

\subsection{Rubine3D, a 3D Extension of Rubine}
\label{sec:rubine3D}
Rubine3D (\textit{R3D}) extends the feature-Based 2D gesture recognizer Rubine~\cite{Rubine:1991} to handle 3D gesture.  It is inspired by the Rubine3D implemented in the iGesture framework~\cite{Signer:2007} (Subsection~\ref{subsec:Related_Rubine3D}). The Rubine3D recognizer combines three individual 2D Rubine recognizers for each plane: $XY$, $YZ$, and $ZX$. The projection on each plane is achieved by transforming each point of the 3D trajectory to the three coordinate system planes $(XY,YZ,ZX)$. A plane in an orthogonal coordinate system is represented by the linear equation $Ax+By+Cz +D=0$, where $n(A,B,C)$ are the coordinates of the normal vector to a plane. If the plane passes through the origin, the equation has constant term $D{=}0$. The $XY$, $YZ$, and $ZX$ planes passing through the origin have equations $z=0$, $x=0$, and $y=0$ respectively. For a point $M(i, j, h)$ in the 3D trajectory, $P, Q, R$ are the 2D points where the coordinates of $M$ intersect the $(XY, YZ, ZX)$ planes. These points are the orthogonal projections of $M$ onto each plane with vectors parallel to the $X$, $Y$, and $Z$ axes. Thus, $P(i,j)$, $Q(j,h)$, and $R(h,i)$ are the coordinates of the 2D projections on $(XY, YZ, ZX)$.


First, the raw data are preprocessed by scaling and filtering points. Thus, a point is discarded if the distance from the previous point is under a threshold of $d{=}.005$.
Next, for each training gesture projected on each plane, the thirteen original features $(f_1,..,f_{13})$ ~\cite{Rubine:1991} are calculated.

After that, every class mean feature vector and covariance matrix are computed for all planes. We then calculate the common covariance matrix and the inverse matrix, of which the gesture class weights are estimated.
Thus, several actions are required to determine the possible class of the candidate gesture. To begin, the gesture is projected on each plane.

Next, the feature vectors are extracted from the three gestures projections. After that, the gestures projections are defined as one of the possible gesture classes with a linear function. This function evaluates which class maximizes the evaluation function result.

 Finally, a heuristic inspired by iGesture is used to determine the final result class of the 3D gesture. If all three projections have the same result class, this is called the final result. However, if there are two or three different classes, we calculate the scores of these classes for the three planes and multiply each score result by a weight factor. The weight factors for the planes are \textit{$W_{XY}=0.4, W_{YZ}=0.3, W_{ZX}=0.3$}, based on the ease of producing gestures in the sagittal plane, then in the transversal plane, and finally in the frontal plane, which is particularly observed for head and shoulder gestures~\cite{Vanderdonckt:2019}. We found that using only two planes decreased accuracy. The final gesture class is determined by summing the scores.
 
 However, the gesture rejection part of Rubine has not been implemented. This decision was made to ensure consistency in computational complexity with other recognizers and to focus solely on the classification problem.

\subsection{Rubine-Sheng, Another 3D Extension of Rubine}
\label{sec:rubinesheng}
Rubine-Sheng (RS) uses a vector of sixteen features computed from a 3D gesture $G = \{p_t = (x_t, y_t, z_t) | \forall t = 1,...,n\}$. These features extend Rubine's original feature vector to 3D by incorporating three additional features proposed in the AdaBoost recognizer and applied to 3D gestures~\cite{Sheng:2003}. Similar to its 2D counterpart, the Rubine-Sheng recognizer classifies a candidate gesture by identifying its corresponding gesture class. First, each class's mean feature vectors and covariance matrix are estimated using the feature vectors of the training samples. Then, the common covariance matrix is computed and its inverse is used to determine the weights of the gesture classes. Finally, the class of a candidate gesture is the class that maximizes the result of the discrimination function.

\subsection{\texorpdfstring{$\$P^3$}{\$P3}, a 3D Extension of \texorpdfstring{$\$P$}{\$P}}
\label{sec:dollarp}
The $\$P$ algorithm matches a candidate gesture $C$ with each template $T$ in the training set. This is done using a function $M$ that associates each point $C_i{\in}C$ with one point $T_j{\in}T$, where $T_j{=}M(C_i)$. The classification result determines the closest template $T$ to the candidate $C$ by calculating the matching distance: $C \in \text{class of } T \text{ where } T{=}argmin_T \{\$P(C,T)\}$. This principle is extended to 3D by defining a set of points with three-dimensional coordinates: $C=\{p_i = (x_i, y_i, z_i) | \forall i=1,...,n\}$.

$\$P^3$ is the 3D extension of $\$P$, inspired by the implementation of the \$P3D recognizer by Cook \etal~\cite{Cook:2016}. However, unlike \$P3D, the \$P\textsuperscript{3} recognizer does not support 3D static poses and 2D dynamic gesture recognition. 

To minimize gesture variations, the recognizer performs three pre-processing steps:

\begin{itemize}
    \item Resampling the point clouds of gestures to ensure they have the same number of points $n$ for matching.
    \item Scaling the gesture points to a non-uniform reference box.
    \item Translating the centroid of the gesture $(Gx, Gy, Gz)$, to the origin $O=(0,0,0)$.
\end{itemize}
The matching score is defined as the sum of Euclidean distances of all pairs of points from $M$. This calculation is generalized into 3D by incorporating the third coordinate, $z$. Let's assume that point $C_i$ from the candidate gesture $C$ is matched to point $T_j$ from the template $T$. The score is then given by:
 \begin{equation}
 \label{eq:euc}
 \sum_{i=1}^{n} \parallel C_i - T_j \parallel =\sum_{i=1}^{n} \sqrt{(C_i.x - T_j.x)^2+(C_i.y-T_j.y)^2+(C_i.z-T_j.z)^2}
 \end{equation}
 
In order to compute the dissimilarity score between two clouds of points, we performed a one-to-one time-free alignment between points, inspired by Vatavu's matching heuristic named \textsf{Greedy-5} because it gave the best results among the tested methods~\cite{Vatavu:2012b}. The heuristic matches each point from the first cloud with one point in the second cloud, with the condition that it has not been matched before. Next, it matches points from the second cloud that have not been matched yet can be matched to one point in the first cloud resulting in a complexity of $\mathcal{O}(n^2)$. The algorithm runs several times with different starting points considered circularly through all points and returns the minimum matching of all runs. The returned matching sum is multiplied by a weight representing a confidence degree on the matching.

\begin{equation}
 w = \sum_{i} w_i \cdot \parallel C_i - T_j \parallel
\end{equation} 
The $\epsilon$ threshold controls the number of runs and affects the complexity of the heuristic to $\mathcal{O}(n^{2+\epsilon}))$. We take into account that the direction of matching impacts the result of the heuristic.
%\textit{min (GREEDY-5(C,T), GREEDY-5(T,C))}.

%\subsection{$\$FP^3$ -- A Flexible P Dollar Recognizer} - Peut-être remplacer le nom plus tard
 \subsection{\texorpdfstring{$\$F$}{\$F}, a Flexible Variant of \texorpdfstring{$\$P^3$}{\$P3}}
\label{sec:dollarf}
$\$F$ extends the previously mentioned $\$P^3$ with the flexible cloud matching of $\$P+$~\cite{Vatavu:2017}. As usual, both the candidate and template points are resampled to equidistantly-spaced points, scaled within a unit box, and translated so that their centroid is at the origin $(0,0,0)$. The template with the lowest dissimilarity score is considered the best matching template for the candidate. 

The $\$P+$'s cloud matching process involves matching the points from the first cloud with their closest point from the second cloud, and then matching the remaining points from the second cloud with their closest point from the first cloud. To investigate the possibility of a more flexible matching, $\$F$ allows the matching of two points that have already been matched, hence its name $\$F$.

\subsection{FreeHandUni, a 3D Extension of \$P}
\label{sec:freehanduni}
The FreeHandUni (\textit{FH}) recognizer is based on the pseudocode of the FreeHand recognizer, which extends the 2D gesture recognizer \$P++~\cite{Vatavu:PseudAlgo}. Additionally, it uses full-hand information for free-hand gesture recognition.
FreeHandUni is adapted to recognize the unipath character of 3D trajectories by replacing the hand pose structure with a 3D point structure $(x,y,z)$. With this modification, FreeHandUni can be seen as an improvement of $\$P^3$, using a flexible cloud matching approach based on one-to-many alignment between points~\cite{Vatavu:2017}. The pre-processing step remains the same, and the matching process uses Euclidean distance (Eq.~\ref{eq:euc}) with increased flexibility. Each point from the template cloud is matched to the closest point from the candidate cloud, and each remaining point from the candidate cloud is matched to the closest point from the template cloud. The gesture class is determined based on the template cloud with the lowest dissimilarity score compared to the candidate cloud, and vice versa. Unlike $\$F$, FreeHandUni does not implement early abandoning to keep the computational complexity close to $\$P^3$.


\subsection{\texorpdfstring{$\$P+^3$}{\$P+3}, a 3D Extension of \texorpdfstring{$\$P+$}{\$P+}}
\label{sec:dollarpplus}
One of the major improvements with respect to $\$P$ lies in expanding the matching from one-to-one to one-to-many points, which clears the weights of the starting point used in $\$P$. Based on the 3D points, we compute the turning angle at each point with points coordinates gesture:
\begin{equation}
\scalebox{1}{
$
C_{i.\theta} = \frac{1}{\pi} \arccos{( \gamma)}
$
}
\end{equation}
where
\begin{equation}
\vspace{-8pt}
\hspace{-10pt}
\scalebox{0.92}{
$
\gamma = 
\frac{(C_{i+1}.x−C_i.x)\cdot(C_{i}.x−C_{i-1}.x)+(C_{i+1}.y−C_i.y)
\cdot(C_{i+1}.y−C_i.y) +(C_i.z−C_{i-1}.z)
\cdot(C_{i}.z−C_{i-1}.z)}{||C_{i+1}−C_i ||\cdot|| C_i - C_{i−1}||} 
$}
\vspace{8pt}
\end{equation}

  
To optimize the execution time, the early abandoning used in $\$Q$ is added, as well as the third coordinate in the computation of the point distance beside the angle as in $\$P+$:

\begin{equation}
\scalebox{.8}{
$
D(C_i,T_j)=\sqrt{(T_j.x-C_i.x)^2+(T_j.y-C_i.y)^2+(T_j.z-C_i.z)^2+(T_j.a-C_i.a)^2}
$
}
\end{equation}

\subsection{\texorpdfstring{$\$Q^3$}{\$Q3}, a 3D Extension of \texorpdfstring{$\$Q$}{\$Q}}
\label{sec:dollarq}
$\$Q^3$, is the 3D extension of \$Q~\cite{Vatavu:2018}. It performs the same preprocessing as $\$P+^3$, except for the calculation of a 3D Look-Up-Table (LUT) offline for each template and its storage with the cloud points. The look-up point-matching technique is generalized to three dimensions using a $16{\times}16{\times}16$ 3D grid of equidistant points. Each cloud point $C$ is positioned within this grid. The index of the row, column, and layer of the closest point in the grid is stored for each point, resulting in a computational complexity of $\mathcal{O}(n\cdot m^3)$. To recognize a candidate, the closest point from a template to $C_i$ is determined by iterating through the cloud points and summing the Euclidean distances of Eq.~\ref{eq:euc} between each pair of points. This computation is stopped when the sum exceeds the minimum dissimilarity score.
\section{Datasets}
\subsection{SHREC2019}
The \textbf{SHREC2019} dataset~\cite{Caputo:2019} was used as a benchmark for the Eurographics 2019 SHape Retrieval Contest (SHREC) track on online gesture recognition. It consists of 195 3D trajectories performed by 13 participants using their whole hand, representing five iconic gestures: \enquote{Cross} (X), \enquote{Circle} (O), \enquote{V-mark} (V), \enquote{Caret} (/\textbackslash), and \enquote{Square} ([]). The dataset is relevant because it is recent and contains simple gestures.

To detect command gestures from hand movements in a virtual reality context, the training set and testing set were combined to create a single dataset. Unnecessary hand movements were filtered by clipping the sequences before and after the main section~\cite{Kratz:2015}.

\begin{figure}[h]
    \centering
    \includegraphics[width=.84\textwidth]{Figures/Chap4/SHREC-Images.pdf}
    \vspace{-4pt}
    \caption{The SHREC2019 dataset~\cite{Caputo:2019}.}
    \vspace{-8pt}
   \label{fig:dataset_SHREC2019}
\end{figure}

\subsection{3DTCGS}
The \textbf{3DTCGS} dataset~\cite{Caputo:2017} is used to test the classification performance of the 3\textcent~\cite{Caputo:2017} recognizer on interface command gestures. The data is provided in a segmented form, making it easier to exploit during experimentation for the classification task. This dataset contains gestures of varying complexity, ranging from simple ones like a \enquote{3D swipe} to more complex ones like a \enquote{3D spiral}. Similar trajectories also include different directions (\textit{\eg}, \enquote{left-swipe} vs. \enquote{right-swipe}, \enquote{arc3Dleft} vs. \enquote{arc3Dright}), to test the direction-invariance property.

Each subject produces one sample per gesture, which makes the user-independent evaluation particularly challenging.

Participants were asked to perform short iconic gestures with their dominant hand forefinger. The collected data consists of 347 sequences of 3D coordinates, combined with a timestamp. These sequences describe 26 gesture classes performed by 14 subjects and recorded with a Leap Motion (see Figure~\ref{fig:dataset_3DTCGS}).

\begin{figure}[h]
    \centering
    \vspace{-8pt}
    \includegraphics[width=.97\textwidth]{Figures/Chap4/3DM-Images.pdf}
    \vspace{-4pt}
    \caption{The 3DTCGS dataset~\cite{Caputo:2017}.}
    \vspace{-8pt}
    \label{fig:dataset_3DTCGS}
\end{figure}


\subsection{MadLabSD}
  The \textbf{MadLabSD} (MadLab Sketch Dataset) dataset ~\cite{Huang:2019} is publicly available online since 2018. It was used to assess a new gesture-based system~\cite{Huang:2019}. This dataset consists of mid-air single stroke gestures, which are palm-centered dynamic motion trajectories. It offers a wide range of gestures, from common symbols  (\textit{\ie,}, alphabets and numerals) to uncommon ones (\textit{\eg}, CAD primitives), making it particularly attractive.
  
 The dataset was recorded by a group of 10 users who captured 40 segmented symbols and sketches. These were split into four domains, as shown in Figure~\ref{fig:dataset_domains}, resulting in a total of 4,000 samples. Each user recorded 10 samples per gesture, which included the 3D position coordinates of the center of their palm along with a timestamp.
 
 The dataset also includes the extraction of the center palm from depth grayscale images, which were recorded using a SoftKinetic DepthSense DS325. The gesture was recorded based on the static hand pose, allowing the user to switch between active (hand open) and inactive (closed hand) states.
 
It is important to note that the raw data from the depth camera is not preprocessed, potentially containing spurious data, such as noisy points at the start of the sketch, due to user latency after mode switching ~\cite{Huang:2019}.

The dataset comprises four domains:

\begin{itemize}
\item \textbf{(D1) Arabic numerals}: \textit{0-9}
\item \textbf{(D2) English alphabets (Handwritten)}: \textit{a-j}
\item \textbf{(D3) Simulation symbols}: \textit{Spring-Mass-Wheel-Pulley-Hinge-Fast Forward-Rewind-Play-Pause-Delete}
\item \textbf{(D4) CAD primitives}: \textit{Cuboid-Cylinder-Sphere-Rectangular Pipe-Hemisphere-Cylindrical Pipe-Pyramid-Tetrahedron-Cone-Toroid}
\end{itemize}




\begin{figure}[h]
    \centering
    \includegraphics[width=1\textwidth]{Figures/Chap4/Domains-Images2.pdf}
    \vspace{-8pt}
    \caption{The four domains of the MadLabSD dataset.~\cite{Huang:2019}}
    \label{fig:dataset_domains}
\end{figure}

\vspace{-8pt}
\section{Design}
\label{sec:Unipath_Comparative_evaluation}
We evaluated the seven recognizers using both traditional user-dependent and user-independent scenarios~\cite{Anthony:2010, Anthony:2012, Vatavu:2012b, Vatavu:2017} on 3D trajectories from six datasets. The goal was to demonstrate the efficiency of these recognizers under different conditions (\textit{\ie}, parameters values). We focused solely on evaluating the classification task, not the preprocessing, as this task can be done offline.
\vspace{-6pt}

\subsection{Experiment}
Our study was within-factors with four independent variables:
\begin{itemize}
    \item \textsc{Recognizer}: This nominal variable with 7 conditions, representing the various recognizers implemented to recognize 3D trajectories: $\$P^3$, $\$P+^3$, $\$Q^3$, $\$F$, $FH$ (FreeHandUni), $R3D$ (Rubine3D), and $RS$ (Rubine-Sheng).
    \item \textsc{Dataset}: This nominal variable with 6 conditions (see Table~\ref{tab:datasets-description} and Figure~\ref{fig:dataset_SHREC2019},\ref{fig:dataset_3DTCGS},\ref{fig:dataset_domains}), representing two datasets considered as a whole, namely SHREC2019~\cite{Caputo:2019} and 3DTCGS~\cite{Caputo:2017}, and four domains of 3DMadLabSD~\cite{Huang:2019}-D1, 2, 3, and 4. 
    \item \textsc{Number of Templates}: This numerical variable with 5 conditions, representing the number of templates per gesture class used to train the recognizer: $T{=}\{1,2,4,8,16\}$.
    \item \textsc{Sampling}: This numerical variable with 5 values representing the number of points per gesture: $N{=}\{4,8,16,32,64\}$.
\end{itemize}

\begin{table}[h]
\vspace{-10pt}
  \caption[Description of selected datasets.]{Description of selected datasets. Notation for the Ground truth information: Timestamp (T), Label (L), Hand joints (J).}
\resizebox{\columnwidth}{!}{
  \begin{tabular}{l|lcccc}
    \toprule
   \thead{\large \textbf{Name}}&\thead{\large \textbf{Sensor}}&\thead{\large \textbf{Subjects}}&\thead{\large \textbf{Classes}}&\thead{\large \textbf{Instances}}&\thead{\large \textbf{Ground truth} \\\large \textbf{information}}\\
    \midrule
    SHREC2019~\cite{Caputo:2019} & Leap Motion & 13 & 5 & 195 & T, L, J\\
    3DTCGS~\cite{Caputo:2017} & Leap Motion & 13 & 26 & 347 & T, L\\
    \makecell{3DMadLabSD~\cite{Huang:2019}} & \makecell[t]{SoftKinetic DepthSense DS325} & \makecell{10\\ } &   \makecell[t]{40 \\(10 x 4 Domains)} & \makecell{4000\\ } & T, L \\
  \bottomrule
\end{tabular}
} 
%  \vspace{-6pt}
  \label{tab:datasets-description}
\end{table}
\vspace{-12pt}
\subsubsection{Apparatus}
 We used a sexa-core Intel Core i7 2.20 GHz CPU and running a Windows 10 Home Edition operating system. The RAM was 16 GB DDR4  memory with 2400 MHz. We ran the framework described in Chapter~\ref{chap:QuantumLeapTesting} with the seven recognizers on the six individual datasets.



\subsection{Procedures and Quantitative Measures}\label{subsec:Comparative_scenarios}
We compute the \textit{recognition rate} (Which is the ratio of positive recognitions divided by the total number of trials) and the \textit{execution time} (measured in milliseconds as the time taken to preprocess a candidate gesture and recognize its result class) for the 7 (\textsc{Recognizer}) $\times$ 6 (\textsc{Dataset}) = 42 basic configurations following the typical procedure used in the literature to evaluate gesture recognizers~\cite{Akl:2010,Anthony:2010,Anthony:2012,Vatavu:2012b,Vatavu:2018,Wobbrock:2007}.
In the \enquote{user-independent scenario}, recognition is evaluated on gestures produced by users other than those used to train the recognizer. In the \enquote{user-dependent scenario}, recognition is evaluated on gestures produced by the same users used to train the recognizer. In total, 665,000 trials were performed.

\subsubsection{User-Independent Scenario}
For the user-independent scenario, one template is randomly selected for each gesture class from all participants and saved for testing. Then, a training set is created by randomly choosing $T$ templates for each gesture class across all users. These templates should be different from those selected for testing. The recognizer is then trained using the resulting training set.

To evaluate the performance, the recognizer is tested 100 times for each $T$ value: $T{=}\{1,2,4,8,16\}$ for SHREC2019 and MadLabSD, and $T{=}\{1,2,4,8,11\}$ for 3DTCGS. 
For the Rubine's recognizer 3D extensions ($R3D$ and $RS$),  \break $T{=}\{2,3,4,8,16\}$ for the SHREC2019 and MadLabSD datasets, and \break $T{=}\{2,3,4,8,11\}$ for the 3DTCGS dataset. The number of correctly recognized gestures is counted, and this is repeated for each $T$ value. Finally, the average recognition rate is calculated by averaging the number of recognized gestures and formatting it as a percentage for each $N$ and for each recognizer. In total,105,000 recognition trials were performed by running 6 (\textsc{Dataset}) $\times$ 5 (\textsc{Sampling}) $\times$ 5 (\textsc{Number of Templates}) $\times$ 100 (Repetitions) $\times$ 7 (\textsc{Recognizer}).

\subsubsection{User-Dependent Scenario}
For the user-dependent scenario, one template from each gesture class is randomly selected for each user. Then, for each user, a set of randomly selected $T$ templates is created for each gesture class, excluding the templates selected for testing by the same user. This training dataset is used to train the recognizer. 
Similar to the user-independent scenario, the recognizer is tested 100 times for each user and for each $T$ value, depending on the recognizer: $T{=}\{2,3,4,8\}$ for $R3D$ and $RS$, and $T{=}\{1,2,4,8\}$ for the others. The number of correctly recognized gestures is counted, and the average recognition rate is calculated by averaging the results for each user and $T$ value. The recognition rate is then presented in percentage format for each $N$ and for each recognizer.
Only the MadLabSD dataset was used because SHREC2019 does not identify the emitter of the gesture, and 3DTCGS contains only one template per user for each gesture class. In total, we performed 560,000 recognition trials by running 4 (\textsc{Dataset}) $\times$ 10 (Users) $\times$ 5 (\textsc{Sampling}) $\times$ 4 (\textsc{Number of Templates}) $\times$ 100 (Repetitions) $\times$ 7 (\textsc{Recognizer}).


\section{Results}
Table~\ref{tab:summary} provides a summary of these results for all datasets as well as for each individual dataset.

All reported results are averaged on all datasets, ranging from the simplest one to the most complex one (\textit{\eg}, Domain 4). 


\begin{table}[h]
%\vspace{-16pt}
  \caption[Summary of recognition rates and execution times for all datasets and per individual dataset.]{Summary of recognition rates and execution times for all datasets and per individual dataset. Recognizers are sorted in decreasing order of their average recognition rate for all datasets.}
\resizebox{\columnwidth}{!}{%
\begin{tabular}{@{}lrrrcrrrrrrrrrr@{}}
\toprule
\multicolumn{2}{c}{\textbf{Recognizer}}  &   & \phantom{.} &     \multicolumn{10}{c}{\textbf{Datasets}} \\
\cmidrule{3-15}
 & & \multicolumn{2}{c}{\textbf{Overall}} & & \multicolumn{1}{c}{\textbf{SHREC}} & \multicolumn{1}{c}{\textbf{3DTCGS}} & \multicolumn{2}{c}{\textbf{Domain 1}} & \multicolumn{2}{c}{\textbf{Domain 2}} & \multicolumn{2}{c}{\textbf{Domain 3}} & \multicolumn{2}{c}{\textbf{Domain 4}} \\
\cmidrule{3-4} \cmidrule{6-15}
& & \multicolumn{1}{c}{UD} & \multicolumn{1}{c}{UI} & & \multicolumn{1}{c}{UI} & \multicolumn{1}{c}{UI} & \multicolumn{1}{c}{UD} & \multicolumn{1}{c}{UI} & \multicolumn{1}{c}{UD} & \multicolumn{1}{c}{UI} & \multicolumn{1}{c}{UD} & \multicolumn{1}{c}{UI} & \multicolumn{1}{c}{UD} & \multicolumn{1}{c}{UI} \\
 \midrule
$\$P+^3$ & rate (M) & 74.40  & 87.48 & & 86.94 & 84.28 & 98.45 & 88.08 & 98.81 & 87.87 & 99.26 & 94.45 & 93.86 & 66.36 \\
         & rate (SD) & 1.73 & 13.68 & & 0.75 & 2.22 & 4.13 & 2.34 & 3.67 & 1.27 &  2.78 & 0.76 & 0.92 & 1.65 \\
\rowcolor{blue!45!gray!35}
      & time (M) &0.53	&	0.77	&	&	0.36	&	1.24	&	0.54	&	0.76	&	0.53	&	0.77	&	0.49	&	0.69	&	0.54	&	0.81 \\
\rowcolor{blue!45!gray!35}
    & time (SD) & 1.43	&	0.85	&	&	0.64	&	2.00	&	0.87	&	1.37	&	0.87	&	1.39	&	0.78	&	1.24	&	0.88	&	1.46
  \\
       \hline 
$\$F$ & rate (M) & 68.13 & 79.54 & & 79.36 & 78.83 & 96.22 & 76.38 & 96.78 & 75.45 & 97.67 & 85.66 & 91.76 & 57.94 \\
         & rate (SD) & 1.89 & 16.08 & & 3.55 & 2.76 & 6.98 & 1.63 & 6.70 & 3.28 & 5.26 & 1.28 & 1.07 & 1.68 \\
\rowcolor{blue!45!gray!35}
      & time (M) & 0.42	&	0.59	&	&	0.29	&	0.91	&	0.41	&	0.57	&	0.42	&	0.58	&	0.38	&	0.53	&	0.45	&	0.67 \\
\rowcolor{blue!45!gray!35}
    & time (SD) & 1.04	&	0.63	&	&	0.48	&	1.40	&	0.63	&	0.96	&	0.64	&	1.00	&	0.57	&	0.91	&	0.69	&	1.15
  \\
       \hline
$FH$ & rate (M) & 68.46 & 79.49 & & 79.28 & 78.84 & 96.21 & 76.05 & 96.69 & 75.78 & 97.70 & 85.55 & 91.96 & 57.75 \\
         & rate (SD) & 1.89 & 16.28 & & 0.93 & 2.80 & 6.97 & 3.20 & 6.79 & 3.24 & 5.28 & 1.26 & 1.04 & 1.70 \\

  \rowcolor{blue!45!gray!35}
      & time (M) &0.94	&	1.23	&	&	0.49	&	2.41	&	0.92	&	1.07	&	0.92	&	1.08	&	0.93	&	1.14	&	0.97	&	1.19 \\
\rowcolor{blue!45!gray!35}
    & time (SD) & 2.55	&	1.65	&	&	0.94	&	4.22	&	1.62	&	2.04	&	1.62	&	2.07	&	1.66	&	2.13	&	1.70	&	2.30  \\
         \hline
$\$Q^3$ & rate (M) & 67.36 & 79.09 & & 79.36 & 78.45 & 95.79 & 75.20 & 96.28 & 74.55 & 97.51 & 84.88 & 91.31 & 57.02 \\
         & rate (SD) & 1.91 & 16.54 & & 0.93 & 2.94 & 7.22 & 3.21 & 7.05 & 3.22 & 5.47 & 1.31 & 1.09 & 1.72 \\

  \rowcolor{blue!45!gray!35}
      & time (M) & 3.12	&	3.76	&	&	3.01	&	4.81	&	3.14	&	3.55	&	2.97	&	3.53	&	3.13	&	3.42	&	3.25	&	4.27 \\
\rowcolor{blue!45!gray!35}
    & time (SD) & 4.16	&	3.14	&	&	2.94	&	5.39	&	3.11	&	3.70	&	2.89	&	3.67	&	3.19	&	3.51	&	3.36	&	4.92
  \\
       \hline
$\$P^3$ & rate (M) & 65.90 & 77.50 & & 78.10 & 77.43 & 95.28 & 72.57 & 95.70 & 71.55 & 97.18 & 82.92 & 90.68 & 54.98 \\
         & rate (SD) & 1.93 & 16.88 & & 0.96 & 2.99 & 7.70 & 3.19 & 7.54 & 3.28 & 5.93 & 1.40 & 1.12 & 1.73 \\

    \rowcolor{blue!45!gray!35}
      & time (M) & 8.71	&	10.94	&	&	4.80	&	20.79	&	8.59	&	9.83	&	8.65	&	10.19	&	8.72	&	9.87	&	8.88	&	10.16 \\
\rowcolor{blue!45!gray!35}
    & time (SD) & 24.45	&	16.76	&	&	10.02	&	39.79	&	16.49	&	20.69	&	16.62	&	21.38	&	16.76	&	20.87	&	17.15	&	21.12\\
         \hline
$R3D$ & rate (M) & 60.80 & 67.22 & & 52.51 & 74.48 & 75.60 & 72.57 & 75.94 & 70.33 & 76.93 & 76.52 & 72.99 & 54.40 \\
         & rate (SD) & 2.98 & 24.93 & & 1.61 & 2.29 & 38.46 & 6.49 & 38.48 & 6.45 & 38.83 & 3.43 & 3.74 & 2.69 \\
\rowcolor{blue!45!gray!35}
      & time (M) & 0.08	&	0.08	&	&	0.05	&	0.12	&	0.06	&	0.07	&	0.06	&	0.06	&	0.07	&	0.08	&	0.10	&	0.10 \\
\rowcolor{blue!45!gray!35}
    & time (SD) & 0.027	&	0.016	&	&	0.013	&	0.018	&	0.004	&	0.009	&	0.005	&	0.005	&	0.005	&	0.009	&	0.002	&	0.007  \\
         \hline
$RS$ & rate (M) & 50.04 & 57.45 & & 40.82 & 68.82 & 69.07 & 58.37 & 70.76 & 58.10 & 73.36 & 67.98 & 65.04 & 43.49 \\
         & rate (SD) & 2.63 & 23.56 & & 1.51 & 2.50 & 36.18 & 5.77 & 36.86 & 5.66 & 38.01 & 3.24 & 3.48 & 2.35 \\
\rowcolor{blue!45!gray!35}
      & time (M) & 0.03	&	0.03	&	&	0.02	&	0.05	&	0.03	&	0.03	&	0.03	&	0.03	&	0.03	&	0.03	&	0.04	&	0.04 \\
\rowcolor{blue!45!gray!35}
    & time (SD) & 0.012	&	0.007	&	&	0.006	&	0.011	&	0.002	&	0.002	&	0.005	&	0.005	&	0.003	&	0.003	&	0.004	&	0.002 \\\bottomrule
\end{tabular}
}
\vspace{+4pt}
  \label{tab:summary}
  \vspace{-15pt}
\end{table}


\begin{table}[b!]
%\hspace{-80pt}
%\vspace{-8pt}
  \caption[ANOVAs computed for the 7 \textsc{Recognizers} in the user-independent scenario: G1=group 1, G2=group 2.]{ANOVAs computed for the 7 \textsc{Recognizers} in the user-independent scenario: G1=group 1, G2=group 2. For each dataset, three data are provided: the $q$ value resulting from the ANOVA, the significance of the \textit{p} value if any (***$p{\leq}$), and Cohen's d coefficient for effect size ((S)mall when $d{\geq}.02$, (M)edium when $d{\geq}.05$, (L)arge when $d{\geq}.08$,  and (--) when there is no significant effect size ($d{<}.02$)).}
     \resizebox{\columnwidth}{!}{
\begin{tabular}{ll|l|l|l|l|l|l|l}
\hline
\multicolumn{2}{c|}{\textbf{Recognizer}}         & \multicolumn{7}{c}{\textbf{Dataset (q-stat, \textit{p} value, Cohen's \textit{d})}}                                                                                                                                                         \\ \hline
\multicolumn{1}{c}{G1} & \multicolumn{1}{c|}{G2} & \multicolumn{1}{c|}{Overall} & \multicolumn{1}{c|}{SHREC2019} & \multicolumn{1}{c|}{3DTCGS} & \multicolumn{1}{c|}{Domain 1} & \multicolumn{1}{c|}{Domain 2} & \multicolumn{1}{c|}{Domain 3} & \multicolumn{1}{c}{Domain 4} \\ \hline
\rowcolor{blue!45!gray!35}$\$P+^3$             & $\$F$                     & 21.18,***, S            & 16.80,***, S            & 26.64,***,  M         & 27.48,***, M        & 29.11,***, M            & 21.30,***, M            & 21.35,***, S            \\ 
             \rowcolor{blue!45!gray!35}            & $FH$                   & 21.30,***, --           & 16.97,***,  S            &   26.63,***, M        & 28.15,***, M        & 28.33,***, M           &  21.57,***, M            & 21.83,***, S                              \\ 
       \rowcolor{blue!45!gray!35}                  & $\$Q^3$                & 22.89,***, --           & 16.80,***, S              &   28.50,***, M          & 30.23,***, M        & 31.22,***, M          &   23.19,***, S             &  23.70,***, M                             \\ 
        \rowcolor{blue!45!gray!35}                 & $\$P^3$                & 27.50,***, S            & 19.58,***, S             &    36.49,***,M          &  36.41,***, M       &  38.24,***, M          &  27.96,***, M            &  28.85,***, M                             \\ 
      \rowcolor{blue!45!gray!35}                   & $R3D$                       & 37.75,***,  S            &  76.28,***, L            &   47.90,***, L       &  36.42,***, M       &  41.11,***, L           &   43.47,***, L             &  30.32,***, M                             \\ 
        \rowcolor{blue!45!gray!35}                 & $RS$                      & 61.51,***, M            & 102.19,***, L            &  75.64,***, L     &  69.78,***, L        &  69.77,***, L          &   64.17,***, L              &  58.02,***, L                             \\ 
$\$F$                    & $FH$                    & 0.11, \textit{n.s.}            & 0.17, \textit{n.s.}                   &  0.01, \textit{n.s.}       &  0.77, \textit{n.s.}          &  0.77, \textit{n.s.}           &  0.27, \textit{n.s.}                 & 0.47, \textit{n.s.}                              \\ 
                         & $\$Q^3$                & 1.70, \textit{n.s.}                   & 0.01, \textit{n.s.}                & 1.85, \textit{n.s.}        &   2.75, \textit{n.s.}         &   2.10, \textit{n.s.}          &  1.89, \textit{n.s.}                 & 2.34, \textit{n.s.}                              \\ 
                         & $\$P^3$                & 6.31,***, M             & 2.78,***,  S             &  6.86,***, --               &  8.93,***, --          &   9.13,***, --          &  6.66,***, --                 & 7.50,***, --                              \\ 
                         & $R3D$                       & 16.56,***, L            & 59.48,***, L            &   21.28,***, S          & 8.93,***, --           &   12.00,***, S         & 22.17,***, S                & 8.97,***, --                              \\ 
                         & $RS$                      & 40.32,***, L            & 85.39,***, L           &  48.99,***, L          & 42.29,***, L         &   40.65, ***, L       &  42.87,***, L               & 36.67,***, M                              \\ 
\rowcolor{blue!45!gray!35}$FH$                  & $\$Q^3$                & 1.58, \textit{n.s.}             & 0.17, \textit{n.s.}             &   1.87, \textit{n.s.}                & 1.98, \textit{n.s.}          &   2.88, \textit{n.s.}           & 1.61, \textit{n.s.}                 &  2.34, \textit{n.s.}                             \\ 
                \rowcolor{blue!45!gray!35}         & $\$P^3$                & 6.19,***, M             & 2.60, \textit{n.s.}           &    6.86,***, --                & 8.14,***, --          &   9.91, ***, --          & 6.39,***, --                  &  7.02,***, --                             \\ 
            \rowcolor{blue!45!gray!35}             & $R$                       & 16.44,***, L            & 59.30,***, L           &   21.29,***, S           &  8.16,***, --          &  12.78,***, S           &  21.90,***, S               & 8.49,***, --                              \\ 
           \rowcolor{blue!45!gray!35}              & $RS$                      & 40.22,***, L            & 85.21,***, L           &  49.01,***, L          & 41.52,***, L         &  41.43,***, L           &  42.60,***, L              & 36.19,***, M                              \\ 
$\$Q^3$                 & $\$P^3$                & 4.61,***, S              & 2.78, \textit{n.s.}            &    4.98,**, --                  &  6.18,***, --         &   7.02,***, --            &  4.77,**, --                 & 5.15,**, --                              \\ 
                         & $R3D$                       & 14.86,***, L            & 59.48,***, L           & 19.42,***, S              &  6.18,***, --         &   9.89, ***, --            & 20.28,***, S              & 6.62,***, --                              \\ 
                         & $RS$                      & 36.61,***, L            & 85.39,***, L           &  47.14,***, L           &  39.54,***, M      &  38.54,***, M         & 40.98,***, L               & 34.32,***, M                              \\ 
\rowcolor{blue!45!gray!35}$\$P^3$                 & $R3D$                       & 10.24,***, L           & 56.59,***, L           &  14.43,***, S               &   0.01, \textit{n.s.}         &  2.86, ***, --          & 15.50,***, S               & 1.47, \textit{n.s.}                              \\ 
         \rowcolor{blue!45!gray!35}                & $RS$                      & 34.01,***, L            & 82.61,***,  L           &   42.15,***, L         &  33.36,***, M      & 31.52,***, M          &  36.20,***, M             & 29.17,***, M                              \\ 
$R3D$                        & $RS$                      & 23.75,***, L            & 25.91,***, M           &   27.71,***, M     & 33.36,***, M       &  28.65,***, M         &  20.69,***, S               & 27.69,***, M                              \\ \bottomrule
\end{tabular}
}

  \label{tab:anova_ui}
\end{table}
\subsection{User-Independent Scenario}
\subsubsection{Recognition Rate}
Recognizers are ranked in descending order according to their average recognition rate on all datasets, as shown in the following Figure~\ref{fig:Overall}): $\$P+^3$ is superior to $\$F$, then $FH$, $\$Q^3$, $\$P^3$, followed by the two Rubine recognizers $R3D$ and $RS$. $\$P+^3$ is 9.98\% more accurate than its successor $\$F$, which is roughly in the same interval as $FH$, $\$Q^3$, and $\$P^3$, which is in turn 15.29\% more accurate than $R3D$. 


The overall difference between the recognizers was statistically very highly significant ($F_{6,10459}=352.89$, $^{***}p{<}.001$, $\eta^2{=}.019$ (--)). Table~\ref{tab:anova_ui} details the results of ANOVAs for all datasets (column \enquote{Overall}) and per dataset. For example,  $\$P+^3$ is more accurate than $\$F$ with a very high significant difference ($q{=}21.18$) and a small effect size, more accurate than $FH$ with a very high significant difference but without effect size, more accurate than $\$Q^3$ without any effect size, and so forth. In short, $\$P+^3$ is more accurate than all other recognizers with a very highly significant difference, the effect size ranging from none to medium. The difference between $\$F$ and $FH$, $\$Q^3$, is not significant but becomes significant over $\$P^3$ with a medium effect size, over $R3D$, and $RS$ with a large effect size.

\begin{figure*}
    \centering
    \includegraphics[width=.54\textwidth]{Figures/Chap4/Overall.pdf}
    \vspace{-8pt}
    \caption{Recognition rates of all recognizers aggregated for all datasets in the user-independent scenario: individual rates (top) for $N{=}4,...,64$ (top) and global rates (bottom) $\forall N$. Error bars show their respective standard deviation.}
    \label{fig:Overall}
\end{figure*}


\subsubsection{Execution Time}
Figure~\ref{fig:ExecTime1} depicts the execution time in milliseconds for all recognizers in all conditions for the 3DTCGS and  SHREC2019 datasets.
\begin{itemize}
    \item Overall, $R3D$ ($M{=}.047$, $SD{=}.013$) and $RS$ ($M{=}.017$, $SD{=}.006$) have the lowest and the most constant times in all configurations.
    \item  On the contrary, the \$-like recognizers are the most time-consuming: $\$F$ ($M{=}.289$,  $SD{=}.490$) is the fastest recognizer among them, followed by $\$P+^3$ ($M{=}.363$, $SD{=}.649$), $FH$ ($M{=}.492$, $SD{=}.956$), $\$Q^3$ ($M{=}3.005$, $SD{=}3.005$) and $\$P^3$ ($M{=}4.798$, $SD{=}10.224$). 
    
    \item This difference suggests that feature-based recognizers, like $R3D$ and $RS$, are not influenced by the same parameters as template-based ones (\textit{\ie,} $\$P^3$, $\$P+^3$, $\$Q^3$, $\$F$, and $FH$).

    \item $RS$ is the fastest recognizer in most $N$ and $T$ conditions, except when $N{=}4$ and $T{\leq}4 $.

    \item The times for all \$-like recognizers increase while $N$ and $T$ grow.

    \item $\$Q^3$ is the slowest recognizer for $N{=}4,8$. Other curves are under the $\$Q^3$ one but from $N{=}16$, $\$P^3$ execution times increase quickly and exceed the execution time of $\$Q^3$  for $T{=}16$ and it occurs even quicker for $N{=}32$ and $N{=}64$, where $\$P^3$ exceeds $\$Q^3$ immediately after $T{=}4$, resp., $T{=}2$.

    \item $RS$ is three-time faster than $R3D$: the ratio of their averaged execution times is $\frac{t_RS}{t_R3D}{=}2.82$. This ratio makes sense considering that $R3D$ computes 39 features of the candidate (3 times 13 features for the three planes $XY$, $YZ$, and $ZX$) while $RS$ computes 16 features once. 
\end{itemize}
    
    For the feature-oriented recognizers, the candidate is preprocessed and its features are extracted. These features are then multiplied by precomputed weights for each class and summed. The candidate's class is designated by the larger sum resulting from this calculation, which confirms a constant time influenced by the number of classes.
    Among the \$-like recognizers, $\$F$, $\$P+^3$, and $FH$ show shorter execution times. Comparatively, $\$F$ and $\$P+^3$ outperform $FH$ due to the early abandoning strategy employed by these two recognizers.
    All conditions for this dataset have execution times below 100 ms, which is the limit for a user to perceive real-time execution of a system~\cite{Nielsen:1994}, as mentioned above.

\begin{figure*}
\vspace{-10pt}
    \centering
    \includegraphics[width=.97\textwidth]{Figures/Chap4/Time_Graphics_Shrec_3DTCGS.pdf}
    \vspace{-12pt}
    \caption{Execution times of all recognizers for two datasets (left: SHREC2019~\cite{Caputo:2019}, right: 3DTCGS~\cite{Caputo:2017}): individual times for $N{=}4,...,64$.}
    \label{fig:ExecTime1}
\end{figure*}

%\vspace{-29pt}

The overall picture for 3DTCGS remains the same as for SHREC2019. The execution times increase for all recognizers and conditions, causing $\$P^3$ to exceed the 100 millisecond limit. This is due to the increase of the number of classes and samples per class in the dataset (26 classes compared with 5 gestures for SHREC2019). We observe a rapid  increase of the $\$P^3$ curve that exceeds $\$Q^3$ just after $T{=}4$ for $N{=}8$.

The four 3DMadLab domains results are analogous to the first two datasets. Rubine-Sheng remains the fastest recognizer ($M_{D1}{=}.26$ ms, $M_{D2}{=}.28$ ms, $M_{D3}{=}.31$ ms, and $M_{D4}{=}.41ms$), whereas the $\$P^3$ stays the slowest one  ($M_{D1}{=}9.83$ ms, $M_{D2}{=}10.19$ ms, $M_{D3}{=}9.87$ ms, $M_{D4}{=}10.16$ ms). For the fourth domain containing the CAD symbols, the execution times are slightly above the other domains, probably because their shape complexity with more crossing points and angle variations require more comparisons.


\subsection{User-Dependent Scenario}
\subsubsection{Recognition Rate}
The recognizers are sorted in decreasing order of their recognition rate averaged on \textit{all} datasets as follows:
$\$P+^3$ is superior to $FH$, then $\$F$, $\$Q^3$, $\$P^3$, ended by the two Rubine conditions $R3D$ and $RS$. $\$P+^3$
is only 2.04\% more accurate than its successor $FH$, which is roughly in the same interval as $\$F$, $\$Q^3$, and $\$P^3$, which is in turn 25.66\% more accurate than $R3D$. 


The overall difference between the recognizers was again statistically very highly significant. Table~\ref{tab:anova_ud} details the results of ANOVAs for the four 3DMadLab domains (column \enquote{Overall}) and per domain. For example,  $\$P+^3$ is more accurate than $\$F$ ($q{=}14.38$), $FH$ ($q{=}13.62$), $\$Q^3$ ($q{=}16.13$), and $\$P^3$ ($q{=}19.50$), all with a very high significant difference and a small effect size. In short, $\$P+^3$ is more accurate than all other recognizers with a very high significant difference, the effect size ranging from small to large. Similarly, the difference between $\$F$ and $FH$, $\$Q^3$ is not significant, but becomes significant over $\$P^3$ without any effect size, over $R3D$ and $RS$ with a small and large effect size, and so forth.



\begin{table}[h]
%\vspace{-8pt}
  \caption[ANOVAs computed for the 7 \textsc{Recognizers} in the user-dependent scenario: G1=group 1, G2=group 2.]{ANOVAs computed for the 7 \textsc{Recognizers} in the user-dependent scenario: G1=group 1, G2=group 2. For each dataset, three data are provided: the $q$ value resulting from the ANOVA, the significance of the \textit{p} value if any (***$p{\leq}$), and Cohen's d coefficient for effect size ((S)mall when $d{\geq}.02$, (M)edium when $d{\geq}.05$, (L)arge when $d{\geq}.08$,  and (--) when there is no significant effect size ($d{<}.02$)).}
\resizebox{\columnwidth}{!}{%
\begin{tabular}{@{}ll|l|llll}
  \toprule
  \multicolumn{2}{c}{\textbf{Recognizer}}         & \multicolumn{5}{c}{\textbf{Dataset (q-stat, \textit{p} value, Cohen's \textit{d})}}                                                                                                                                                    \\  \hline    \cmidrule{1-7}
  \multicolumn{1}{c}{G1} & \multicolumn{1}{c}{G2} & \multicolumn{1}{|c|}{Overall}                               & \multicolumn{1}{c}{Domain 1} & \multicolumn{1}{c}{Domain 2} & \multicolumn{1}{c}{Domain 3} & \multicolumn{1}{c}{Domain 4} \\ \midrule
  $\$P+^3$               & $\$F$                  & \cellcolor[HTML]{bebee6}{\color[HTML]{333333}14.38,***, S}  & 15.21,***, S                 & 13.78,***, --                & 10.73,***, --                & 13.99,***, --                \\
                         & $FH$                   & \cellcolor[HTML]{bebee6}{\color[HTML]{333333}13.62,***, S}  & 15.27,***, S                 & 14.39,***, --                & 10.54,***, --                & 12.66,***, --                \\
                         & $\$Q^3$                & \cellcolor[HTML]{bebee6}{\color[HTML]{333333}16.13,***, S}  & 18.15,***, S                 & 17.15,***, --                & 11.78,***, --                & 16.98,***, --                \\
                         & $\$P^3$                & \cellcolor[HTML]{bebee6}{\color[HTML]{333333}19.50,***, S}  & 21.63,***, S                 & 21.10,***, --                & 14.04,***, --                & 21.16,***, --                \\
                         & $R3D$                  & \cellcolor[HTML]{bebee6}{\color[HTML]{333333}31.19,***,  M} & 155.71,***, L                & 154.95,***, L                & 150.50,***, L                & 138.79,***, L                \\
                         & $RS$                   & \cellcolor[HTML]{bebee6}{\color[HTML]{333333}55.87,***, L}  & 200.24,***, L                & 190.04,***, L                & 174.59,***, L                & 191.70,***, L                \\ \midrule
  $\$F$                  & $FH$                   & 0.76, \textit{n.s.}                                         & 0.05, \textit{n.s.}          & 0.61, \textit{n.s.}          & 0.18, \textit{n.s.}          & 1.32, \textit{n.s.}          \\
                         & $\$Q^3$                & 1.75, \textit{n.s.}                                         & 2.94, \textit{n.s.}          & 3.36, \textit{n.s.}          & 1.05, \textit{n.s.}          & 2.99, \textit{n.s.}          \\
                         & $\$P^3$                & \cellcolor[HTML]{bebee6}{\color[HTML]{333333}5.11,**, -- }  & 6.41,***, --                 & 7.32,***, --                 & 3.30, \textit{n.s.}          & 7.17,***, --                 \\
                         & $R3D$                  & \cellcolor[HTML]{bebee6}{\color[HTML]{333333}16.81,***, S}  & 140.49,***, L                & 141.16,***, L                & 139.77,***, L                & 124.80,***, L                \\
                         & $RS$                   & \cellcolor[HTML]{bebee6}{\color[HTML]{333333}41.48,***, L}  & 185.02,***, L                & 176.26, ***, L               & 163.85,***, L                & 177.70,***, L                \\ \midrule
  $FH$                   & $\$Q^3$                & 2.51, \textit{n.s.}                                         & 2.88, \textit{n.s.}          & 2.75, \textit{n.s.}          & 1.24, \textit{n.s.}          & 4.31,*, --                   \\
                         & $\$P^3$                & \cellcolor[HTML]{bebee6}{\color[HTML]{333333}5.87,***, --}  & 6.36,***, --                 & 6.70, ***, --                & 3.49, \textit{n.s.}          & 8.49,***, --                 \\
                         & $R3D$                  & \cellcolor[HTML]{bebee6}{\color[HTML]{333333}17.57,***, S}  & 140.44,***, L                & 140.55,***, L                & 139.96,***, L                & 126.13,***, L                \\
                         & $RS$                   & \cellcolor[HTML]{bebee6}{\color[HTML]{333333}42.25,***, L}  & 184.97,***, L                & 175.64,***, L                & 164.04,***, L                & 179.03,***, L                \\ \midrule
  $\$Q^3$                & $\$P^3$                & 3.36, \textit{n.s.}                                         & 3.47, \textit{n.s.}          & 3.95, \textit{n.s.}          & 2.25, \textit{n.s.}          & 4.18,*, --                   \\
                         & $R3D$                  & \cellcolor[HTML]{bebee6}{\color[HTML]{333333}15.06,***, S}  & 137.55,***, L                & 137.80, ***, L               & 138.72,***, L                & 121.81,***, L                \\
                         & $RS$                   & \cellcolor[HTML]{bebee6}{\color[HTML]{333333}39.73,***, L}  & 182.08,***, L                & 172.89,***, L                & 162.80,***, L                & 174.71,***, L                \\ \midrule
  $\$P^3$                & $R3D$                  & \cellcolor[HTML]{bebee6}{\color[HTML]{333333}11.69,***, S}  & 134.08,***, L                & 133.84, ***, L               & 136.46,***, L                & 117.63, ***, L               \\
                         & $RS$                   & \cellcolor[HTML]{bebee6}{\color[HTML]{333333}36.37,***, M}  & 178.61,***, L                & 168.93,***, L                & 160.54,***, L                & 150.53,***, L                \\ \midrule
  $R3D$                  & $RS$                   & \cellcolor[HTML]{bebee6}{\color[HTML]{333333}24.67,***, S}  & 44.52,***, S                 & 35.09,***, S                 & 24.08,***, --                & 52.90,***, S                 \\ \bottomrule
\end{tabular}
}

  \label{tab:anova_ud}
  \vspace{-16pt}
\end{table}
\break
\subsubsection{Execution Time}
The execution time curves of the four domains are depicted in Figure~\ref{fig:Time Exec UDep D1&2} and Figure~\ref{fig:Time Exec UDep 3&4}, resp. In contrast to the user-independent scenario, $RS$ remains the fastest recognizer in all conditions ($M_{D1}{=}.030$ ms, $M_{D2}{=}.027$ ms, $M_{D3}{=}.030$ ms, and $M_{D1}{=}.042 ms$). The $\$P^3$ curve is located under the $\$Q^3$ for $N{=}4$ ($M_{\$P^3}{=}.058$ ms and for $M_{\$Q^3}{=}.447$ ms) and $N{=}8$ ($M_{\$P^3}{=}.444$ ms and $M_{\$Q^3}{=}.853$ ms). 


The $\$P^3$ execution time curve progressively increases as $N$ increases and it intersects with the $\$Q^3$ at different points, the intersection point is determined by $T$.


The execution time for this scenario is larger than for the user-independent scenario because of the method used in JavaScript to return the timestamp during the recognition. For the user-independent scenario, we used \textsf{performance.now()}, which is a high-performance method returning a value representing the time spent since the beginning of the program with a precision of up to one microsecond, whereas in the user-dependent scenario, we used \textsf{date.now()} which is a method returning the number of milliseconds elapsed since UNIX epoch, which limits the precision by rounding it to 1 millisecond.



\begin{figure*}
\vspace{-10pt}
    \centering
    \includegraphics[width=.95\textwidth]{Figures/Chap4/Time_Graphics_UDep_MadLabS1&2.pdf}
    \vspace{-8pt}
    \caption{Recognition times of all recognizers for two datasets from MadLabSD (left: Domain 1, right: Domain 2~\cite{Huang:2019}): individual times for $N{=}4..64$ for the User-dependent scenario.}
    \label{fig:Time Exec UDep D1&2}
\end{figure*}

\begin{figure*}
\vspace{-10pt}
    \centering
    \includegraphics[width=.95\textwidth]{Figures/Chap4/Time_Graphics_UDep_MadLabS3&4.pdf}
    \vspace{-8pt}
    \caption{Recognition times of all recognizers for two datasets from MadLabSD (left: Domain 3, right: Domain 4~\cite{Huang:2019}): individual times for $N{=}4..64$ for the User-dependent scenario.}
    \label{fig:Time Exec UDep 3&4}
\end{figure*}


%\vspace{-8pt}
\vspace{-4pt}




% \begin{table}[]
% \begin{tabular}{@{}ccccc@{}}
% \toprule
% Recognizer & Effect & \textit{F} & \textit{p} & \eta \\ \midrule
% $\$P^3$ & Datasets $\times$ Rates & \textit{F}_{5,144} = 21.75 & \cellcolor[HTML]{C6E0B4}{\color[HTML]{333333} \textless{}.0001} & 0.43 (L) \\
%  & Datasets X Time & \textit{F}_{5,144} = 1.15 & \textit{n.s.} &  (--) \\
% $\$Q^3$ & Datasets $\times$ Rates & \textit{F}_{5,144} = 21.87 & \cellcolor[HTML]{C6E0B4}{\color[HTML]{333333} \textless{}.0001} & 0.43 (L) \\
%  & Datasets $\times$ Time & \textit{F}_{5,144} = 0.61 & \textit{n.s.} &  (--) \\
% $\$P+^3$ & Datasets $\times$ Rates & \textit{F}_{5,144} = 41.48 & \cellcolor[HTML]{C6E0B4}{\color[HTML]{333333} \textless{}.0001} & 0.59 (L) \\
%  & Datasets $\times$ Time & \textit{F}_{5,144} = 0.95 & \textit{n.s.} &  (--) \\
% $\$F$ & Datasets $\times$ Rates & \textit{F}_{5,144} = 22.1 & \cellcolor[HTML]{C6E0B4}{\color[HTML]{333333} \textless{}.0001} & 0.43 (L) \\
%  & Datasets $\times$ Time & \textit{F}_{5,144} = 0.93 & \textit{n.s.} &  (--) \\
% $FH$ & Datasets $\times$ Rates & \textit{F}_{5,144} = 22.48 & \cellcolor[HTML]{C6E0B4}{\color[HTML]{333333} \textless{}.0001} & 0.44 (L) \\
%  & Datasets $\times$ Time & \textit{F}_{5,144} = 1.55 & \textit{n.s.} &  (--) \\
% $R$ & Datasets $\times$ Rates & \textit{F}_{5,144} = 3.64 & \cellcolor[HTML]{C6E0B4}{\color[HTML]{333333} \textless{}.01} & 0.11 (L) \\
%  & Datasets $\times$ Time & \textit{F}_{5,144} = 147.18 & \cellcolor[HTML]{C6E0B4}{\color[HTML]{333333} \textless{}.0001} & 0.84 (L) \\
% $RS$ & Datasets $\times$ Rates & \textit{F}_{5,144} = 6.07 & \cellcolor[HTML]{C6E0B4}{\color[HTML]{333333} \textless{}.0001} & 0.17 (L) \\
%  & Datasets $\times$ Time & \textit{F}_{5,144} = 89.39 & \cellcolor[HTML]{C6E0B4}{\color[HTML]{333333} \textless{}.0001} & 0.76 (L) \\ \bottomrule
% \end{tabular}
% \caption{The effect of dataset on  the recognition rate and the execution time of the recognizers with the effect size for the user-independent scenario.}
%   \label{tab:effect_dataset}
% \end{table}

% \begin{table}[]
% \begin{tabular}{@{}ccccc@{}}
% \toprule
% Recognizer & Effect & \textit{F} & \textit{p} & \eta \\ \midrule
% $\$P^3$ & Datasets X Rates & \textit{F}_{3,76} = 8.66 & \cellcolor[HTML]{C6E0B4}{\color[HTML]{333333} \textless{}.0001} & 0.25 (L) \\
%  & Datasets X Time & \textit{F}_{3,76} = 0 & n.s. & 0 (--) \\
% $\$Q^3$ & Datasets X Rates & \textit{F}_{3,76} = 8.86 & \cellcolor[HTML]{C6E0B4}{\color[HTML]{333333} \textless{}.0001} & 0.26 (L) \\
%  & Datasets X Time & \textit{F}_{3,76} = 0.03 & n.s. & 0 (--) \\
% $\$P+^3$ & Datasets X Rates & \textit{F}_{3,76} = 20.9 & \cellcolor[HTML]{C6E0B4}{\color[HTML]{333333} \textless{}.0001} & 0.45 (L) \\
%  & Datasets X Time & \textit{F}_{3,76} = 0.02 & n.s. & 0 (--) \\
% $\$F$ & Datasets X Rates & \textit{F}_{3,76} = 8.45 & \cellcolor[HTML]{C6E0B4}{\color[HTML]{333333} \textless{}.0001} & 0.25 (L) \\
%  & Datasets X Time & \textit{F}_{3,76} = 0.04 & n.s. &  0 (--) \\
% $FH$ & Datasets X Rates & \textit{F}_{3,76} = 8.28 & \cellcolor[HTML]{C6E0B4}{\color[HTML]{333333} \textless{}.0001} & 0.25 (L) \\
%  & Datasets X Time & \textit{F}_{3,76} = 0 & n.s. & 0 (--) \\
% $R$ & Datasets X Rates & \textit{F}_{3,76} = 0.04 & n.s. & 0 (--) \\
%  & Datasets X Time & \textit{F}_{3,76} = 357.48 & \cellcolor[HTML]{C6E0B4}{\color[HTML]{333333} \textless{}.0001} & 0.93 (L) \\
% $RS$ & Datasets X Rates & \textit{F}_{3,76} = 0.19 & n.s. & 0.01 (--) \\
%  & Datasets X Time & \textit{F}_{3,76} = 65.55 & \cellcolor[HTML]{C6E0B4}{\color[HTML]{333333} \textless{}.0001} & 0.72 (L) \\ \bottomrule
% \end{tabular}
% \caption{The effect of dataset on  the recognition rate and the execution time of the recognizers with the effect size for the user-dependent scenario.}
%   \label{tab:effect_dataset_UDep}
% \end{table}





%Figure~\ref{fig:Compar1} plots the recognition rates for all recognizers on the two first datasets, \textit{\ie,}, SHREC2019 and 3DTCGS, side by side for comparison. 
%and the values obtained by the one-way ANOVAs computed for evaluating potential differences between recognizers. We use Tukey's HSD post-hoc analysis when Levene's test does not indicate unequal variances, and Games-Howell when it does.
% We first summarize the results for SHREC2019 dataset:
% \begin{itemize}
%     \item $\$P+^3$ turns out to be the most accurate recognizer in all conditions ($N{=}\{4,8,16,32,64\}$) in terms of absolute values for all values of $T$. When $N{=}64$, $\$P+^3$ is almost joined by $\$Q^3$, thus suggesting that $\$Q$ exhibits a better accuracy when the amount of templates is increasing. The more points are included for each template, the more accurate $\$Q^3$ becomes as it is the case for $\$Q$~\cite{Vatavu:2018}. $\$P+^3$ is appropriate for gestures with high sampling, even with a low number of templates (\textit{\eg}, $\tau{=}80.00\%$ with $T{=}4$, rapidly increasing above 90\%).  
    
%     \item $\$P+^3$ is significantly more accurate than all other recognizers. We computed Student's $t$ test for two paired samples since the samples were the same. There was a very highly significant difference in the global recognition rates for $\$P+^3$ ($M{=}86.94\%$, $SD{=}15.00\%$) and all the other conditions conditions (all with \textit{df}${=}2499$ and $p^{***}{<}.001$):\\
%     $\$F$ ($M{=}79.36\%$, $SD{=}18.08\%$), t$=\!18.74$, with a Cohen's~$d{=}.37$ representing a small effect~\cite{Cohen:1988};\\
%     $\$Q^3$ ($M{=}79.36\%$, $SD{=}18.60\%$), t$=\!18.67$, with a Cohen's~$d{=}.37$ representing a small effect~\cite{Cohen:1988};\\
%     $FH$ ($M{=}79.28\%$, $SD{=}18.51\%$), t$=\!18.20$, with a Cohen's~$d{=}.36$ representing a small effect~\cite{Cohen:1988};\\
%     $\$P^3$ ($M{=}78.10\%$, $SD{=}19.20\%$), t$=\!21.13$, with a Cohen's~$d{=}.42$ representing a small effect~\cite{Cohen:1988};\\
%     $R3D$ ($M{=}52.51\%$, $SD{=}32.24\%$), t$=\!52.35$, with a Cohen's~$d{=}1.05$ representing a large effect~\cite{Cohen:1988};\\
%     $RS$ ($M{=}40.82\%$, $SD{=}30.22\%$), t$=\!73.61$, with a Cohen's~$d{=}1.47$ representing a very large effect~\cite{Sawilowsky:2003}.
%     These results suggest that using the $\$P+^3$ recognizer has a positive effect on global recognition rates. This is also confirmed by the values obtained by a one-way ANOVA with Tukey's HSD post-hoc analysis: the between-group analysis also returned a very highly significant effect ($SS{=}11036.5$, \textit{df}${=}6$, $F{=}1444.58$, $p^{***}{<}.001$).
    
%     \item $\$P+^3$ could accommodate gestures with low sampling: already with one template on four points ($N{=}4$,  $T{=}1$), it benefits from a reasonable rate of $\tau{=}70\%$, the lowest value for this recognizer in all configuration. When $N{=}4$, better to have at least 8 points to reach a rate of $\tau{=}80\%$. When the number of points reaches 8 ($N{=}8$), two or more template are admissible to get a rate of at least ($\tau{=}86\%$). This observation seems to confirm the observation that for \$-family members, a sampling of 8 points already gives acceptable results~\cite{Vatavu:2012:Impact}. When $N{\geq}16$, one template is already enough and there is no need to go up to 16, even if the best rate is obtained when $N{=}32$ with $\tau{=}95.40\%$.
    
%     \item Apart from $\$P+^3$, all other \$-like recognizers display a global rate of $\tau{\geq}78\%$. Their differences are very light: they remain included in the same envelope with similar rates and progression when $N{\nearrow}$, $T{\nearrow}$ and the curves are almost superimposed all the time.
    
%     \item Although $\$P^3$ returned the lowest global rate of all \$-like recognizers, the one-way ANOVA did not return any significant difference between them, but all of them are always significantly more accurate than $R3D$ and $RS$.

%     \item Regarding the two feature-oriented recognizers, Rubine3D (R3D) is always more accurate than its counterpart Rubine-Sheng (RS) in all conditions. A Student's $t$ test for two paired samples revealed a very highly significant difference in the global recognition rates for $R3D$ ($M{=}52.51$, $SD{=}32.20$) and $RS$ ($M{=}40.82\%$, $SD{=}30.22\%$): $t{=}20.16$, $df{=}2499$, $p^{***}{<}.001$, with a Cohen's $d{=}.40$ representing a small effect. Both exhibit an acceptable rate only when $N{\geq}8$. Below this threshold, they are not useful. Surprisingly, when $N{=}4$, $R3D$ and $RS$ outperform other \$-like recognizers, apart from $\$P+^3$, when $T{\geq}4$. After that, they are obviously completely surpassed.
    
%     \item Rubine-Sheng is the least accurate recognizer in almost all conditions as it is outperformed by all other recognizers tested. Therefore, computing more geometric features for a 3D trajectory does not really help and keeping the original features in the three planes is enough to outperform Rubine-Sheng.
    
%     \item If we assume that a human-acceptable rate of $\tau{\geq}90\%$, the ideal cases are all when $T{=}16$ for all conditions.
% \end{itemize}

% For the 3DTCGS dataset, the results (see right part of Figure~\ref{fig:Compar1}) are quite similar to those obtained for the SHREC2019 dataset, except that, overall the curves increase more rapidly. The global recognition rates are quite similar for all \$-like recognizers, but are better for $R3D$ ($\tau{=}74.48\%$ instead of $\tau{=}52.51\%$) and for $RS$ ($\tau{=}68.80\%$ instead of $\tau{=}40.82\%$).

% Since we evaluated all recognizers against all datasets in this user-independent scenario, it is worth examining their aggregated recognition rates to provide figures applicable to the whole set in this most demanding scenario. Figure~\ref{fig:Overall} plots the recognition rates for the seven datasets considered in the user-independent scenario only with the following results:

% \begin{itemize}
%     \item $\$P+^3$ remains the most accurate recognizer in all conditions ($N{=}\{4,8,16,32,64\}$) in terms of absolute values for all values of $T$ and for its aggregated rate ($M{=}87.48\%$, $SD{=}13.50\%$). 
%     The rates of $\$F$ ($M{=}79.54\%$, $SD{=}16.98$), FreeHandUni ($M{=}79.49\%$, $SD{=}16.28\%$), $\$Q^3$ ($M{=}79.09\%$, $SD{=}16.54\%$), and $\$P^3$ ($M{=}77.50\%$, $SD{=}16.88$) fit in a pocket square. Their recognition curves are almost asymptotical to their maximum rate while $N$ increases. The two feature-oriented recognizers occupy the last two seats: Rubine3D ($M{=}67.22\%$, $SD{=}24.93\%$) and Rubine-Sheng ($M{=}57.45\%$, $SD{=}23.56\%$), being the least accurate recognizer in this scenario.
    
%     \item $\$P+^3$ is significantly more accurate than all other recognizers. A Student's $t$ test for two paired samples reveals that there was a very highly significant difference in the global recognition rates for $\$P+^3$ and all the other conditions conditions (all with \textit{df}${=}14,999$ with $p^{***}{<}.001$):
%     \begin{itemize}
%     \item 
%     for $\$F$ (t${=}66.53$), with a medium effect size for both Pearson's correlation ($r{=}.48$~\cite{Cohen:1992}) and Cohen's~($d{=}.54$~\cite{Cohen:1988});
%     \item
%     for FreeHandUni (t${=}66.53$), with a medium effect size for both Pearson's correlation ($r{=}.48$~\cite{Cohen:1992}) and Cohen's~difference ($d{=}.54$~\cite{Cohen:1988});
%     \item
%     for $\$Q^3$ (t${=}70.02$), with a large effect size for Pearson's correlation ($r{=}.50$~\cite{Cohen:1992}) and a medium one for Cohen's difference ($d{=}.57$~\cite{Cohen:1988});
%     \item
%     for $R3D$ (t${=}79.06$), with a large effect size for Pearson's correlation ($r{=}.54$~\cite{Cohen:1992}) and a medium one for Cohen's difference ($d{=}.64$~\cite{Cohen:1988});
%     \item
%     for $RS$ (t${=}137.79$), with a large efect size for Pearson's correlation ($r{=}.75$ ~\cite{Cohen:1992}) and a medium one for Cohen's difference ($d{=}1.12$~\cite{Cohen:1988}).
%     \end{itemize}
%     \item Although the recognition rate of the four recognizers ranked 2-5 belong to the same small range, they differ statistically in a few conditions: $\$F$ is not significantly superior to FreeHandUni, but is very highly statistically superior to $\Q^3$ (t${=}5.32$ with a very small effect $d{=}.04$), to $\$P^3$ (t${=}19.80$ with a very small effect $d{=}.16$), to $R3D$ (t${=}34.43$ with a small effect $d{=}.28$) and to $R3D$ (t${=}90.41$ with a medium effect $d{=}.73$).  
%     \item The error bars of Figure~\ref{fig:Overall} depict the standard deviation of their respective rates computed on all datasets instead of a confidence interval. These confidence intervals computed with $\alpha{=}.05$ were so small that they almost impossible to render on the figure. We notice that $\$P+^3$ benefits from the smallest standard deviation (from about 75\% to about 99\%) on its rate distribution. Subsequent \$-like recognizers see their standard deviation belonging to a very similar interval without any particular significant difference. However, Rubine3D has the largest standard deviation followed by Rubine-Sheng, having its deviation spanning from the minimum value to about 80\% at best. The standard deviation of recognizers tends to increase as their recognition rate decreases (from 13.50\% to 24.93\%), which suggests that their overall recognition behavior is more constant.
% \end{itemize}

 


\subsubsection{User-dependent \textit{vs.} User-independent Scenario}
\begin{itemize}
    \item The $\$P+^3$ recognizer remains the most accurate for the first domain (\textit{\ie,} the ten digits) in both user-dependent ($M{=}98.45\%$) and user-independent scenarios ($M{=}88.08\%)$.
    \item  For all values of $N$,  a high recognition rate is obtained as soon as with one template and the best recognition rate for more templates.
    \item With two or more templates ($T{>}2$) used for training, $R3D$'s curve surpasses all the other curves  except the $\$P+^3$ in the user-independent scenario and is competitive with other \$-like recognizers in the user-dependent scenario.
    \item With less than three templates, $R3D$ is inaccurate, which explains why its overall rate is inferior to the others, except for $RS$. 
    \item In the user-independent scenario, \$-like recognizers belong to an envelope that progressively grows when $N$ grows.
    \item In the user-dependent scenario, their curves are confounded most of the time.
    \item The recognition rates of the second domain, \textit{\ie,} the lowercase letters a-j,  are quite similar to the first domain with a margin of about 1-2\%.
    \item For the third domain, \textit{\ie,} the simulation symbols, \$-like recognizers are close to perfection in the user-dependent scenario with $\$P+^3$ being the champion in both cases ($M{=}99.26\%$ and $M{=}94.45\%$, respectively).
    \item The results for the user-independent scenario are slightly superior to those obtained for the previous domains.
    \item However,  the rates are very low for the fourth domain (\textit{\ie,} the CAD symbols) in the user-independent scenario (they are all below $66\%$), but still very good for the user-dependent scenario (between $M{=}93.86\%$ for $\$P+^3$ and $M{=}90.68\%$ for $\$P^3$).
\end{itemize}
   
   
%This domain probably contains the least familiar symbols. 


For all four domains, the recognition rate significantly decreases when we are switching from the user-dependent scenario to the user-independent scenario, which was expected. Recognizing a 3D trajectory that is different from the templates used for training remains more demanding than re-identifying an already existing one. Figure~\ref{fig:RateDifferences} depicts the loss of recognition rate in terms of a difference of percentage from user-dependent to user-independent. $R3D$ wins the lowest averaged difference of percentage ($M{=}11.75\%$), but its global recognition rate is below the \$-like recognizers. The second place is occupied by $\$P+^3$, which undergoes the next lowest averaged difference of percentage ($M{=}17.50\%$ on four domains), followed by $\$F$, $FH$, $\$P+^3$, $\$P^3$, and $RS$.

In conclusion, $\$P+^3$ reasonably resists to user-independence while keeping the best recognition rate. The two first domains, \textit{\ie,} digits and letters, remain constant in terms of recognition loss for each recognizer, again with  a loss of 12\% for $\$P+^3$. Overall, the third domain benefits from the minimum loss, probably because of straightforward symbols, and the fourth domain suffers from the maximum loss, probably because of the most uncommon symbols. 
\begin{figure*}[h]
    \centering
%    \vspace{-24pt}
    \includegraphics[width=\textwidth]{Figures/Chap4/RateDifferences.pdf}
    \vspace{-18pt}
    \caption{Difference of percentage in recognition rate loss for all recognizers on the MadLabSD dataset~\cite{Huang:2019}, all domains, when restricting the scenario from user-dependent to user-independent. Recognizers are sorted in decreasing order of their global recognition rate.}
    \label{fig:RateDifferences}
    \vspace{-10pt}
\end{figure*}



\subsection{Impact on Recognition Rate and Execution Time}
We conducted a series of one-way ANOVAs using IBM SPSS V27 to evaluate the effect of the recognizer, number of templates $T$, and sampling $N$ on recognition rate and execution time. When Levene's test~\cite{Levene:1960} indicated unequal variances, we used Games-Howell post-hoc analysis. Otherwise, we used Tukey's HSD. Before analysis, the data underwent a Bonferroni type I correction.


\subsubsection{Impact of Datasets on Recognition Rate and Execution Time}
To determine the potential impact of the datasets (\textit{\ie,} SHREC2019, 3DTCGS, MadLabSD Domains 1--4 in the user-independent scenario and MadLabSD Domains 1--4 in the user-dependent scenario) on the recognition rate and the execution time, we computed another series of one-way ANOVAs (Table~\ref{tab:effect_dataset}). 
\begin{itemize}
    \item The analyses showed a significant effect of the datasets on the recognition rate with a large effect size, but no significant effect on the execution time in both scenarios.
    \item A medium effect size in the user-independent scenario was observed on the $R3D$ recognition rate, but no significant effect on it and on $RS$ recognition rate in the user-dependent scenario.
    \item  Tukey’s HSD and Games-Howell post-hoc analyses revealed that the recognizers got the lowest rate for Domain 4 in the two scenarios. For instance, there were significant differences for the $\$P+^3$ recognition rate in user-independent scenario  between the different datasets, Domain 4 \textit{vs.} SHREC2019 ($M_{\text{Diff}}{=}-20.588$,$^{***}p{<}.001$), Domain 4 \textit{vs.} 3DTCGS ($M_{\text{Diff}}{=}-17.923$,$^{***}p{<}.001$), Domain 4 \textit{vs.} Domain 1 ($M_{\text{Diff}}{=}-21.720$,$^{***}p{<}.001$), Domain 4 \textit{vs.} Domain 2 ($M_{\text{Diff}}{=} -21.512$,$^{***}p{<}.001$), Domain 4 \textit{vs.} Domain 3 ($M_{\text{Diff}} {=} -28.092$,$^{***}p{<}.001$), and Domain 4 \textit{vs.} 3DMadLab ($M_{\text{Diff}}{=}-20.588$,$^{***}p{<}.001$). 
    \item There was also a significant effect of the datasets on $R3D$ and $RS$ execution time. The $R3D$ execution times are significantly different between each pair of datasets.
\end{itemize}


\begin{table}[t]
\caption[The effect of dataset on the recognition rate and the execution time with the effect size]{The effect of dataset on the recognition rate and the execution time with the effect size ( (L)arge when $\eta^2{\geq}.14$, (M)edium when $\eta^2{\geq}.06$, (S)mall when $\eta^2{\geq}.01$, and (–-) when there is a null effect size).}
 \begin{adjustbox}{max width=\textwidth}
\begin{tabular}{@{}ll|rcrc|rcr@{}}
\toprule
%Recognizer & Effect & \textit{F} & \textit{p} & \eta \\ 
\textbf{Recognizer} & \textbf{Effect} & \multicolumn{3}{c}{\textbf{User independent}} & \phantom{a} & \multicolumn{3}{c}{\textbf{User dependent}} \\ 
\cmidrule{3-5} \cmidrule{7-9}
           &        &  \multicolumn{1}{c}{$F_{5,144}$} & \multicolumn{1}{c}{$p$} & \multicolumn{1}{c}{$\eta^2$} && \multicolumn{1}{c}{$F_{3,76}$} & \multicolumn{1}{c}{$p$} & \multicolumn{1}{c}{$\eta^2$} \\
\midrule
$\$P^3$ & Datasets $\times$ Rates & 21.75 & \cellcolor[HTML]{bebee6}{\color[HTML]{333333} $^{***}$} & .43 (L) &&   8.66 & \cellcolor[HTML]{bebee6}{\color[HTML]{333333} $^{***}$} & .25 (L)\\
 & Datasets $\times$ Time & 1.15 & \textit{n.s.} & .04 (--) &&   0 & \textit{n.s.} & 0 (--) \\
$\$Q^3$ & Datasets $\times$ Rates & 21.87 & \cellcolor[HTML]{bebee6}{\color[HTML]{333333} $^{***}$} & .43 (L) &&   8.86 & \cellcolor[HTML]{bebee6}{\color[HTML]{333333} $^{***}$} & .26 (L) \\
 & Datasets $\times$ Time &  0.61 & \textit{n.s.} & .02 (--) &&  0 &  \textit{n.s.} & 0 (--) \\
$\$P+^3$ & Datasets $\times$ Rates & 41.48 & \cellcolor[HTML]{bebee6}{\color[HTML]{333333} $^{***}$} & .59 (L) && 20.9 & \cellcolor[HTML]{bebee6}{\color[HTML]{333333} $^{***}$} & .45 (L) \\
 & Datasets $\times$ Time & 0.95 & \textit{n.s.} & .03 (--) &&   0 &  \textit{n.s.} & 0 (--)\\
$\$F$ & Datasets $\times$ Rates & 22.1 & \cellcolor[HTML]{bebee6}{\color[HTML]{333333} $^{***}$} & .43 (L) && 8.45  &\cellcolor[HTML]{bebee6}{\color[HTML]{333333} $^{***}$} & .25 (L)   \\
 & Datasets $\times$ Time & 0.93 & \textit{n.s.} & .03 (--) &&   0 &  \textit{n.s.} & 0 (--)\\
$FH$ & Datasets $\times$ Rates & 22.48 & \cellcolor[HTML]{bebee6}{\color[HTML]{333333} $^{***}$} & .44 (L) && 8.28   & \cellcolor[HTML]{bebee6}{\color[HTML]{333333} $^{***}$} & .25 (L)  \\
 & Datasets $\times$ Time & 1.55 & \textit{n.s.} & .05 (--) &&   0 &  \textit{n.s.} & 0 (--)\\
$R3D$ & Datasets $\times$ Rates & 3.64 & \cellcolor[HTML]{bebee6}{\color[HTML]{333333} $^{**}$} & .11 (M) &&  0 &  \textit{n.s.} & 0 (--)\\
 & Datasets $\times$ Time & 147.18 & \cellcolor[HTML]{bebee6}{\color[HTML]{333333} $^{***}$} & .84 (L) && 357.48  &  \cellcolor[HTML]{bebee6}{\color[HTML]{333333} $^{**}$} & .93 (L) \\
$RS$ & Datasets $\times$ Rates & 6.07 & \cellcolor[HTML]{bebee6}{\color[HTML]{333333} $^{***}$} & .17 (L) && .19  0 &  \textit{n.s.} & .01 (--)\\
 & Datasets $\times$ Time &  89.39 & \cellcolor[HTML]{bebee6}{\color[HTML]{333333} $^{***}$} & .76 (L) && 65.55  & \cellcolor[HTML]{bebee6}{\color[HTML]{333333} $^{***}$} & .72 (L)\\
 \bottomrule
\end{tabular}
\end{adjustbox}
  \label{tab:effect_dataset}
  \vspace{-16pt}
\end{table}
  
\subsubsection{Impact of Number of Templates and Sampling on Recognition Rate and Execution Time}
Both  $T$ and  $N$ have a significant impact on the recognition rate and  execution time of most recognizers (Table~\ref{tab:effect_dataset2}):
\begin{itemize}
    \item Regarding the $\$P^3$, $\$Q^3$, $\$P+^3$, $\$F$, and $FH$, we observe a significant effect of both factors, with a large effect size in the two scenarios. The Games-Howell analyses revealed that the execution time is significantly slower as the number of points increases.
    \item The $\$Q^3$ execution time is not significantly impacted by the number of templates in both scenarios. The effect of the number of templates on the execution time of $\$F$ is significant with a medium effect size, the Game-Howell post-hoc analysis revealed no significant difference between the execution times of each $T$ in the user-dependent scenario. 
    \item As per $R3D$ and $RS$, only $T$ had a significant effect on the recognition rates with a large effect size. The Game-Howell confirms that the recognizers are more accurate with more templates.
\end{itemize}

\begin{table}[ht!]
\vspace{-10px}
\caption[The effect of number of templates and points on  the recognition rate and the execution time with the effect size]{The effect of number of templates and points on  the recognition rate and the execution time with the effect size ( (L)arge when $\eta^2{\geq}.14$,(M)edium $\eta^2{\geq}.06$, (S)mall $\eta^2{\geq}.01$, and (–) when null effect size).}
 \begin{adjustbox}{max width=\textwidth}
\begin{tabular}{@{}llrcrcrcr@{}}
\toprule
%Recognizer & Effect & \textit{F} & \textit{p} & \eta \\ 
\textbf{Recognizer} & \textbf{Effect} & \multicolumn{3}{c}{\textbf{User independent}} & \phantom{a} & \multicolumn{3}{c}{\textbf{User dependent}} \\ 
\cmidrule{3-5} \cmidrule{7-9}
           &        &  \multicolumn{1}{c}{$F_{4,145}$} & \multicolumn{1}{c}{$p$} & \multicolumn{1}{c}{$\eta^2$} && \multicolumn{1}{c}{$F_{3,76}$} & \multicolumn{1}{c}{$p$} & \multicolumn{1}{c}{$\eta^2$} \\
\midrule
$\$P^3$ & \cellcolor[HTML]{bebee6}{Templates $\times$ Rates} & 16.31 & \cellcolor[HTML]{bebee6}{\color[HTML]{333333} $^{***}$} & .31 (L) &&   16.80 & \cellcolor[HTML]{bebee6}{\color[HTML]{333333} $^{***}$} & .40 (L)\\
 & \cellcolor[HTML]{bebee6}{Points $\times$ Rates} & 9.40 & \cellcolor[HTML]{bebee6}{\color[HTML]{333333} $^{***}$} & .21 (L)  &&  5.98 & \cellcolor[HTML]{bebee6}{\color[HTML]{333333} $^{***}$} & .24 (L) \\
 & Templates $\times$ Time & 5.69 &  \cellcolor[HTML]{bebee6}{\color[HTML]{333333} $^{***}$} & .14 (L) &&   4.04 & \cellcolor[HTML]{bebee6}{\color[HTML]{333333} $^{***}$} & .14 (L) \\
 & Points $\times$ Time & 25.27 &  \cellcolor[HTML]{bebee6}{\color[HTML]{333333} $^{***}$} & .41 (L) &&   24.87 & \cellcolor[HTML]{bebee6}{\color[HTML]{333333} $^{***}$} & .57 (L) \\
$\$Q^3$ & \cellcolor[HTML]{bebee6}{Templates $\times$ Rates} & 15.31 & \cellcolor[HTML]{bebee6}{\color[HTML]{333333} $^{***}$} & .30 (L) &&   15.44 & \cellcolor[HTML]{bebee6}{\color[HTML]{333333} $^{***}$} & .38 (L)\\
 & \cellcolor[HTML]{bebee6}{Points $\times$ Rates} & 9.90 & \cellcolor[HTML]{bebee6}{\color[HTML]{333333} $^{***}$} & .21 (L) &&   6.22 & \cellcolor[HTML]{bebee6}{\color[HTML]{333333} $^{***}$} & .25 (L) \\
 & Templates $\times$ Time & 0.64 & \textit{n.s.} & .02 (--) &&   0.09 & \textit{n.s.} & 0 (--) \\
 & Points $\times$ Time & 311.36 & \cellcolor[HTML]{bebee6}{\color[HTML]{333333} $^{***}$} & .90 (L) &&   1587.00 & \cellcolor[HTML]{bebee6}{\color[HTML]{333333} $^{***}$} & .99 (L) \\
$\$P+^3$ & \cellcolor[HTML]{bebee6}{Templates $\times$ Rates} & 11.46 & \cellcolor[HTML]{bebee6}{\color[HTML]{333333} $^{***}$} & .24 (L) &&   7.99 & \cellcolor[HTML]{bebee6}{\color[HTML]{333333} $^{***}$} & .24 (L)\\
 & \cellcolor[HTML]{bebee6}{Points $\times$ Rates} & 2.65 & \cellcolor[HTML]{bebee6}{\color[HTML]{333333} $^{*}$} & .07 (M) &&   2.16 & \textit{n.s.} & .10 (-) \\
 & Templates $\times$ Time & 6.26 & \cellcolor[HTML]{bebee6}{\color[HTML]{333333} $^{***}$} & .15 (L) &&   3.51 & \cellcolor[HTML]{bebee6}{\color[HTML]{333333} $^{*}$} & .12 (M) \\
 & Points $\times$ Time & 33.96 & \cellcolor[HTML]{bebee6}{\color[HTML]{333333} $^{***}$} & .48 (L) &&  36.02 & \cellcolor[HTML]{bebee6}{\color[HTML]{333333} $^{***}$} & .66 (L)\\
$\$F$ & \cellcolor[HTML]{bebee6}{Templates $\times$ Rates} & 16.77 & \cellcolor[HTML]{bebee6}{\color[HTML]{333333} $^{***}$} & .32 (L) &&  13.89 & \cellcolor[HTML]{bebee6}{\color[HTML]{333333} $^{***}$} & .36 (L)\\
 & \cellcolor[HTML]{bebee6}{Points $\times$ Rates} & 8.66 & \cellcolor[HTML]{bebee6}{\color[HTML]{333333} $^{***}$} & .19 (L) &&   6.88 & \cellcolor[HTML]{bebee6}{\color[HTML]{333333} $^{***}$} & .27 (L) \\
 & Templates $\times$ Time& 6.70 & \cellcolor[HTML]{bebee6}{\color[HTML]{333333} $^{***}$} & .16 (L) &&   3.52 & \cellcolor[HTML]{bebee6}{\color[HTML]{333333} $^{*}$} & .12 (M) \\
 & Points $\times$ Time & 34.95 & \cellcolor[HTML]{bebee6}{\color[HTML]{333333} $^{***}$} & .49 (L) &&   39.12 & \cellcolor[HTML]{bebee6}{\color[HTML]{333333} $^{***}$} & .68 (L) \\
$FH$ & \cellcolor[HTML]{bebee6}{Templates $\times$ Rates} & 16.25 & \cellcolor[HTML]{bebee6}{\color[HTML]{333333} $^{***}$} & .31 (L) &&  13.83 & \cellcolor[HTML]{bebee6}{\color[HTML]{333333} $^{***}$} & .35 (L)\\
 & \cellcolor[HTML]{bebee6}{Points $\times$ Rates} & 8.79 & \cellcolor[HTML]{bebee6}{\color[HTML]{333333} $^{***}$} & .20 (L) &&   7.26 & \cellcolor[HTML]{bebee6}{\color[HTML]{333333} $^{***}$} & .28 (L) \\
 & Templates $\times$ Time & 6.62 & \cellcolor[HTML]{bebee6}{\color[HTML]{333333} $^{***}$} & .15 (L) &&   4.76 & \cellcolor[HTML]{bebee6}{\color[HTML]{333333} $^{**}$} & .16 (L) \\
 & Points $\times$ Time & 22.65 & \cellcolor[HTML]{bebee6}{\color[HTML]{333333} $^{***}$} & .39 (L) &&   23.83 & \cellcolor[HTML]{bebee6}{\color[HTML]{333333} $^{***}$} & .56 (L) \\
$R3D$  & \cellcolor[HTML]{bebee6}{Templates $\times$ Rates} & 89.56 & \cellcolor[HTML]{bebee6}{\color[HTML]{333333} $^{***}$} & .71 (L) &&   12047.47 & \cellcolor[HTML]{bebee6}{\color[HTML]{333333} $^{***}$} & 1 (L)\\
 & \cellcolor[HTML]{bebee6}{Points $\times$ Rates} & 0 & \textit{n.s.} & 0 (--) &&   0 & \textit{n.s.} & 0 (--) \\
 & Templates $\times$ Time & 0.42 & \textit{n.s.} & .01 (--) &&   0.13 & \textit{n.s.} & .01 (--) \\
 & Points $\times$ Time & 0.14 & \textit{n.s.} &  0 (--) &&   0.1 & \textit{n.s.} & .01 (--) \\
$RS$ & \cellcolor[HTML]{bebee6}{Templates $\times$ Rates} & 70.35 & \cellcolor[HTML]{bebee6}{\color[HTML]{333333} $^{***}$} & .66 (L) &&   2258.11 & \cellcolor[HTML]{bebee6}{\color[HTML]{333333} $^{***}$} & .99 (L)\\
 & \cellcolor[HTML]{bebee6}{Points $\times$ Rates} & 0 & \textit{n.s.} & 0 (--) &&   0.09 & \textit{n.s.} & 0 (--) \\
 & Templates $\times$ Time & 1.21 & \textit{n.s.} & .03 (--) &&   0.37 & \textit{n.s.} & .01 (--) \\
 & Points $\times$ Time & 0.19 & \textit{n.s.} & .01 (--) &&   0.10 & \textit{n.s.} & .01 (--) \\
 \bottomrule
\end{tabular}
\end{adjustbox}
  \label{tab:effect_dataset2}
  \vspace{5pt}
\end{table}



% \begin{table}[b!]
% \hspace{50pt}
% \vspace{-8pt}
% \begin{tabular}{@{}llclcl@{}}
% \toprule
% \multicolumn{2}{c}{\textbf{Recognizer}} & \phantom{a}        & \multicolumn{3}{c}{\textbf{Scenario}}                  \\
% \cmidrule{1-2} \cmidrule{4-6}
% \multicolumn{1}{c}{G1} & \multicolumn{1}{c}{G2} & & \multicolumn{1}{c}{UI} & \phantom{a}  & \multicolumn{1}{c}{UD}\\ \midrule
% $\$P+^3$             & $\$F$                     & & \cellcolor[HTML]{C6E0B4}{\color[HTML]{333333} 21.18,***, S}            &  & \cellcolor[HTML]{C6E0B4}{\color[HTML]{C6E0B4}}{\color[HTML]{333333} 14.38,***, S}            \\ 
%                          & $FH$                   & & \cellcolor[HTML]{C6E0B4}{\color[HTML]{333333}21.30,***, --}           & & \cellcolor[HTML]{C6E0B4}{\color[HTML]{C6E0B4}}{\color[HTML]{333333} 13.62,***, S}            \\
%                          & $\$Q^3$                & & \cellcolor[HTML]{C6E0B4}{\color[HTML]{333333}22.89,***, --}           & & \cellcolor[HTML]{C6E0B4}{\color[HTML]{C6E0B4}}{\color[HTML]{333333} 16.13,***, S}              \\ 
%                          & $\$P^3$                & & \cellcolor[HTML]{C6E0B4}{\color[HTML]{333333}27.50,***, S }           & & \cellcolor[HTML]{C6E0B4}{\color[HTML]{C6E0B4}}{\color[HTML]{333333} 19.50,***, S}   \\ 
%                          & $R3D$                     &  & \cellcolor[HTML]{C6E0B4}{\color[HTML]{333333}37.75,***,  S}        &    &  \cellcolor[HTML]{C6E0B4}{\color[HTML]{C6E0B4}}{\color[HTML]{333333} 31.19,***, M} \\ 
%                          & $RS$                   &   & \cellcolor[HTML]{C6E0B4}{\color[HTML]{333333}61.51,***, M}       &     & \cellcolor[HTML]{C6E0B4}{\color[HTML]{C6E0B4}}{\color[HTML]{333333} 21.18,***, S}           \\ \midrule
% $\$F$                    & $FH$                 &   & 0.11, \textit{n.s.}        &    & 0.76, \textit{n.s.}                                              \\ 
%                          & $\$Q^3$                & & 1.70, \textit{n.s.}            &       & 1.75, \textit{n.s.}                  \\ 
%                          & $\$P^3$                & & \cellcolor[HTML]{C6E0B4}{\color[HTML]{333333}6.31,***, M}            &  & \cellcolor[HTML]{C6E0B4}{\color[HTML]{333333}5.11,**, --}            \\ 
%                          & $R3D$                     &  & \cellcolor[HTML]{C6E0B4}{\color[HTML]{333333}16.56,***, L}         &   & \cellcolor[HTML]{C6E0B4}{\color[HTML]{333333}16.81,***, S}            \\ 
%                          & $RS$                  &    & \cellcolor[HTML]{C6E0B4}{\color[HTML]{333333}40.32,***, L}         &   & \cellcolor[HTML]{C6E0B4}{\color[HTML]{333333}41.48,***, L}           \\ \midrule
% $FH$                  & $\$Q^3$                & & 1.58, \textit{n.s.}            &  & 2.51, \textit{n.s.}                                          \\ 
%                          & $\$P^3$                & & \cellcolor[HTML]{C6E0B4}{\color[HTML]{333333}6.19,***, M}          &   & \cellcolor[HTML]{C6E0B4}{\color[HTML]{333333}5.87,***, --}             \\ 
%                          & $R3D$                    &   & \cellcolor[HTML]{C6E0B4}{\color[HTML]{333333}16.44,***, L}        &    & \cellcolor[HTML]{C6E0B4}{\color[HTML]{333333}17.57,***, S}            \\ 
%                          & $RS$                   &   & \cellcolor[HTML]{C6E0B4}{\color[HTML]{333333}40.22,***, L}         &   & \cellcolor[HTML]{C6E0B4}{\color[HTML]{333333}42.25,***, L}               \\ \midrule
% $\$Q^3$                 & $\$P^3$                & & \cellcolor[HTML]{C6E0B4}{\color[HTML]{333333}4.61,***, S}            &  & 3.36, \textit{n.s.}             \\ 
%                          & $R3D$                    &   & \cellcolor[HTML]{C6E0B4}{\color[HTML]{333333}14.86,***, L}         &   & \cellcolor[HTML]{C6E0B4}{\color[HTML]{333333}15.06,***, S}          \\ 
%                          & $RS$                 &    & \cellcolor[HTML]{C6E0B4}{\color[HTML]{333333}36.61,***, L}       &     & \cellcolor[HTML]{C6E0B4}{\color[HTML]{333333}39.73,***, L}           \\ \midrule
% $\$P^3$                 & $R3D$                    &   & \cellcolor[HTML]{C6E0B4}{\color[HTML]{333333}10.24,***, L}        &   & \cellcolor[HTML]{C6E0B4}{\color[HTML]{333333}11.69,***, S}              \\ 
%                          & $RS$                &    & \cellcolor[HTML]{C6E0B4}{\color[HTML]{333333}34.01,***, L}         &   & \cellcolor[HTML]{C6E0B4}{\color[HTML]{333333}36.37,***, M}          \\ \midrule
% $R3D$                        & $RS$                   &   & \cellcolor[HTML]{C6E0B4}{\color[HTML]{333333}23.75,***, L}         &   & \cellcolor[HTML]{C6E0B4}{\color[HTML]{333333}24.67,***, S}           \\ \bottomrule
% \end{tabular}
%   \caption{ANOVAs computed for the 7 \textsc{Recognizers} in the user-independent (UI) and user-dependent  (UD) scenarios: G1=group 1, G2=group 2. For each dataset, three data are provided: the $q$ value resulting from the ANOVA, the significance of the \textit{p} value if any (***$p{\leq}.001$), and Cohen's \textit{d} coefficient for effect size ((S)mall when $d{\geq}.02$, (M)edium when $d{\geq}.05$, (L)arge when $d{\geq}.08$,  and (--) when no significant effect size ($d{<}.02$)).}
%   \label{tab:anova_ui_ud}
% \end{table}






 \newpage



%The above conclusions are mainly exploiting the two quantitative measures,\textit{\ie,}, the recognition rate and the execution time. Other measures could be considered for assessing the overall performance of a recognizer, such as computational and statistical efficiency analyses. Computational or algorithmic efficiency  measures the amount of time or memory required for a given recognizer to perform all calculations for all steps, such as an evaluation of a log posterior or penalized likelihood. This efficiency is often measured in terms of order according to Landau's notation
%: $\mathcal{O}(log{}n)$ for logarithmic order, $\mathcal{O}(n)$ for linear order, $\mathcal{O}(n log{}n)$ for linearithmic order, and so forth.

%Statistical efficiency typically involves optimizing steps in the recognition algorithm by statistically formulating a better model --but this is very challenging-- or by changing parameters (\textit{\ie,}, by reparameterization) so that sampling algorithms mix better.

%But all in all, we are more interested to maximize the recognition rate while minimizing the execution time, as most devices and platforms today offer acceptable computational resources to run these algorithms, especially in regard to machine-learning algorithms such as in learning-time optimization~\cite{Saito:2015}.

%Therefore, we define a new performance measure, the \textit{ratio \enquote{rate/time}}, which is hereby defined as the recognition rate of a recognizer divided by its execution time. To compute this measure, we want to know the  time consumed for comparing a candidate with one template during classification. For this purpose, we remove the influential factors which differ depending on the recognizer and affect the execution time. We saw that  Rubine3D and Rubine-Sheng are influenced by the $C$, the number of gesture classes in a dataset.  $\$P^3$, $\$Q^3$, $\$P+^3$, $\$F$ and FreeHandUni are influenced by $N$, the number of sampling points, and by $T$, the number of templates for each class. In order to smooth these results, the ratio is normalized into a $[0,...,1]$ range.
%We present four charts of this ratio \enquote{rate/time} applied to all the recognizers for SHREC2019 and 3DTCGS for the two minimal conditions $T{=}8$ (left parts of Figure~\ref{fig:Rate_Time_Shrec2019} and Figure~\ref{fig:Rate_Time_Sdtcgs}) and $T{=}16$ (right parts of Figure~\ref{fig:Rate_Time_Shrec2019} and Figure~\ref{fig:Rate_Time_Sdtcgs}).

%We notice that the charts are quite similar in Figure~\ref{fig:Rate_Time_Shrec2019}. For a small number of points ($N{\leq}8$), the $\$P+^3$ has the greatest ratio for the two conditions $T{=}8$ and $T{=}16$  ($q{=}0.549$ and $q{=}0.665$) followed by the two recognizers $\$F$ and FH. Then, the ratio of $\$P+^3$ decreases to intersect with $\$F$ at $N{=}8$, the tendency of curves shows that the \$-recognizers' ratios decrease to approach zero.

%Compared to the SHREC2019 results, the curves in Figure~\ref{fig:Rate_Time_Sdtcgs} show that the recognizers perform better for the 3DTCGS. In the same ways the  $\$P+^3$ has the best ratio then it drops like the \$-recognizers' ratios  to near zero.
%We can see that Rubine3D and Rubine-Sheng keep the same ratio for the two datasets without being impacted by the number of points, even if their ratios are low, from $N{=}16$  their curves are above the \$-recognizers' curves.
%From these results, the computed ratio indicates that the recognizers perform identically for the SHREC2019 and 3DTCGS datasets.

% \begin{figure}[h]
%  \centering
% %    \includegraphics[width=1\textwidth,trim=0 0 -90 0cm]{Images/Rate_Time_Charts/Rate_Time_SHREC2019.pdf}
% \includegraphics[width=\textwidth]{Images/Rate_Time_Charts/Rate_Time_SHREC2019.pdf}
%     \vspace{-18pt}
%     \caption{Normalized rate/time curve for the SHREC2019 dataset (left: $T{=}8$, right: $T{=}16$).}
%     \label{fig:Rate_Time_Shrec2019}
% \end{figure}

% \begin{figure}[h]
%  \centering
%     \includegraphics[width=\textwidth]{Images/Rate_Time_Charts/Rate_Time_3DTCGS.pdf}
%     \vspace{-18pt}
%     \caption{Normalized rate/time curve for the 3DTCGS dataset (left: $T{=}8$, right: $T{=}16$). }
%     \label{fig:Rate_Time_Sdtcgs}
% \end{figure}


% \begin{figure*}[t]
%     \centering
% %    \vspace{-24pt}
%     \includegraphics[width=\textwidth]{Figures/Chap4/Multipath_Presentation1.pdf}
%     \vspace{-18pt}
%     \caption{A comparison of percentage in recognition rate for two 3D recognizers  \textsf{\$P\textsuperscript{3}+X} and  $\$P^3+$ on the SHREC2019 dataset~\cite{Caputo:2019}, for some number of joints A, and a number of points N=8, in a user independent scenario.}
%     \label{fig:Rate_Multipaths}
%     \vspace{-10pt}
% \end{figure*}

%\newpage
\begin{figure}[h]
    \centering
    \includegraphics[width=.9\textwidth]{Figures/Chap4/Chap4_dollarFamily_Instantiation.pdf}
    \vspace{-8pt}
    \caption{The \$-like recognizers instantiation.}
    \vspace{-8pt}
   \label{fig:design-dollar}
\end{figure}
\begin{figure}[t]
    \centering
    \includegraphics[width=.9\textwidth]{Figures/Chap4/Chap4_Rubine_Instantiation.pdf}
    \vspace{-8pt}
    \caption{The Rubine3D and Rubine-Sheng instantiation.}
    \vspace{-8pt}
   \label{fig:design-rubine}
\end{figure}

\section{Conclusion}
This chapter focuses on the comparative testing of template-based hand gesture recognizers on \enquote{unipath dynamic} gestures, specifically on 3D trajectories. These gestures involve single-point movements performed in the air. The goal is to instantiate the comparative testing systematic procedure to evaluate the recognition rate and execution time of recognizing these 3D trajectories using different gesture recognizers in different scenarios. The scenarios include an \enquote{intra-device} scenario, where the gestures are recorded with the same device for different stages of the gesture recognition process. Additionally, there are user-independent and user-dependent scenarios. 

To achieve the goal, four 2D stroke gesture recognizers ($\$P$, $\$P+$, $\$Q$, and Rubine) selected from the literature review are extended to the third dimension. The extended recognizers are: $\$P+^3$, $\$F$, FreeHandUni, $\$Q^3$, $\$P^3$, Rubine3D, and Rubine-Sheng.

Significant differences were observed among the recognizers by comparing individual recognition rates and execution times on different datasets, as well as aggregated results across all datasets. $\$P+^3$ is the top performer among the tested 3D trajectory recognizers, consistently achieving the highest recognition rate in almost all conditions. This statistical significance is only diminished in a few cases, such as when the number of templates is very limited. In addition to its robustness in user-independent scenarios, $\$P+^3$ also boasts excellent execution time in almost all scenarios. As a result, it is the preferred choice for efficiently recognizing 3D trajectories, particularly when compared to alternative recognizers.
After the $\$P+^3$ recognizer, there is a recurrent pattern in the ordering of subsequent recognizers both in terms of recognition rate and execution time: a first batch of other \$-like recognizers appears with $\$F$, $FH$, $\$Q^3$, and $\$P^3$ and a second batch containing the two feature-oriented algorithms, \textit{\ie}, $R3D$ and $RS$. We notice that the recognizers share two common instantiations. The first instantiation, represented in Figure~\ref{fig:design-rubine}, is shared by the Rubine recognizers (Rubine3D and Rubine-Sheng) and uses the \enquote{Feature vector} classification approach. The second instantiation, represented in Figure~\ref{fig:design-dollar}, shows the values covered by the other \$-like recognizers ($\$F$, FH, $\$Q^3$, $\$P^3$, and $\$P+^3$) and uses the \enquote{Between-points} classification approach. 
%\chapter[Use Case 2: Comparative Testing on Multipath Dynamic Gestures]{Use Case 2: Comparative Testing of Recognizers on Multipath Dynamic Gestures}\label{chap:LMCmultipathComparative}

\vspace{-12pt}
The work discussed in this chapter was originally published in \textit{HCI International 2022 - Late Breaking Papers. Multimodality in Advanced Interaction Environments.} Published in 2022 by Cham: Springer Nature Switzerland AG \cite{Ousmer:2022}. It is reproduced with permission from Springer Nature.

In the comparative testing of the \enquote{unipath dynamic} gestures, we observed the effectiveness of using comparative testing to evaluate recognizers in a \textit{uni-device configuration}. The gesture sets used were created from different input device categories, including \enquote{Leap Motion controller} for SHREC2019 and 3DTCGS, and \enquote{SoftKinetic DepthSense DS325} for the 3DMadLabSD. This highlights the capabilities of the comparative testing method for template-based hand gesture recognizers. We further investigate this method by extending the exploration of the design space for hand gesture recognizer adapted to rapid prototyping (Section~\ref{sec:Design_Space_Recognizers}). 

In this chapter, we will instantiate the procedure for comparative testing of a set of template-based gesture recognizers on \enquote{multipath dynamic} hand gestures recorded with a Leap Motion controller grouped into two gesture sets. We selected two recognizers from the set of recognizers we evaluated in Chapter~\ref{chap:Unipath_Comparative} on \enquote{unipath dynamic} gestures, $\$P^3+$ (The same as in Section~\ref{sec:dollarpplus}), $\$F$ (Section~\ref{sec:dollarf}), we also selected two recognizers from the targeted literature review Jackknife~\cite{Taranta:2017} and $3~Cent$~\cite{Caputo:2017}. Jackknife has been also successfully used for dynamic gestures captured by a radar~\cite{Sluyters:2022:IUI,Sluyters:2023}, thus making it an eligible candidate for challenging gestures. In addition to two new state-of-the-art recognizers: $\$P^3+$X and PennyPincher3D, a recognizer with a \enquote{Between-Vectors} classification approach. 

To ensure consistent comparison of gesture recognizers under identical conditions (\textit{\ie}, parameter values), comparative testing should examine the impact of various factors on performance. These factors include the number of templates, the number of sampling points, the number of fingers, and their configuration with other hand parameters such as hand joints, palm, and fingertips. The procedure's results identify the configurations in which each recognizer is most accurate or fastest. Specifically, we focus on identifying the best gesture recognizers that have both high accuracy and low response time. We then define a method to determine the optimal conditions for designating these recognizers.
\vspace{-5pt}
\section{Recognizers}
\vspace{-5pt}
In this comparative evaluation procedure, we used template-based gesture recognizers, among them, 2 recognizers used in the comparative tests performed in Chapter~\ref{chap:Unipath_Comparative}: $\$P^3+$ ($\$P+^3$ adapted to multipath dynamic gestures), $\$F$, We added two recognizers from the targeted literature review in Chapter~\ref{chap:Related}, Jackknife~\cite{Taranta:2017} and $3~Cent$~\cite{Caputo:2017}. Finally, we added two new state-of-the-art recognizers: \$P\textsuperscript{3}+X and PennyPincher3D.
\vspace{-5pt}
\subsection{ \texorpdfstring{\$P\textsuperscript{3}+X}{\$P3+X} Recognizer}
A variant of \textit{\$P\textsuperscript{3}+}~\cite{Vatavu:2017} that takes into account the direction-invariance by tracking conflicting templates (\textit{\ie}, templates of the same gesture but performed in different directions). If a gesture matches with a conflicting template, its direction is compared with the direction of each conflicting template, and the nearest one is chosen.
\vspace{-5pt}
\subsection{PennyPincher3D Recognizer}  
\textit{PennyPincher3D} is an adaptation of the 2D recognizer \textit{PennyPincher} ~\cite{Taranta:2015}. The gestures are represented as a set of $N-1$ vectors linking between $N$ equidistant points. The recognizer matches the candidate gesture with the template that maximizes a dissimilarity score, computed as the sum of the angles between the vectors. The computation relies on basic mathematical operations such as additions and multiplications. The gestures require just a resampling as prepossessing. This recognizer is scale- and position-invariant as most of the \$-recognizers.
\section{Datasets} \label{sec:MulComp_Datasets}
\subsection{SHREC2019}
The \textbf{SHREC2019} dataset~\cite{Caputo:2019} includes a sequence of 3D points and a sequence of quaternions for each hand joint. However, we ignored the quaternions in our experiments. The provided gestures represent one of five different gesture classes (Figure~\ref{fig:SHREC2019-Gestures}): \enquote{Cross}(X), \enquote{Circle}(O), \enquote{V-mark}(V), \enquote{Caret}(/\textbackslash), and \enquote{Square}([])). It served in (SHREC) track, a contest on online gesture recognition to detect command gestures from hands' movements in a virtual reality context. The proposed dataset consists of 195 3D  movements performed by 13 participants with the whole hand. The dataset contains unsegmented gestures. The training set and the testing set were merged to create a unique dataset in which, unnecessary hand movements were removed from the gestures.

\label{fig:SHREC2019_datasets}
\begin{figure}[ht]
    \vspace{-10pt}
	\centering
	\captionsetup{justification=centering}
	\includegraphics[width=1\linewidth]{Figures/Chap5/Datasets/SHREC2019.pdf}
    \vspace{-10pt}
	\caption{The SHREC2019 gesture classes and samples.~\cite{Caputo:2019}}
	\label{fig:SHREC2019-Gestures}
\end{figure}
\vspace{-12pt}
\subsection{Jackknife-LM} 

The \textbf{Jackknife-LM (Jackknife-LeapMotion)} dataset ~\cite{Taranta:2017}  contains 3D complex gestures of the hand and saved as 3D skeleton which is provided by the Leap Motion controller device.
We used the segmented gestures composed of 360 samples of 9 different gesture classes, for example, \enquote{Fist Circles}, \enquote{Snip Snip}, \enquote{Explode}(Figure~\ref{fig:Jackknife-Gestures}). It was used to test a rejection method of non-gesture sequences from a continuous data stream. While segmented gestures make up the training set, authors employ unsegmented sessions of samples in the test~\cite{Taranta:2017}.


\vspace{-12pt}
\begin{figure}[h!]
	\centering
	\captionsetup{justification=centering}
	\includegraphics[width=0.85\linewidth]{Figures/Chap5/Datasets/JackknifeLM.pdf}
	\caption{The Jackknife-LM gesture classes.~\cite{Taranta:2017}}
	\label{fig:Jackknife-Gestures}
\end{figure}
\vspace{-2pt}
\section{Design}
\label{sec:Multipath_Comparative_evaluation}
The comparative testing performed aims at filling this gap in the literature. As such, we selected two available gesture sets for a number of reasons, including reproducibility. We evaluated the described recognizers following the typical method used in the literature to evaluate gesture recognizers ~\cite{Anthony:2010,Anthony:2012,Ousmer:2020,Vatavu:2012b,Vatavu:2018,Wobbrock:2007} in a user-independent scenario. We tested them on full hand gestures provided by the LMC skeletal hand model (Figure~\ref{fig:LMC-HandModel}) to show the efficiency of these recognizers under different conditions.

\vspace{-8pt}
\begin{figure}[!ht]
\captionsetup{justification=centering}
\vspace{-12pt}
\subfigure[LMC~hand~model.~\cite{LeapMotionBlog:2014}]{\label{fig:LMC-HandModel}\includegraphics[width=.50\textwidth,trim= 0 -30 0 -20, clip]{Figures/Chap5/LMC_Hand_hierarchy.pdf}}
\subfigure[The hand joints selected for the testing.]{\label{fig:LMC-joints}\includegraphics[width=0.75\textwidth,trim= 0 0 0 0, clip]{Figures/Chap5/LMC_Hand_Joints.pdf}}
  \centering
  \caption{Overview of the LMC Hand model and joints.}
   \vspace{-15pt}
   \label{LMC Hand Joints}
\end{figure}

\subsection{Experiment}
Our evaluation was within factors with five independent variables:
\begin{enumerate}
    \item \textsc{Recognizer}: nominal variable with 6 conditions, representing the various recognizers implemented for recognizing 3D gestures $\$P^3+$, $\$F$, Jackknife~\cite{Taranta:2017} and $3~Cent$~\cite{Caputo:2017}. And the two new recognizers which are  described above:  $\$P^3+X$ and \textit{PennyPincher3D}.
    \item \textsc{Dataset}: nominal variable with 2 condition, representing the datasets considered, \textit{\ie}, SHREC2019~\cite{Caputo:2019} and Jackknife-LM~\cite{Taranta:2017} described in Section~\ref{sec:MulComp_Datasets}.
    \item \textsc{Joints}: Nominal variable with 8 conditions, representing the hand joints used. Since the fingers contain several joints, we decided to use the information provided by the tips of the fingers (Figure ~\ref{fig:LMC-joints}): \textbf{\enquote{1(P)}} = \{Palm\}, \textbf{\enquote{2(P+I)}} = \{Palm, Index\},  \textbf{\enquote{2(T+I)}}=\{Thumb, Index\}, \textbf{\enquote{2(I+M)}} = \{Index, Middle\} , \textbf{\enquote{3(P+I+M)}} =\{Palm, Index, Middle\} , \textbf{\enquote{3(T+I+M)}} =\{Thumb, Index, Middle\}, \textbf{\enquote{5(AF)}} = \{Thumb, Index, Middle, Ring, Pinky\},  \textbf{\enquote{6(P+AF)}} = \{Palm, Thumb, Index, Middle, Ring, Pinky\}.
    \item \textsc{Number of Templates}: numerical variable with 5 conditions, representing the number of templates per gesture for training: $T{=}\{1,2,4,8,16\}$.
    \item \textsc{Sampling}: numerical variable with 5 values representing the number of points per gesture: $N{=}\{4,8,16,32,64\}$.
\end{enumerate}
\vspace{-12pt}
\subsubsection{Apparatus} We used a hexa-core Intel Core i7 2.20 GHz CPU and a Windows 10 Home Edition operating system. The RAM was 16 GB DDR4  memory with 2400 MHz.\vspace{-12pt}

\subsection{Procedure and Quantitative Measures}
We compute the \textit{recognition rate} (computed as the ratio of positive recognitions divided by the total number of trials) for the 6 (\textsc{Recognizer}) $\times$ 2 (\textsc{Dataset}) = 12 basic configurations in the \textit{user-independent scenario}. This scenario evaluates the recognition of gestures produced by users who are different from those used for training the recognizer. In this scenario, the basic configurations are refined depending on $A$, the number of joints, on  $T$, the number of templates, and depending on $N$, the number of resampling points to train the recognizer.
For each gesture class, a template is randomly selected from all participants and saved for testing. Then, a training set is created by randomly choosing $T$ templates for each gesture class from the remaining users. These templates should be different from the ones previously selected for testing. The recognizer is then trained using this training set. This task is repeated $R{=}$100 times for each template in the $T$ set.
\section{Results}
\label{sec:results}
 Overall, we conducted 2 (\textsc{Dataset}) $\times$ 8 (\textsc{Joints}) $\times$ 5 (\textsc{Sampling}) $\times$ 5 (\textsc{Number of Templates}) $\times$ 100 (repetitions) $\times$ 6 (\textsc{Recognizer}) = 240,000 recognition trials for each dataset. 
Finally, the number of recognized gestures is averaged to calculate the recognition rate, which is then formatted as a percentage. Statistical computations are performed using GraphPad Prism.
%\vspace{-0.25cm}

\begin{figure}[ht!]
    \vspace{-5pt}
	\captionsetup{justification=centering}
\hspace{0cm}
	\includegraphics[width=\linewidth]{Figures/Chap5/SHREC2019/Rate_AllCond_Overall_ByGesture.pdf}
	\vspace{-14pt}
	\caption{Recognition rates of all recognizers for the SHREC2019 for all conditions. Error bars show a confidence interval of $95\%$.}
	\label{fig:SHREC2019-Overall}
	\vspace{-12pt}
\end{figure}
\vspace{-12pt}
\subsection{SHREC2019 Dataset}
\subsubsection{\textbf{Overall Recognition Rate}}\label{sub:All-Cond}
Figure~\ref{fig:SHREC2019-Overall} displays the average recognition rate for all tests under all conditions, with the recognizers listed in descending order. The $\$P^3+$ has the highest average recognition rate ($M{=}85.90\%$, $SD{=}15.44\%$), followed by $\$P^3+X$ and Jackknife with ($M{=}84.90\%$, $SD{=}16.45\%$) and ($M{=}82.11\%$, $SD{=}14.06\%$), respectively. The average recognition rate of \textit{PennyPincher3D} is ($M{=}80.02\%$, $SD{=}12.38\%$). For the last two recognizers, the average rates do not exceed $80\%$: $3~Cent$ has an average recognition rate of ($M{=}78.09\%$, $SD{=}18.12\%$) and $\$F$ has an average recognition rate of ($M{=}77.36\%$, $SD{=}19.27\%$).


We performed four normality tests on the \textsc{Recognizer} variable: the Anderson-Darling K-sample test, D'Agostino's K2 test and Kolmogorov-Smirnov's KS. None of the recognition rates followed a normal distribution. Due to the large number of samples, the Shapiro-Wilk test could not be used to assess normality. Then, we performed a Kruskal-Wallis test with Dunn's multiple comparisons on the measures. The overall difference between the recognizers is very highly significant ($p{<}.001^{***}$). Figure~\ref{fig:SHREC2019-Overall} shows that $\$P^3+$ is the most accurate Recognizer, significantly outperforming all other recognizers. $\$P^3+$ is better than $\$F$ with a highly significant difference ($Z{=}48.84,p{<}.001^{***}$) and significantly better than Jackknife ($Z{=}27.86$, $p{<}.001^{***}$). Also, $\$P^3+$ significantly outperforms $\$P^3+X$ ($Z{=}5.247,p{<}.001^{***}$). However, $\$F$ is not significantly different from $3~Cent$ ($Z{=}1.928,{n.s.}$), while this later is significantly outperformed by \textit{PennnyPincher3D} ($Z{=}3.055,p{<}.05^{*}$).


Furthermore, the overall recognition rate per gesture class indicates that:
\begin{itemize}
    \item The gestures \enquote{Caret}(/\textbackslash) and \enquote{V-mark}(V) have average rates greater than or equal to $90\%$. However, similarly to the average recognition rates of the other classes, these rates vary among the different recognizers.
    \item  The \enquote{Cross} (X) gesture is better recognized than the \enquote{Square} ([]) and the \enquote{Circle} (O) by the $\$F$, $\$P^3+$ and $\$P^3+X$ recognizers.
    \item The best average recognition rate for \enquote{Cross} (X) gesture class is achieved by Jackknife ($M{=}93. 98\%$); this value goes down respectively to ($M{=}83.35\%$) for $\$P^3+$ and to ($M{=}67.75\%$) for $\$F$ with a difference of $10.53\%$ and $26.45\%$.     
    \item For the Jackknife and \textit{PennyPincher3D} recognizers,  the \enquote{Square}([])) and \enquote{Cross}(X) classes  have similar average recognition rates. 
    \item On the contrary, the gesture class \enquote{Circle}(O) has a very low average recognition rate for Jackknife ($M{=}34.43\%$), \textit{PennyPincher3D} ($M{=}21.91\%$) and $3~Cent$ ($M{=}46.50\%$). 
\end{itemize} 


From the detailed recognition rate for each condition, some recognizers have low or almost non-existent recognition rates that meet the expectation of end users ($\tau{\geq}90\%$)~\cite{Marin:2016,Wang:2015}. These include \textit{PennyPincher3D}, $3~Cent$, and $\$F$. On the other hand, $\$P^3+$ and $\$P^3+X$ consistently achieve recognition rates of $\tau{\geq}90\%$. Since the goal is to conduct a comparative test of recognizers using the SHREC2019 dataset, we relaxed the aforementioned constraint and refined the results to require a recognition rate of $\tau{\geq}80\%$.



\subsubsection{\textbf{Recognition Rates by Number of Joints}}
Figure~\ref{fig:plot-Recog-by-Joint-SHREC2019} illustrates the average recognition rates for each value of condition \textsc{Joints} (A). Each recognition rate represents the average of the recognition rates of all conditions (T) and (N). The values range from $M{=}75.58\%$ for $\$F$ in \textsc{$A=2\:(I + M)$} to $M{=}87.04\%$ for $\$P^3+$ in \textsc{$A= 3\:(P + I +T)$}. Generally, the recognition rates do not vary significantly across different values of (A). $\$P^3+$ achieves the highest average recognition rate for \textsc{$A=3\ :(P + I + T)$} ($M{=}87.04\%$) while $\$F$ has the lowest average rate for \textsc{$A=2\ :(M + I)$} ($M{=}75.58\%$). 

\begin{figure}[ht]
	\centering
	\captionsetup{justification=centering}
	\includegraphics[width=\linewidth]{Figures/Chap5/SHREC2019/Rate_Avg_per_Joints.pdf}
	\vspace{-18pt}
	\caption{Recognition rates of all recognizers for the SHREC2019 dataset~\cite{Caputo:2019}, the plot shows average rates by joint $A=\{1\:(P), ... ,6\:(P + AF)\}$.}
	\label{fig:plot-Recog-by-Joint-SHREC2019}
	\vspace{-12pt}
\end{figure}

Although, there are some conditions such as \textsc{$A=2\:(I + M)$} and \textsc{$A=3\:(P + I + T)$}, where the average rate of $\$P^3+X$ exceeds that of $\$P^3+$ (\eg   \textsc{$A=2\:( I  + M)$} and  \textsc{$A= 3\:(P + I +M)$}). Similarly, for some values of (A), $3~Cent$ has lower recognition rates than $\$F$ (\eg  \textsc{$A=1\:(P)$} and \textsc{$A=2\:(T + I)$}).


As a way to help understand the behavior of the recognizer with respect to both the number of points as well as the number of joints. Figure~\ref{fig:SHREC2019-rank-Overall} is an overview of recognizers ranked by recognition rate. This ranking is presented along two axes, the number of joints (A) and the number of sampling points (N). It shows which recognizers are better than the others for each pair of conditions $(A, N)$. To respect the restriction on the recognition rate that we have defined above, all recognizers in this figure comply with two criteria:
 \begin{enumerate}
     \item The recognizer must achieve a recognition rate $>80\%$. 
     \item The recognition rate must be the highest value among the conditions of (T).
 \end{enumerate}
 Globally, the position of the recognizers varies a lot according to the defined conditions, and few recognizers keep a constant ranking by varying the number of joints or the number of points (\eg, \textit{PennyPincher3D} for $N{=}16$ and $N{=}32$). For $N{=}4$, we observe that some recognizers do not appear, because of the first criterion defined, except for $3~Cent$, which has at least a recognition rate that respects the criterion. We notice that for several conditions, the $\$P^3+$ recognizer takes the first place, except for the condition where \textsc{$A=2 \:(P + I)$} and \textsc{$A=2 \:(I + M)$} where $\$P^3+X$ is designated as the best recognizer. In contrast to $\$P^3+$, the \textit{PennyPincher3D} often achieves the lowest recognition rate and is ranked last.
\vspace{-0.25cm}


\clearpage

\begin{figure}[ht!]
\hspace{0.5cm}
\centering
    \vspace{-0.5cm}
	\captionsetup{justification=centering}
	\hspace{-0.5cm}
	\includegraphics[width=\linewidth]{Figures/Chap5/SHREC2019/Rate_Ranking_Sup80_A_N_Cond.pdf}
	%\vspace{-0.75cm}
	\caption{Ranking of the best individual recognition rates above 80\% by the number of joints (A) and the number of points (N) for the SHREC2019 dataset.}
	\label{fig:SHREC2019-rank-Overall}
\end{figure}

\subsubsection{\textbf{Recognition Rate for the Optimal Conditions}}
To evaluate the efficiency of the recognizers, we planned to test them in a precise scenario where the conditions are well-defined. For this reason, we determine which values of the variables \textsc{Joints} (A), \textsc{Sampling} (N), and \textsc{Number of templates} (T) allow an optimal comparison of the different recognizers in the recognition rate measure. We completed a de Borda ranking of all combinations of conditions for each recognizer, and the result is reported in Table~\ref{tab:SHREC2019-Borda}. The de Borda ranking selects the best combination of conditions, the one with the highest overall Borda score. This implies that the elected combination of values has a high recognition rate with many recognizers. Furthermore, we notice that many conditions share the same rank and score in the ranking for each recognizer. This situation is due to a defined sensitivity value of $2\%$ in the recognition rate. We determine the case of a tie between the last ranked combination of conditions and the following combination to be ranked on a specific recognizer, if the difference between their recognition rates is less than the defined sensitivity value. According to Table~\ref{tab:SHREC2019-Borda}, the result of the Borda ranking gives this combination of conditions (\textsc{$A=2\:(T + I)$}/ $N{=}32$/ $T{=}16$) as the best.
\begin{table}[ht]
\centering
\captionsetup{justification=centering}
\caption{Global position of the recognizers for each condition and overall, according to de Borda method across all recognizers (the higher, the better).}
\vspace{1pt}
\resizebox{1.25\columnwidth}{!}{%
\hspace{10px}

\begin{tabular}{|l|c|c|c|cccccc|llllllllll}
\cline{1-10}
\multicolumn{3}{c}{\textbf{Condition}} & \multicolumn{1}{l}{\textbf{}} & \multicolumn{6}{c}{\textbf{Recognizer   (\#Rank, Score)}} & \multicolumn{2}{c}{\textbf{}} &  &  & \multicolumn{2}{c}{\textbf{}} &  &  & \multicolumn{2}{c}{\textbf{}} \\  \cline{1-3} \cline{5-10}
\multicolumn{1}{|c|}{A} & N & T & \textbf{Overall} & $\$P^3+$  & $\$F$ & \textit{PP3D} & Jackknife & $\$P^3+X$  & $3~Cent$ &  &  &  &  &  &  &  &  &  &  \\  \cline{1-3} \cline{5-10}
2 (T + I) & 32 & 16 & \cellcolor[HTML]{bebee6}{\#1, 1187} & \#1, 200 & \#1, 200 & \#2, 196 & \#1, 200 & \#1, 200 & \#2, 191 &  & \multicolumn{8}{l}{} &  \\
3 (P + I +M) & 32 & 16 & \cellcolor[HTML]{bebee6}{\#2, 1180} & \#1, 200 & \#3, 184 & \#2, 196 & \#1, 200 & \#1, 200 & \#1, 200 &  &  &  &  &  &  &  &  &  &  \\
2 (P + I) & 32 & 16 & \cellcolor[HTML]{bebee6}{\#3, 1177} & \#1, 200 & \#2, 197 & \#2, 196 & \#2, 193 & \#1, 200 & \#2, 191 &  &  &  &  &  &  &  &  &  &  \\
3 (P + I +M) & 16 & 16 & \cellcolor[HTML]{bebee6}{\#4, 1175} & \#1, 200 & \#3, 184 & \#1, 200 & \#1, 200 & \#1, 200 & \#2, 191 &  &  &  &  &  &  &  &  &  &  \\
2 (I + M) & 32 & 16 & \cellcolor[HTML]{bebee6}{\#5, 1166} & \#2, 186 & \#1, 200 & \#2, 196 & \#2, 193 & \#1, 200 & \#2, 191 &  &  &  &  &  &  &  &  &  &  \\
1 (P) & 64 & 16 & \cellcolor[HTML]{bebee6}{\#6, 1161} & \#2, 186 & \#2, 197 & \#3, 185 & \#2, 193 & \#1, 200 & \#1, 200 &  &  &  &  &  &  &  &  &  &  \\
2 (T + I) & 16 & 16 & \cellcolor[HTML]{bebee6}{\#7, 1158} & \#1, 200 & \#1, 200 & \#3, 185 & \#1, 200 & \#2, 182 & \#2, 191 &  &  &  &  &  &  &  &  &  &  \\
5 (AF) & 32 & 16 & \cellcolor[HTML]{bebee6}{\#8, 1157} & \#1, 200 & \#2, 197 & \#2, 196 & \#2, 193 & \#1,200 & \#3, 171 &  &  &  &  &  &  &  &  &  &  \\
6 (P + AF) & 32 & 16 & \cellcolor[HTML]{bebee6}{\#9, 1153} & \#1, 200 & \#3, 184 & \#3, 185 & \#2, 193 & \#1, 200 & \#2, 191 &  &  &  &  &  &  &  &  &  &  \\
3 (P + I + T) & 32 & 16 & \cellcolor[HTML]{bebee6}{\#10, 1148} & \#1, 200 & \#2, 197 & \#3, 185 & \#2, 193 & \#2, 182 & \#2, 191 &  &  &  &  &  &  &  &  &  &  \\
... & ... & ... & \cellcolor[HTML]{bebee6}{...} & ... & ... & ... & ... & ... & ... \\  \cline{1-3} \cline{5-10}
\end{tabular}
}
\label{tab:SHREC2019-Borda}

\end{table}


Based on de Borda's results, we evaluate the recognizers on the SHREC2019 dataset under the elected conditions. The upper part (a) of Figure~\ref{fig:SHREC2019-best-cond}  shows the average recognition rate for the defined conditions. According to these results, the $\$P^3+$ has the best average recognition rate ($M{=}97.00\%$, $SD{=}7.18\%$), followed by the $\$P^3+X$ ($M{=}95.20\%$, $SD{=}8.59\%$) and the $\$F$ ($M{=}94.20\%$, $SD{=}9. 55\%$), while \textit{PennyPincher3D} is the least accurate recognizer ($M{=}87.80\%$, $SD{=}13.30\%$), for the other two recognizers, the average rate of Jackknife ($M{=}93.2\%$, $SD{=}10.34\%$) and $3~Cent$ ($M{=}90.80\%, SD{=}13.16\%$). 


As for the overall recognition rate on all conditions in part~\ref{sub:All-Cond}, we calculated the four normality tests: K sample's Anderson-Darling test, Shapiro-Wilk W test, D'Agostino's K2 test and Kolmogorov-Smirnov's KS.

None of the recognition rates followed a normal distribution. Then, we calculated a Kruskal-Wallis test with Dunn's multiple comparisons. The high difference between the recognizer $\$P^3+$  and  \textit{PennyPincher3D} ($10\%$)  is statistically highly significant ($Z{=}5.736$, $p{<}.001^{***}$). Same thing for $\$P^3+X$ that is significantly better than \textit{PennyPincher3D}  ($Z{=}4.420$, $p{<}.001^{***}$). Moreover,  $\$P^3+$ is significantly more accurate than $3~Cent$ by $6.2\%$ ($Z{=}3.699$, $p{<}.01^{**}$). $\$F$ is superior by  $6.4\%$ to \textit{PennyPincher3D} with a highly significant difference  ($Z{=}3.788$, $p{<}.01^{**}$). However, $\$P^3+$ is not significantly better than Jackknife ($Z{=}2.581$, \textit{n.s.}), while this later is significantly better than \textit{PennnyPincher3D} ($Z{=}3.156$, $p{<}.05^{*}$).


\begin{figure}[ht]
	\centering
	\vspace{-16pt}
	\captionsetup{justification=centering}.
%    \hspace{-1pt}
	\includegraphics[width=0.95\linewidth]{Figures/Chap5/SHREC2019/Rate_BestCond_Overall_ByGesture.pdf}
	\vspace{-6pt}
	\caption{Recognition rate  (a) and the number of recognized gestures per class over 100 trials (b) of all recognizers for the user-independent scenario,  for the optimal conditions defined by the de Borda ranking: \textsc{$A{=}2\:(T + I)$}, $N{=}32$, $T{=}16$. Error bars show a confidence interval of $95\%$.}
	\label{fig:SHREC2019-best-cond}
\end{figure}

In the bottom part of the same figure (Figure~\ref{fig:SHREC2019-best-cond}(b)), the bar chart shows the number of recognized gestures per gesture class on 100 repetitions, noting that all recognizers achieve a perfect score for at least one gesture class. Among the gesture classes in question, there is  the \enquote{Caret}(/\textbackslash) for \textit{PennyPincher3D} and $3~Cent$ recognizers, and also the \enquote{V-mark}(V) gesture class for $\$P^3+X$, $\$F$ recognizers. While $\$P^3+$ achieves an accuracy rate of ($\frac{100}{100}$) for two gesture classes  :  the\enquote{V-mark}(V)  and the \enquote{Cross}(X). Although Jackknife achieves a flawless recognition for three gesture classes: \enquote{Caret}(/\textbackslash) , \enquote{Cross}(X)  and  \enquote{V-mark}(V), its recognition of the \enquote{Circle}(O)  gesture class is weak  ($\frac{75}{100}$). This difficulty is also encountered by \textit{PennyPincher3D}  for the same gesture class ($\frac{59}{100}$) as for $3~Cent$.  However, the latter achieves the best accuracy rate for the \enquote{Square}([]) gesture class ($\frac{99}{100}$) which means that \textit{PennyPincher3D} is well designed to recognize this gesture class, especially when the other recognizers perform less well in recognizing it.

\subsubsection{\textbf{Overall Conditions Execution Time}}
 In Figure~\ref{fig:SHREC2019-Avg-ExecTime}, the average execution times of the different recognizers for all conditions varied between ($M{=}0.047$ ms, $SD{=}0.060$ ms) for the \textit{PennyPincher3D} and ($M{=}1.611$ ms, $SD{=}3.328$ ms) for the $\$P^3+$. The execution times do not follow a normal distribution. We calculated a Kruskal-Wallis test which indicates a significant difference in the execution times of the recognizers. After that, the Dunn's multiple comparisons test shows significant differences for all pairs of recognizers except between $3~Cent$ and $\$P^3+X$. The  \textit{PennyPincher3D} ($M{=}0.047 $ms, $SD{=}0.060 $ms) is significantly faster than other recognizers; It is significantly better than Jackknife ($M{=}0.195$ ms, $SD{=}0.2457$ ms) with a difference of $148 \mu s$ ($Z{=}73.62$, $p{<}.001^{***}$), and outperforms significantly  the slowest recognizer $\$P^3+$ ($Z{=}147.7$, $p{<}.001^{***}$).


\begin{figure}[ht]
\captionsetup{justification=centering}
  \subfigure[Execution times for all conditions.]{\label{fig:SHREC2019-Avg-ExecTime}\includegraphics[width=0.75\linewidth, trim= -25 0 0 0 ]{Figures/Chap5/SHREC2019/Time_Avg_AllCond_SHREC2019.pdf}}
  \centering
   \subfigure[Exec. times for the optimal conditions.]{\label{fig:SHREC2019-BestCond-ExecTime}\includegraphics[width=0.75\linewidth, trim= -25 0 0 0 ]{Figures/Chap5/SHREC2019/Time_BestCond_SHREC2019.pdf}}
  \caption{\centering Average execution times by recognizer. Error bars show a confidence interval of 95\%. }
    \vspace{-20pt}
\end{figure}
\clearpage
\subsubsection{\textbf{Overall Execution Time by Number of Joints}} The graphs in Figure~\ref{fig:SHREC2019-Avg-Time-Joints} shows the execution times according to the values of the variable \textsc{Joints} (A). The smallest value in the graphic is the averaged execution time of \textit{PennyPincher3D} for the condition \textsc{$A=1\:(P)$} ($M{=}0.017$ ms, $SD{=}0.020$ ms), while the longest execution time is achieved by $\$P^3+$ for  \textsc{$A=6\:(P+AF)$} ($M{=}3.204$ ms, $SD{=}5.936$ ms). The \textit{PennyPincher3D} is the fastest recognizer and is ahead of the Jackknife. According to the results, we distinguish two groups of recognizers with regard to their execution times. The one formed by Jackknife ($M{=}0.195$ ms, $SD{=}0.246$ ms) and \textit{PennyPincher3D}, with a small slope, are not much affected by the variation in the number of articulations. The second group consisting of $3~Cent$($M{=}1.130$ ms, $SD{=}0.683$ ms), $\$F$ ($M{=}1.170$ ms, $SD{=}3.328$ ms), $\$P^3+X$($M{=}1.298$ ms, $SD{=}2.731$ ms) and $\$P^3+$ in order from fastest to slowest recognizer; All the recognizers are significantly impacted by the variation of Number of Joints ($p{<}.001^{***}$).

\begin{figure}[ht]
	\centering
	\vspace{-16pt}
	\captionsetup{justification=centering}.
	\includegraphics[width=\linewidth]{Figures/Chap5/SHREC2019/Time_Avg_per_Joints.pdf}
	\vspace{-16pt}
	\caption{Average execution times by recognizer per number of joints (A).}
	\label{fig:SHREC2019-Avg-Time-Joints}
%	 \vspace{-16pt}
\end{figure}

\subsubsection{\textbf{Execution Time for the Best Condition}} Figure~\ref{fig:SHREC2019-BestCond-ExecTime} shows the execution time of the recognizers for the best condition defined by the de Borda method. The recognizers appear in the same order as in the Figure~\ref{fig:SHREC2019-Avg-ExecTime}. \textit{PennyPincher3D} remains the fastest ($M{=}0.069$ ms, $SD{=}0.005$ ms) and $\$P^3+$ the slowest ($M{=}2.371$ ms, $SD{=}0.213$ ms). The Kruskal-Wallis shows  significant differences between all the recognizers except between the $\$P^3+$ and $\$P^3+X$ ($Z{=}1.910, {n.s.}$).

\vspace{-8pt}
\subsection{Jackknife-LM Dataset}
\subsubsection{\textbf{Overall Recognition Rate}}
The Jackknife-LM dataset has more gesture classes than SHREC2019, mainly complex gestures where different joints can move independently of the hand movement. Figure~\ref{fig:Jackknife-rate-Overall}  sums up the averaged recognition results of each of the recognizers for all tests under all conditions. The results give an overview of the recognition with this dataset which are different from the results obtained for the SHREC2019. In general, none of the recognizers went beyond $80\%$. The Jackknife recognizer ($M{=}73.60\%$, $SD{=}19.75\%$) takes the lead and outperforms significantly other recognizers, followed by the $\$P^3+$ ($M{=}68.75\%$, $SD{=}19.73\%$), which is slightly better than $\$P^3+X$ ($M{=}68.15\%$, $SD{=}20.51\%$) by ($0.60\%$). They are followed by  $3~Cent$ and $\$F$ with respectfully ($M{=}63.66\%$, $SD{=}18.29\%$) and ($M{=}62.96\%$, $SD{=}20.12\%$) which are significantly better than \textit{PennyPincher3D}  the least efficient of the recognizers ($M{=}56.87\%$, $SD{=}19.99\%$).

\begin{figure}[ht!]
    \vspace{-25px}
    \subfigure[Recognition rates for all conditions.]{\label{fig:Jackknife-rate-Overall} \includegraphics[width=0.487\textwidth,trim= 0 -20 0 -20]{Figures/Chap5/JackKnife/Rate_AllCond_Overall.pdf}}
    \hfill 
    \subfigure[Execution times for all conditions.]{\label{fig:Jackknife-time-Overall} \includegraphics[width=0.48\textwidth,trim= 0 -20 0 -20]{Figures/Chap5/JackKnife/Time_AllCond_Overall.pdf}}
  \hfill  \newline
      \subfigure[Overall recognition rates for all conditions by gesture class.]{\label{fig:Jackknife-rate-Byclass}\hspace{20px}\includegraphics[width=0.9\textwidth]{Figures/Chap5/JackKnife/Rate_AllCond_ByGestures.pdf}}
\label{fig:JN-LM}
\caption{Execution times of all recognizers for the Jackknife-LM for the user-independent scenario. Error bars show a confidence interval of $95\%$.}
\end{figure}

Three gesture classes (\enquote{FistCircles}, \enquote{Knock} and \enquote{Sideways}) are well recognized for all the recognizers (\textit{\ie}, $\tau{\geq}80\%$ - Figure~\ref{fig:Jackknife-rate-Byclass}, bottom). An exception to this is  \textit{PennyPincher3D} where the \enquote{Knock} gesture has an average recognition rate of ($M{=}62.77\%$). However, the \enquote{Love}, \enquote{BendIndex}, \enquote{DevilHorns}, and \enquote{SnipSnip} have the worst average recognition rates for most recognizers (\textit{\ie}, $\tau{\leq}60\%$),  except for Jackknife, where the \enquote{SnipSnip} and \enquote{BendIndex} are above $70\%$. These results indicate that for many recognizers, many conditions perform very well with gestures where the fingers remain static, regardless of hand movements. Whereas the recognition rate drops for many conditions with gestures that include finger movements.
%From the table in the Appendix~\ref{app:reco-ui-JK-LM} which details the average recognition rates for the different conditions, the orange cells are the predominant ones in the table for the majority of the recognizers, which express a low recognition rate for many conditions $\tau{\leq}80\%$ (\eg \textit{PennyPincher3D}, $3Cent$ and $\$F$). For \textit{PennyPincher3D}, $3Cent$, no single condition achieves a $90\%$ recognition rate. While for other recognizers, many conditions reach rates above $90\%$ with \textsc{$A=5\:(AF)$} and \textsc{$A=6\:(P+AF)$}, which denotes the ability to handle complex gestures under certain conditions.

\subsubsection{\textbf{Recognition Rates by Number of Joints}}
With regard to the average recognition rates by articulation in Figure~\ref{fig:plot-Recog-by-Joint-Jk-LM}, the lowest recognition rate is $M{=}39.78\%$ for $\$F$ in \textsc{$A=1\:(P)$},  while the highest recognition rate is $M{=}87.78\%$ for $\$P^3+X$ in \textsc{$A= 6\:(P +AF)$}. The Jackknife performs better than other recognizers for the conditions with a reduced number of articulations ($A{\leq}3$), and it is joined by $\$P^3+$ and $\$P^3+X$ at \textsc{$A= 5\:(AF)$}. The average recognition rate increases as the number of joints increases for all of the recognizers, except for some specific cases, such as for $3~Cent$, \textit{PennyPincher3D}, and $\$P^3+$, whose rate drops between \textsc{$A=5\:(AF)$} and \textsc{$A=6\:(P + AF)$} (\eg a drop of $2.6\%$ for $3~Cent$). However, the recognition rates of $\$P^3+X$, $\$P^3+$, and $\$F$ under conditions with the same number of joints vary depending on how the joints are configured (\eg,  \textsc{$A{=}2 (P+I)$}, \textsc{$ A{=}2(T+I)$}, \textsc{$A{=}2 (I+M)$}). Therefore, some joints are more relevant than others as they carry more information about gestures.

\begin{figure}[ht]
	\centering
	\captionsetup{justification=centering}
	\includegraphics[width=\linewidth]{Figures/Chap5/JackKnife/Rate_Avg_per_Joints.pdf}
	\vspace{-16pt}
	\caption{Recognition rates of all recognizers for the Jackknife-LM dataset~\cite{Taranta:2017}, the plot shows average rates by joint $A=\{1\:(P), ... ,6\:(P + AF)\}$.}
	\label{fig:plot-Recog-by-Joint-Jk-LM}
	\vspace{-12pt}
\end{figure}

\subsubsection{\textbf{Overall Conditions Execution Time}}
Figure~\ref{fig:Jackknife-time-Overall} shows the average execution times for all conditions, representing the execution time for the different recognizers. The execution time of \textit{PennyPincher3D} ($M{=}0.079$ ms, $SD{=}0.108$ ms) differs significantly from that of $3~Cent$ ($M{=}10.585$ ms, $SD{=}5.911$ ms). The execution time includes both pre-processing and recognition time. In the case of $3~Cent$, the excessive execution time is attributed to the resampling function that uses the Cubic Spline interpolation method. The execution times of the other recognizers, $\$P^3+$ ($M{=}2.587$ ms, $SD{=}4.371$ ms), $\$F$ ($M{=}2.097$ ms, $SD{=}4.184$ ms), and $\$P^3+X$ ($M{=}2.107$ ms, $SD{=}4.371$ ms), are similar, with no significant difference between the latter two ($Z{=}1.772, {n.s.}$). Jackknife, which has the highest recognition rate, is also a fast recognizer ($M{=}0.351$ ms, $SD{=}0.508$ ms), making it suitable for this dataset.

\begin{figure}[h]
    \centering
    \includegraphics[width=.9\textwidth]{Figures/Chap5/Chap5_DollarFamily_Instantiation.pdf}
    \vspace{-8pt}
    \caption{The \$-like recognizers instantiation.}
    \vspace{-8pt}
   \label{fig:design-dollar2}
\end{figure}

\section{Conclusion}
The purpose of this use case is to compare template-based hand gesture recognizers on the \enquote{multipath dynamic} gestures, specifically focusing on the LMC-based gestures. These gestures are represented as a set of hand joint trajectories. The goal is to evaluate six gesture recognizers on two LMC-based datasets, which consist of simple and complex gestures (\ie, SHREC2019 and Jackknife-LM). The evaluation is done in a user-independent scenario and similar to the comparative testing in Chapter~\ref{chap:Unipath_Comparative} in an \enquote{intra-device} configuration where the gestures are recorded with the same device for different stages of the gesture recognition procedure. 

\begin{figure}[t]
    \centering
    \includegraphics[width=.9\textwidth]{Figures/Chap5/Chap5_PennyPincher3D_Instantiation.pdf}
    \vspace{-8pt}
    \caption{The JackKnife and PennyPincher3D instantiation.}
    \vspace{-8pt}
   \label{fig:design-BetweenVectors}
\end{figure}
Based on the design space defined in Section~\ref{sec:Design_Space_Recognizers}, we observe that for this comparative testing, the recognizers have two common instantiations. The first instantiation, shown in Figure~\ref{fig:design-BetweenVectors}, is shared by the JackKnife and PennyPincher3D recognizers, which are based on the  \enquote{Between-Vectors} classification approach. The second instantiation, shown  in Figure~\ref{fig:design-dollar2}, shows the values covered by the other \$-like recognizers ($\$F$,$\$P^3+$, \textit{3 Cent}, and $\$P^3+X$), which rely  on the \enquote{Between-points} classification approach.

For the SHREC2019 dataset, $\$P^3+$ achieved the best recognition rate and outperformed the other recognizers. This was confirmed under particular optimal conditions. However, $\$F$ is the worst for this dataset. It was also observed that certain recognizers are better suited for recognizing specific gestures compared to others. 

For the Jackknife-LM dataset, the Jackknife recognizer achieved good overall recognition rates under certain conditions but failed to meet the high accuracy requirement for many other conditions. The performance of recognizers was found to be influenced by the nature of the gesture being performed. Additionally, some recognizers were slower in processing longer gestures, making them less interesting.


Interestingly, the recognizer using the \enquote{Between-vectors} classification approach, PennyPincher3D, did not perform better than some other recognizers using the \enquote{Between-points} classification approach such as  $\$P^3+$ or the other recognizer relying on \enquote{Between-vectors}, Jackknife, in terms of recognition rate. However, PennyPincher3D was the fastest recognizer among those evaluated.

This comparative testing demonstrates the potential benefits of the systematic procedure for the comparative testing method and provides valuable insights for designers and researchers seeking a reliable and efficient recognizer to meet their needs.

   



 
%\chapter{\textsc{QuantumLeap} Extension: Testing Module}\label{chap:QuantumLeapTesting}
In the previous chapters, two separate experiments were carried out to evaluate the recognition rate and execution time of template-based 3D hand gesture recognizers. A reasonable range of datasets was used to ensure a robust evaluation, including various classes of gestures of different natures (\ie, \enquote{unipath dynamic} and \enquote{multipath dynamic} gestures). The selected recognizers' evaluation included user-dependent and user-independent scenarios for several gesture sets. The findings from these evaluations provide valuable insights for researchers and practitioners, enabling them to select the most suitable algorithm for their specific context of use while considering performance aspects. To accomplish this, there are few, if any, evaluation tools available that automate and support the systematic testing procedure of 3D hand gesture recognizers in multi-device configurations.

However, no tool with complex evaluation functions currently includes multiple recognizers, datasets, scenarios, and parameters. For this reason, we have implemented a testing module, a support tool that automates the evaluation of 3D gesture recognizers on multiple gesture sets. This module can be customized with different parameters. The core of this testing module is inspired by a part of the software architecture of the QuantumLeap Framework~\cite{Sluyters:2022}. 
It uses some essential components to create a pipeline for automating the evaluation of gesture recognizers. The QuantumLeap Framework is designed to facilitate the development of gestural user interfaces using the Leap Motion controller. It simplifies the development process by providing a software architecture with a pipeline for acquiring, segmenting, recognizing, and mapping gestures to application functions.

\section{Evaluation Procedure}
The comparative testing method use cases in this work are classified as an analytic method of evaluation~\cite{Whitefield:1991}. It indicates that the user interface (UI) and user are representational during the evaluation. Inspired by the gesture recognition testing procedure described in Section~\ref{sec:Testing_Procedure}, we have already described a concrete application of the gesture recognition procedure on \enquote{unipath dynamic} gestures in Subsection~\ref{subsec:Comparative_scenarios}. Two scenarios are defined: the user-dependent scenario and the user-independent scenario. In the user-dependent scenario, templates are randomly selected from each participant for testing, and a training set is created by randomly choosing templates from the same user. In the user-independent scenario, templates are randomly selected from a participant for testing, and a training set is created by randomly choosing templates from different users. The evaluation result is determined by calculating the execution time and the recognition rate of the gesture recognizers.
\subsection{User-dependent Scenario}
In this evaluation, a recognizer is trained and tested using gestures performed by the same user from the same dataset. The user-dependent evaluation introduces two important parameters: $T$, which represents the number of training samples used, and $R$, which represents the number of tests performed for each participant. The procedure randomly selects one sample per gesture class for testing and then selects $T$ samples per gesture class from the remaining samples for testing.

\subsection{User-independent Scenario}
In this evaluation, a recognizer is trained using gestures from one dataset performed by one or more users and then tested with gestures performed by a different user from the same dataset. This evaluation introduces a parameter, denoted as $P$, which represents the number of independent participants used to train the recognizer. The procedure randomly selects one participant and then randomly selects one sample from each gesture class for testing. It then randomly selects a group of $P$ independent participants to train the recognizer. For each participant in the training set, $T$ samples per gesture class are randomly selected for training from the same dataset.
\\

These scenarios are seen in this work and applied in the case of \enquote{intra-device } configuration. However, QuantumLeap's ability to handle multi-device configurations expands the number of possible scenarios. This includes using two datasets from two devices from the same input device family or leveraging QuantumLeap's extension capabilities to other input device categories.
In addition to the mentioned parameters, we can introduce new parameters relevant to different contexts of use of gesture recognition. For example, as discussed in Chapter~\ref{chap:LMCmultipathComparative}, we can consider the articulation parameter $A$ or the number of points during gesture resampling in the pre-processing step $N$.

The first main measure in these procedures is the accuracy of a gesture recognizer, which is represented by the average recognition rate of each recognizer, whether it is global (\ie, grouping all conditions) or specific to a particular condition. More detailed results are provided, such as the number of recognized gestures for each trial and the confusion matrix. The second measure is the efficiency of the recognizers, which is determined by the average execution time of each recognizer, whether it is global (\ie, grouping all conditions) or specific to a particular condition. The pre-processing time refers to the time it takes to preprocess the candidate before comparing it with the templates. The classification time refers to the time it takes to compare the candidate with all the templates. The execution time is the total time it takes for preprocessing and classification. All these times are measured in milliseconds (ms).

\section{Testing Module Implementation}
\label{sec:Implementation}
To perform comparative testing on 3D gesture recognizers, a testing module has been implemented in JavaScript within a Node.js environment ~\url{https://github.com/Mehous/TestingFramework_QL_UI}. Node.js is a cross-platform environment that fulfills the testing requirements. The tool is a single-page application designed to launch testing and provide evaluators with several options for different parameters, gesture sets, and recognizers. Installing the framework requires basic knowledge of command-line interfaces and Node.js, while configuration involves modifying the configuration file.
The application requires storing a set of recognizers, datasets, and a configuration file locally. The recognizers should be in JavaScript, while the datasets should be in JSON format. The results are saved as JSON files, with a defined structure for each evaluation scenario. A large amount of data is returned in the results. They can be global, such as the average recognition rate for a participant, the average execution time for a particular condition of $P$, or a confusion matrix. They can also be condition-specific, such as the classification time per class or the number of recognized gestures from a set of gestures for a specific condition of parameters $N$, $P$, and $T$.
\subsection{Configuration}
The tool comes with a configuration UI shown in Figure~\ref{fig:UI} that allows evaluators to easily set the initial evaluation conditions according to their requirements. Before selecting the evaluation condition values on the configuration page, the configuration file needs to be customized. It includes a list of available recognizers, datasets for each recognizer, and various parameters with their respective values.

\begin{figure}[ht!]
    \centering
    \includegraphics[width=.7\textwidth]{Figures/Chap6/App_Comparative_Testing_Module.pdf}
    \vspace{-10pt}
    \caption{The UI configuration page.}
    \label{fig:UI}
    \vspace{-12pt}
\end{figure}
\subsection{Recognizers}
Recognizers are the fundamental components of the test module. They are responsible for assigning a gesture class to a candidate gesture. Like in QuantumLeap framework~\cite{Sluyters:2022}, recognizers implement three methods:
\begin{itemize}
\item \textsf{addGesture(name, sample, [parameters])}: it adds a gesture template to the training set of the recognizer.
\item \textsf{convert(sample,[parameters])}: This function converts the loaded sample in memory into an array of sets of points.
\item \textsf{recognize(sample,[parameters])}: designate the name of the gesture class that corresponds to the candidate sample provided as an argument.
\end{itemize}
These methods are primarily used in conjunction with segmenters (seen in Subsection~\ref{subsec:Gesture_Segmentation}), which detect dynamic gestures from a continuous stream of data.
\subsection{Testing}
The testing is the central element of the module, which contains the different evaluation functions. It calculates the different results and writes them to JSON files located in predefined folders created based on the values of the different variables. the test item allows the evaluator to adjust the values of the parameters under evaluation before initiating the procedure Moreover, the test records logs of the evaluated conditions and saves the gesture templates used for training and testing, allowing other recognizers to be evaluated on the same training and test sets.
\subsection{Datasets}
The testing module must have at least one dataset to evaluate gesture recognizers. This gesture set is stored as files in folders that must follow a standardized structure. This structure is necessary to enable the loader methods to load data from memory into another suitable structure for use in a specific evaluation scenario. The structure is such that each folder contains subfolders representing the participants who performed the gestures stored in JSON format.


A gesture file stored in a dataset must adhere to a certain structure, to ensure optimum use of the data it contains. The gestures used in the testing module share a number of attributes, such as \enquote{$name$}, which designates the name of the gesture class of the sample. The \enquote{$subject$} field, indicates the name or identifier of the user who performed the gesture. the \enquote{$date$} the gesture was created by the user. The gesture file contains a set of paths. The attribute \enquote{$paths$} includes a set of paths that have names similar to the articulations provided by the  \enquote{multipath dynamic} hand gestures of the Leap Motion controller. These paths are named after the associated articulations and contain a list of strokes. A stroke is a set of coordinates  of points ordered chronologically ($x, y, z, alpha, beta, gamma$), along with another attribute \enquote{$timestamp$} and the \enquote{$stroke\_id$}. Finally, the sample defines an object named \enquote{$device$} with various properties related to the device's information. These properties include \enquote{$osBrowserInfo$}, \enquote{$resolutionHeight$}, \enquote{$resolutionWidth$} \enquote{$rwindowHeight$}, \enquote{$rwindowWidth$} ,\enquote{$rpixelRatio$}, \enquote{$mouse$}, \enquote{$pen$}, \enquote{$finger$}, \enquote{$acceleration$} and \enquote{$webcam$}.

\subsection{Results}
Once the tests are completed, the module generates multiple result files in JSON format. These files are stored in folders and subfolders, following a hierarchical structure of $scenario \rightarrow dataset \rightarrow recognizer$. In each recognizer folder, we decided to store the result files by aggregating the results according to the conditions under which the tests were performed, choosing a predefined order of importance for the parameters as shown in Figure~\ref{fig:UDResult}. We cite the measures that are calculated for aggregated data according to different variables.
\begin{itemize}
    \item Recognition Accuracy
    \item Execution Time
    \item pre-processing Time
    \item Classification Time
    \item Confusion Matrix
    \item Execution Time per Class
    \item pre-processing Time per Class
    \item Classification Time per Class
    \item Raw Recognition Rate
    \item Raw Execution Time
    \item Raw pre-processing Time
    \item Raw Classification Time
\end{itemize}
\begin{figure}[ht!]
    \centering
    \includegraphics[width=\textwidth]{Figures/Chap6/Result_Structure.pdf}
    \vspace{-15pt}
    \caption{The structure of a result file in user-dependent evaluation.}
    \label{fig:UDResult}
    \vspace{-12pt}
\end{figure}

\section{Conclusion}
The testing module is a support tool for the comparative testing of 3D gesture recognizers. It is designed to help researchers and practitioners. The module is implemented in JavaScript. This tool supports two evaluation scenarios: user-independent and user-dependent. It requires the recognizers to be implemented in JavaScript and the datasets to be imported in JSON format.
The testing module generates detailed test results that are written in multiple JSON files. These files are stored in automatically created folders that correspond to different testing conditions. The results contain information such as the overall recognition rate and the average pre-processing time for a set of gesture candidates.



 
%\chapter{Conclusion} \label{chap:thesisconclusion}
Developers and practitioners face numerous challenges when developing interactive solutions incorporating hand gesture recognition. This is especially true with the rise and widespread use of mobile devices equipped with various sensors, opening up countless interaction possibilities. The design, testing, and validation of interactive systems are becoming increasingly complex, mainly due to the growing popularity of complex machine learning models in gesture recognition.

This work focuses on the definition and application of a comparative testing systematic procedure for template-based 3D hand gesture recognizers in multiple contexts of use, that permits developers and practitioners to choose a recognizer effectively for developing a gesture interface using rapid prototyping during the design stage. This systematic procedure allows for the comparison of gesture recognizers in multiple contexts of use, which is precisely tailored to 3D hand gesture recognizers adapted for the multi-device.

\begin{itemize}
    \item First, we introduced the existing state-of-the-art 3D template-based gesture recognizers adapted for rapid prototyping. We also described and compared the most impactful recognizers based on various dimensions defined for rapid prototyping hand gesture recognizers derived from observations, such as gesture natures, data types, device types, classification approaches, gesture types, and invariance properties. At the end of the chapter, we concluded that there is a need for an extended exploration of classification and device dimensions.
    \item Second, we have introduced the terminology and defined the fundamental concepts for conducting gesture recognition comparative testing in multi-device configuration. This includes the design of a model determining the main stages of gesture recognition testing, identifying the main elements of the comparative testing method, and featuring the systematic nature of the testing procedure.
    \item We have implemented the systematic procedure of the comparative testing method to evaluate the accuracy and efficiency of 7 state-of-the-art recognizers on\enquote{unipath dynamic} gestures. To achieve this, four 2D stroke gesture recognizers ($\$P$, $\$P+$, $\$Q$, and Rubine) from the literature review were extended to 3D: $\$P+^3$, $\$F$, FreeHandUni, $\$Q^3$, $\$P^3$, Rubine3D, and Rubine-Sheng. Significant differences were observed among these recognizers regarding recognition rates and execution times on three datasets (\ie, SHREC2019, 3DTCGS and MadLabSD). $\$P+^3$ consistently achieved the highest recognition rate and excellent execution time in most scenarios, making it the preferred choice for efficiently recognizing 3D trajectories. After $\$P+^3$, we observe that the subsequent recognizers, including $\$F$, $FH$, $\$Q^3$, $\$P^3$, have similar recognition rates except for $R3D$ and $RS$ which have low recognition rates. There is a significant difference when transitioning from the user-dependent to the user-independent scenario. The number of templates has a significant effect on the recognition rates of all the recognizers, and the number of points has a significant effect on the \$-like recognizers. Similarly, the datasets have a significant impact on the recognition rates of all the recognizers in the user-independent scenario.
    \item  We have implemented the systematic procedure for the comparative testing method to evaluate the accuracy and efficiency of 6 recognizers on\enquote{multipath dynamic} gestures. This study compared six 3D template-based gesture recognizers from the literature. Our focus was on their recognition rate and execution time. We conducted experiments on two LMC-based datasets: SHREC2019 and Jackknife-LM. We defined independent variables with a limited selection of eight joint combinations. The SHREC2019 results indicated that $\$P^3+$ achieved the highest performance, followed by $\$P^3+X$ and Jackknife. The \$-like recognizers also demonstrated good performance for the same classes of gestures.
    In addition, all of the recognizers exhibited low response times, making them suitable for template matching. We used the de Borda ranking to select the best conditions among the gesture recognizers. Among the recognizers tested on the Jackknife-LM dataset, the Jackknife recognizer was the best, while PennyPincher3D and 3 Cent should be avoided.
    \item Finally, we have implemented and described the testing module, which is a tool that facilitates the integration of the systematic procedure for comparative testing of 3D hand gesture recognizers in multiple contexts of use. This offline tool extends the QuantumLeap Framework to evaluate 3D gesture recognizers using different datasets. The testing procedures assess the accuracy and efficiency of the recognizers in two evaluation scenarios: user-dependent and user-independent, and in the intra-device configuration.
\end{itemize}

\section{Advantages and Limitations}
Several advantages are to be considered in this work:

\begin{itemize}
%\item Introduction of terminology and definitions that take into account multi-device, that could enrich the knowledge domain of gesture recognition.
\item Introduction of terminology and extended definitions that consider the multi-device in the gesture recognition procedure.

\item The work allows developers and practitioners to view the problem of gestural interface design from a perspective where the systematic procedure for the comparative testing can precisely meet their needs. This systematic procedure reduces time costs and workload, helping with rapid prototyping and easing the burden of the development process by designating the suitable recognizer in multiple contexts of use.
\item The systematic procedure's ability to cover many possibilities enables it to evolve into more complex comparative testing methods, adding new recognizers, datasets, scenarios, and parameters. However, it is essential to note that we have only tested a limited number of instantiations for template-based 3D gesture recognizers.

\item The testing module, which extends the QuantumLeap framework, is primarily dedicated to the Leap Motion controller. However, it can also integrate other data sets and devices for comparative testing in single and multi-device configurations. Its proven ability to support this comparative procedure is a key feature to ensure reproducibility in experiments. Moreover, the systematic nature of the comparative testing ensures consistency across different experimental conditions for different dimensions, by altering one component of the comparative testing while keeping all others constant. This method is frequently used in competitions and contests for evaluating the performances of new machine-learning models or algorithms as the SHREC contest series is organized every year~\cite{Caputo:2019,Caputo:2021,DeSmedt:2017}.

\end{itemize}

However, there are some limitations and shortcomings in this work:

\begin{itemize}

\item The exploration is limited only to template-based recognizers, disregarding complex recognizers based on more complex machine learning techniques~\cite{Cheng:2016, Yasen:2019}.

\item The conducted experiments within the two use cases do not include the entire values of the different dimensions of the design space, such as \enquote{Simple static} and \enquote{Complex static}, as well as the \enquote{Between-vectors} classification approach implemented in many gesture recognizers such as $\mu_V$~\cite{Magrofuoco:2022}. Given the abundance of datasets that are often flawed and take time to process, I'm building upon the work presented in Chapter~\ref{chap:LMCmultipathComparative}. My focus is on processing seven out of the twelve available Leap-motion-based datasets, and evaluating the results of hundreds of thousands of tests.

\item The application of comparative testing in this work is part of the analytic methods evaluation class that relies on theoretical representations of the user and computer.  However, this method is limited because it is not a user-based evaluation because there is no real end-user, which is consistent with the desire to use the systematic procedure for rapid prototyping~\cite{Macleod:1992}. This classification results from the choice of conducting only the training and testing stages of the gesture recognition procedure.

\item The experiments carried out involved only the \enquote{intra-device} configuration. Furthermore, the data used in the experiments are all of the same type, specifically Segmented3D, as outlined in the design space of the gesture recognizer defined in Chapter~\ref{chap:Related}. While the datasets primarily provide gestures as sets of 3D positions, devices can provide a variety of data types, such as rotations and accelerations. However, due to a lack of sufficient publicly available datasets, these 3D hand gesture recognizers were not tested.

\end{itemize}
\section{Future Work}
Potential avenues for future research could be explored across a variety of dimensions.
\begin{itemize}
    \item  Along the support tool: The development of the testing module, a tool that extended the QuantumLeap framework~\cite{Sluyters:2022}, encounters the limitations of QuantumLeap. Its primary limitation is that it is confined to the Leap Motion controller or a specific device family and does not support other multi-device gesture recognition configurations. To address its issues, we are currently working on a new framework named \textit{zeroG}, which is a tool that helps users with different levels of expertise with the development of gesture-based applications (Figure~\ref{fig:zeroG_framework_Mockups}). The framework also includes a testing tool for users who need to perform comparative testing or benchmarking to explore various configurations of gesture recognizers. The aim is to help them make informed decisions that best suit their needs. It allows the evaluation of one or more recognizers on different datasets and using different parameters in automated sessions, with control over the testing procedure execution(Figure~\ref{fig:zeroGMockups_TestingDashboard}. The tool works by creating data dataflows for gesture recognition, which involves linking different standardized modules that can be customized, as shown in Figure~\ref{fig:zeroGMockups_LaunchedTesting}. The testing configuration GUI will provide numerous opportunities to integrate various devices in multi-device configurations. It should allow gesture recording, acquisition of entire datasets, and their management. The GUI should also provide the ability to create new datasets by assembling gestures from different existing datasets stored in the framework's database. Furthermore, it should allow users to publish and share their own datasets and results, which could be used to provide new datasets and centralize them to create competition on gesture recognition. The results of the testing will be available for download or will be viewable using the results visualization tool (refer to Figure~\ref{fig:zeroGMockups_TestingResults}). This tool will allow the generation of common charts, such as a confusion matrix, and will offer users the ability to manipulate these results. Users will be able to aggregate data based on a specific variable or filter results to show a particular testing configuration. This will provide the opportunity for preliminary analysis of various measures, such as recognition rate and execution time.

      \item Along the scope of the experiments: The shortcomings mentioned above suggest extending the systematic procedure to include more datasets and exploring new gestures that are relevant to specific application fields or performed with different parts of the human body, especially those used for gesture elicitation studies~\cite{Vanderdonckt:2019}. Additionally, more recognizers (such as DeepGRU~\cite{Maghoumi:2019}) and different conditions, such as more or fewer classes or templates per gesture class, could be considered. This would allow for a direct comparison of other recognizers to the best competitor emerging from the previously completed systematic procedure on a common dataset. Like in a competition. Generally, the practice of varying one aspect of comparative testing while keeping the rest constant is often seen in competition~\cite{Caputo:2019,Caputo:2021,DeSmedt:2017,Ruffieux:2013}. There are many possible configurations at the first level. After that, parameters can also be varied, opening up countless possibilities:
     \begin{itemize}
\item \textbf{Vary}: Recognizer. \textbf{Fixed}: Dataset; Method; Scenario; Parameters; Technique
\item \textbf{Vary}: Dataset, \textbf{Fixed}: Recognizer; Method; Scenario; Parameters; Technique
\item \textbf{Vary}: Method, \textbf{Fixed}: Recognizer; Dataset; Scenario; Parameters; Technique
\end{itemize}


 \begin{figure}[ht!] 

\vspace{-0.4cm}
\centering
    \subfigure[zeroG Testing dashboard.]{\label{fig:zeroGMockups_TestingDashboard} \includegraphics[width=0.75\textwidth]{Figures/Chap7/zeroGMockups_TestingDashboard.pdf}}
  
    \subfigure[zeroG Testing session and configuration.]{\label{fig:zeroGMockups_LaunchedTesting} \includegraphics[width=0.75\textwidth]{Figures/Chap7/zeroGMockups_LaunchedTesting.pdf}}

      \subfigure[zeroG Testing results.]{\label{fig:zeroGMockups_TestingResults}\includegraphics[width=0.75\textwidth]{Figures/Chap7/ZeroGMockups_TestingResults.pdf}}
\vspace{-0.4cm}
\caption{ZeroG testing tool mockups UI.}
\label{fig:zeroG_framework_Mockups}
\end{figure}
\clearpage

     \begin{itemize}
\item \textbf{Vary}: Scenario, \textbf{Fixed}: Recognizer; Dataset; Method; Parameters; Technique
\item \textbf{Vary}: Parameters, \textbf{Fixed}: Recognizer; Dataset; Method; Scenario; Technique
  \item \textbf{Vary}: Technique, Fixed: Recognizer; Dataset; Parmeters; Method; Scenario;
\end{itemize}

      \item Along the hypothesis to be tested: The method defined in this work can be applied to other contexts of use, including different multi-device configurations. The hypothesis put forward in this respect is supportable and deserves to be explored further.
       \item Along the design guidance: The systematic procedure could help to define a process for prototyping a gesture-based user interface using the comparative testing method. Once a dataset is decided, the systematic procedure for comparative testing could automatically determine which recognizer is the most suitable depending on the conditions imposed. We envision an automatic generation of a decision tree that guides the practitioner on which recognizer to select for one or several datasets by asking questions at each decision tree node, such as: Do you want to maximize the recognition rate? The execution speed? Under which circumstances? The framework could be designed to integrate a complete testing system and allow for the direct integration of the recognizer resulting from the systematic procedure for comparative testing into a real-time application.
\end{itemize}

 



%----------------------------------------------------------------------------------------
%	THESIS CONTENT - APPENDICES
%----------------------------------------------------------------------------------------
\newpage
\appendix % Cue to tell LaTeX that the following "chapters" are Appendices

% Include the appendices of the thesis as separate files from the Appendices folder
% Uncomment the lines as you write the Appendices

%\input{Appendices/AppendixA}
%\input{Appendices/AppendixB}
%\input{Appendices/AppendixC}

%\newpage\null\thispagestyle{empty}\newpage

%----------------------------------------------------------------------------------------
%	BIBLIOGRAPHY
%----------------------------------------------------------------------------------------
\newpage
\fancyhead[LO,RE]{}%\slshape Bibliography}
\fancyhead[LE,RO]{\textsl{BIBLIOGRAPHY}}%Bibliography}}
%\fancyhead[LE]{\slshape Bibliography}
\printbibliography%[heading=bibintoc]

%----------------------------------------------------------------------------------------

%\newpage\null\thispagestyle{empty}\newpage

\end{document}  
