\thispagestyle{plain}
\begin{center}{\huge\textit{Abstract}\par}\end{center}


\vspace{2cm}
\addchaptertocentry{\abstractname} % Add the abstract to the table of contents
Context-aware Interactive applications are those interactive software that have parts or whole changing depending on the constraints imposed by a dynamically-changing context. The context of use covers the user and the associated tasks, the computing platform and devices, and the physical environment. Therefore, any significant change of any of these three dimensions, i.e., the user, the platform, and the environment, may trigger a change of the software, including the user interface. Until now, such changes have been managed by the system, thus leading to a system-controlled context-awareness. Instead, we want to pursue the goal of letting the end user control the changes, thus leading to a user-controlled context awareness. To enable the end user to have this control facility, there is a need for another user interface than the one of the original software. This is the Extra-User Interface, which is hereby referred to as the user interface to control the user interface of another interactive application to support context awareness. This concept can be applied in principle to any domain of computer science (e.g., ambient intelligence, smart rooms, ubiquitous computing, multimedia). We will instantiate this approach to the area of information visualization.

\vspace{0.5cm}%{0.33cm}
\textbf{Keywords:} Adaptation, Adaptation operation, Adaptive user interfaces, Extra User Interface, Human-Computer Interaction, Information visualization, Meta User Interface, Plasticity of user interfaces, User control 
\checktoopen
\null
\vspace{1em}%{0.33cm}



