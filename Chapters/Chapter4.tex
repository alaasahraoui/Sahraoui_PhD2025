\chapter{Operationalizing Extra-User Interfaces: Model-Driven Methods, Use Cases, and Prototypes}
\label{chap:operationalization}

\section{Introduction}
\label{sec:chap4-intro}
Chapter~\ref{chap:design} proposed a design space for Extra-User Interfaces (Extra-UIs) that characterizes how users can observe, configure, and control user interface (UI) adaptation along three dials: manipulated objects, power of the extra-UI, and perceived qualities. In this chapter, I move from this conceptual design space to concrete operationalizations.

I adopt a broad definition of Extra-UIs as \emph{any user interface aimed at supporting software evolution of the user interface}. In the context of this thesis, I focus on one particular form of software evolution: \emph{post-deployment UI adaptation}. This constraint implies that the Extra-UI is shipped together with the interactive application and is responsible for orchestrating the adaptation life cycle of an already deployed UI.

The objective of this chapter is threefold:
\begin{itemize}
    \item To present a model-driven method for engineering Extra-UIs, grounded in explicit models of users, adaptations, and adaptation operations.
    \item To describe how this method has been instantiated in the SYMBIOTIK project through several successive implementations of an adaptation engine and its Extra-UI.
    \item To analyze these implementations as use cases of the design space introduced in Chapter~\ref{chap:design}, and to extract design lessons for future Extra-UIs.
\end{itemize}

he chapter is structured as follows. Section~\ref{sec:chap4-from-design-space} connects the design space to the requirements and constraints of the SYMBIOTIK project. Section~\ref{sec:chap4-model-driven} presents the underlying models that drive the Extra-UI: user models, adaptation models, and adaptation operations. Section~\ref{sec:chap4-arch} introduces the architecture of the SYMBIOTIK adaptation engine and its Extra-UI. Section~\ref{sec:chap4-usecase-panel} details the SYMBIOTIK adaptation panel as a concrete Extra-UI that guides users through a seven-stage adaptation cycle. Section~\ref{sec:chap4-evolution} describes three successive implementations of the adaptation engine, including the current version used for a user study and for integration with a reinforcement learning (RL) module. Section~\ref{sec:chap4-discussion} discusses the main design lessons with respect to user control, transparency, and the design space. Section~\ref{sec:chap4-conclusion} concludes and prepares the transition toward the evaluation of Extra-UIs in the next chapter.

\section{From Design Space to Project Requirements}
\label{sec:chap4-from-design-space}

\subsection{Extra-UIs as shipped controllers of post-deployment adaptation}
\label{subsec:chap4-extra-ui-definition}

The design space in Chapter~\ref{chap:design} characterizes Extra-UIs along three non-oriented axes: the nature and visualization of manipulated objects, the services and extensibility of the interaction language (power), and the level of categorization and user control (qualities). In SYMBIOTIK, these axes are instantiated under a set of project-specific constraints.

First, the Extra-UI is tightly coupled with a concrete application: an analytical dashboard built with the AEGIS Advanced Visualization Toolkit (AVT). The Extra-UI---the adaptation panel---is not itself the dashboard UI. Instead, it is the UI that governs the adaptation life cycle of the dashboard UI. As such, it must be deployed together with the dashboard, share access to its configuration, and manipulate its visualization state at runtime.

Second, the Extra-UI must support a spectrum of initiative and control, from fully manual adaptability (user-driven) to fully automatic adaptivity (system-driven), as defined in the adaptation levels of the SYMBIOTIK framework. In the lowest level of the spectrum, the end user manually configures and executes all stages of the adaptation cycle. In the highest level, the intelligent driver can execute the entire cycle without any interaction with the user, while the Extra-UI still maintains observability and potential override.

Third, the Extra-UI must mediate between heterogeneous sources of intelligence: model-based rules grounded in the Cameleon Reference Framework~\cite{Calvary:2003}, preference-based and recommendation-based selection of adaptation operations, data-driven learning from user behavior~\cite{Lavie:2010,Zhang:2007}, and RL-based strategies operating on graph visualizations~\cite{Todi:2021}. The Extra-UI is therefore not only a control panel but also a \emph{coordination surface} between multiple adaptation engines.

\subsection{Functional and non-functional requirements}
\label{subsec:chap4-requirements}

The SYMBIOTIK adaptation engine and its Extra-UI are driven by a set of functional and non-functional requirements that can be expressed as user stories. They instantiate the design space in terms of concrete capabilities of the Extra-UI. The main functional epics include:
\begin{itemize}
    \item \textbf{Widget and chart management.} Users must be able to reorder, resize, add, and remove widgets and charts on the visualization canvas to adjust the layout to their analytical needs.
    \item \textbf{AI-enabled adaptation control.} Users must be able to enable or disable AI-enabled adaptations for specific widgets and charts, and to configure criteria under which adaptations should be applied.
    \item \textbf{Adaptation engine evaluation.} Users must be able to evaluate the performance of the adaptation engine, provide feedback on the quality of suggested adaptations, and compare AI-enabled adaptations with manual adjustments.
    \item \textbf{User registration and access control.} Users must be able to sign up, log in, and be assigned roles with specific permissions, ensuring secure usage and controlled access to adaptation capabilities.
\end{itemize}

The non-functional requirements further refine the qualities dial of the design space. They cover:
\begin{itemize}
    \item \textbf{Usability.} The Extra-UI must remain intuitive enough for analysts, with efficient task execution for configuration and evaluation tasks.
    \item \textbf{Performance and scalability.} Adaptations must be applied within a reasonable time frame and the system must remain responsive under an increasing number of users and dashboards.
    \item \textbf{Security and privacy.} Communications between the Extra-UI, the adaptation engine, and the backend must be secure, and user data must comply with data protection regulations.
    \item \textbf{Documentation.} Users, administrators, and developers must have access to comprehensive documentation on how to use, configure, and maintain the adaptation engine and its Extra-UI.
\end{itemize}

These requirements articulate the expectations placed on Extra-UIs in a real-world, multi-partner European project. They guide the model-driven approach presented in the next section.

\section{Model-Driven Methods for Extra-UIs}
\label{sec:chap4-model-driven}

\subsection{Modeling the user: preferences, context of use, and adaptation profile}
\label{subsec:chap4-user-model}

The first pillar of the model-driven method is an explicit user model. In SYMBIOTIK, the user model is captured by a \texttt{User} class and related classes, represented in a UML diagram (Figure~\ref{fig:symbiotik-user-class}). This class encodes both static and dynamic user attributes relevant to UI adaptation:
\begin{itemize}
    \item identity and contact attributes (first and last name, email);
    \item psychological and experiential attributes (MBTI profile, experience level with the adaptation engine);
    \item global adaptation preferences (overall adaptation level);
    \item configuration-related attributes (path to the AVT configuration file associated with the user);
    \item cognitive and sensing attributes (cognitive load threshold, preferences for sensing devices and measures);
    \item adaptation life cycle preferences (cycle preferences, sense and measure preferences, locations of log files).
\end{itemize}

This model serves several roles in the Extra-UI:
\begin{itemize}
    \item It supports initial configuration of adaptation behavior through the user profile page.
    \item It parameterizes the adaptation life cycle, for example by setting thresholds for cognitive load or determining which sensing devices can be used.
    \item It provides features for similarity computations in recommendation-based selection of adaptation operations, by associating each user with an indexed vector over adaptation operations.
\end{itemize}

The Extra-UI exposes this model through several sections in the user profile page: a general section, a preferences section, a sensing devices section, and a measures section. These sections allow users to configure their adaptation profile in a way that is both inspectable and adjustable.

\subsection{Modeling adaptation: goals, initiatives, specifications, transitions, interpretations, evaluations}
\label{subsec:chap4-adaptation-class}

The second pillar is a model of the adaptation itself. SYMBIOTIK adopts a seven-stage adaptation life cycle aligned with the Cameleon Reference Framework~\cite{Calvary:2003} and extended for context-aware adaptive visualizations~\cite{Bouzit:2017}. This life cycle is instantiated by an \texttt{Adaptation} class (Figure~\ref{fig:symbiotik-adaptation-class}) that decomposes an adaptation into six core subcomponents:
\begin{itemize}
    \item \textbf{Goal statement.} The objectives of the adaptation, including the type of goal, optimization operator, target measure, and target visual element.
    \item \textbf{Adaptation initiative.} The initiative or strategy for triggering adaptation, capturing the agent (user, system, mixed-initiative), timing, and conditions.
    \item \textbf{Adaptation specification.} The detailed set of adaptation operations to be performed on the dashboard elements.
    \item \textbf{Adaptation transition.} The way the transition between pre- and post-adaptation states is visualized to the user, including animation or other techniques~\cite{Dessart:2012,Huhtala:2010}.
    \item \textbf{Adaptation interpretation.} The interpretation of changes in user-based measures after adaptation, for example how performance, error rates, or cognitive load evolved.
    \item \textbf{Adaptation evaluation.} User and system evaluations of the adaptation, including ratings, scales, and comparisons with baseline behavior.
\end{itemize}

By grounding each stage in an explicit submodel, the Extra-UI can offer dedicated UI components for editing, inspecting, and evaluating each part of the adaptation. This modularity also supports traceability from user actions in the Extra-UI to changes in the underlying adaptation model.

\subsection{Adaptation operations and lateral transformations}
\label{subsec:chap4-operations}

The third pillar consists of adaptation operations that concretely transform the UI. In SYMBIOTIK, an adaptation operation is defined as a high-level function that modifies one or several parts of a concrete UI model, such as the AVT configuration file, while preserving the level of abstraction. Following the Cameleon Reference Framework, such transformations correspond to \emph{lateral adaptations} at the concrete level~\cite{Calvary:2003}.

Each adaptation operation is characterized by:
\begin{itemize}
    \item a unique identifier and short name;
    \item an optional long description clarifying the expected effect;
    \item a reference to the adaptation strategy it contributes to;
    \item a signature specifying input and output elements (e.g., identifiers of widgets, reference sizes);
    \item a textual rationale explaining why the operation is relevant in certain contexts of use;
    \item an applicability scope (linguistic level and abstract task such as encode, select, navigate, arrange, filter, aggregate, annotate, import, derive, record);
    \item one or more examples illustrating the operation before and after adaptation;
    \item a preference score reflecting how much the operation was preferred by end users in empirical studies.
\end{itemize}

In earlier work on adaptive menus, preference scores were used to rank different adaptive menu variants (e.g., greyscale, transparency, highlighting) according to empirical acceptance~\cite{Lavie:2010}. In SYMBIOTIK, preference scores are combined with other signals to rank operations and to support recommendation.

\subsection{Scoring and recommendation of adaptation operations}
\label{subsec:chap4-scoring-knn}

To promote user control while leveraging past behavior and expert knowledge, the adaptation engine integrates scoring and recommendation mechanisms for adaptation operations. The scoring function considers several perspectives:
\begin{itemize}
    \item a \emph{score of changes} (SC), which increases when a user modifies an operation suggested by the engine;
    \item a \emph{score of unchanges} (SU), which increases when a user leaves a suggested operation unchanged;
    \item a \emph{design score} (SA), which captures expert judgment on the quality and relevance of an operation;
    \item a \emph{global score} (SG), which summarizes how frequently an operation is executed, possibly taking into account recency and popularity.
\end{itemize}

A basic scoring function can be expressed as:
\[
\text{Score}(\text{operation}) = P \times SC + D \times SU
\]
where \(P\) and \(D\) weight how often an operation is actively selected versus merely accepted as default. A more advanced scoring function enriches this with global and design scores:
\[
\text{Score}(\text{operation}) = P \times SC + D \times SU + T \times SG + f(w, SA)
\]
where \(T\) is the total number of executions and \(f(w,SA)\) integrates designer recommendations with configurable weights.

To turn these scores into recommendations, a k-nearest neighbors (k-NN) algorithm is used in a lazy-learning fashion~\cite{Zhang:2007}. Each user is associated with a vector over adaptation operations, where each component reflects whether an operation has been accepted (+1), rejected (-1), or never used (0). A similarity matrix is computed over users, and prediction scores are obtained by weighting other users’ choices by their similarity to the current user. Four compatibility scores distinguish between null, undetermined, compatible, and incompatible usage patterns, allowing the Extra-UI to balance exploration and exploitation when suggesting operations.

This model-driven and data-driven selection mechanism is not directly visible to the user. It is mediated by the Extra-UI, which presents candidate operations in a ranked list, allows the user to inspect and modify them, and logs subsequent behavior for future refinement.

\section{Architecture of the SYMBIOTIK Adaptation Engine}
\label{sec:chap4-arch}

\subsection{Two logical modules: intelligent driver and adaptation engine}
\label{subsec:chap4-modules}

At the architectural level, the SYMBIOTIK adaptation engine comprises two logical modules within an adaptation manager:
\begin{itemize}
    \item The \textbf{Adaptation Intelligent Driver}, responsible for ensuring and managing the entire adaptation life cycle. It parses adaptation strategies (for example, generated by an RL module), selects candidate adaptation operations using the models and scoring functions described above, and orchestrates iterations with the user through the Extra-UI.
    \item The \textbf{Symbiotik Adaptation Engine}, responsible for executing the selected adaptation operations on the concrete UI model. In the SYMBIOTIK project, this corresponds to modifying AVT configuration files that describe dashboard visualizations and pushing updated configurations back to AVT.
\end{itemize}

Figure~\ref{fig:symbiotik-architecture} provides an overview of this architecture, including interactions with the AVT module and external services such as authentication (Keycloak), Kafka-based event streams, and the RL module.

\subsection{Front-end, back-end, database, and AVT integration}
\label{subsec:chap4-tech-arch}

The current implementation is structured around four technical components:
\begin{itemize}
    \item A \textbf{front-end Extra-UI}, implemented in React and TypeScript, that presents the adaptation panel, user profile pages, configuration controls for experiments, and real-time feedback on applied adaptations. Material-UI and Redux are used respectively for styling and state management.
    \item A \textbf{back-end adaptation service}, implemented with NestJS or Node.js/Express, that processes Extra-UI requests, interfaces with MongoDB, manages experiment trials, and orchestrates adaptation operations.
    \item A \textbf{MongoDB database} that stores user accounts, profiles, preferences, adaptation definitions, and logs.
    \item The \textbf{AVT dashboard module}, which maintains and renders the analytical dashboards. The adaptation engine manipulates AVT configuration structures and communicates with AVT through HTTP endpoints, for example to post new configurations or retrieve the current one.
\end{itemize}

Real-time communication between the front-end and back-end is handled via Socket.IO, while user interaction events (e.g., mouse movements, question dialogs) and RL suggestions are transmitted through Kafka topics. This architecture enables the Extra-UI to present live feedback about which graph is currently active, which strategy has been applied, and which adaptation operation has just been executed.

\subsection{The adaptation panel as Extra-UI}
\label{subsec:chap4-panel-extra-ui}

The adaptation panel is the extra-UI of the SYMBIOTIK dashboard. It is conceptually distinct from the dashboard UI in several ways:
\begin{itemize}
    \item It manipulates \emph{models of the dashboard} (AVT configuration, adaptation models, user models), not the underlying data.
    \item It exposes \emph{controls on the adaptation life cycle}, rather than direct analytic filters or visual encodings.
    \item It mediates \emph{mixed-initiative} collaboration between the user and automated modules (scoring engine, RL engine, catalog of operations).
\end{itemize}

From the perspective of the design space in Chapter~\ref{chap:design}, the adaptation panel:
\begin{itemize}
    \item manipulates objects such as adaptation goals, operations, strategies, and measures;
    \item offers services such as creating a new adaptation, inspecting and editing existing adaptations, applying or rolling back an adaptation, and evaluating its impact;
    \item provides qualities such as transparency (visualization of strategies and operations), traceability (logging and replay), and a tunable level of user control along the adaptability--adaptivity continuum.
\end{itemize}

The next section details how this panel is used in a concrete use case to guide the user through a full adaptation cycle.

\section{SYMBIOTIK Use Case: Extra-UI for Controlling Dashboard Adaptations}
\label{sec:chap4-usecase-panel}

\subsection{Authentication and access to the Extra-UI}
\label{subsec:chap4-auth}

When a user reaches the SYMBIOTIK adaptation panel, the first interaction is authentication. The Extra-UI provides sign-in and sign-up pages for creating and accessing user accounts (Figures~\ref{fig:symbiotik-signin} and~\ref{fig:symbiotik-signup}). Users can register with a unique username and password, and administrators can assign roles and permissions.

From the design space perspective, this step configures the \emph{subject} of adaptation: instead of adapting for a generic user, the Extra-UI works with a personalized user model bound to a specific account.

\subsection{Configuring user preferences and sensing}
\label{subsec:chap4-preferences}

Once authenticated, users can access their profile and set preferences that will guide the adaptation life cycle. The profile is structured into several sections:
\begin{itemize}
    \item a \textbf{general section} where users specify their overall adaptation level, experience, and MBTI profile;
    \item a \textbf{preferences section} where they indicate how much they want to rely on automated adaptations versus manual control;
    \item a \textbf{sensing devices section} where they define which sensing devices (such as EEG sensors) are available and should be used;
    \item a \textbf{measures section} where they configure which measures (e.g., cognitive load, engagement, performance) should be monitored.
\end{itemize}

Figures~\ref{fig:symbiotik-profile-general}--\ref{fig:symbiotik-profile-measures} illustrate these sections. These configurations instantiate the design space along the \emph{object} axis (which measures and devices are manipulated) and influence the \emph{qualities} axis (e.g., the granularity and invasiveness of sensing, the level of automation that is acceptable for the user).

\subsection{Creating and running an adaptation cycle}
\label{subsec:chap4-cycle-ui}

The core functionality of the Extra-UI is to support the creation and execution of adaptation cycles. Users access a ``New adaptation'' page where the seven stages of the SYMBIOTIK adaptation framework are presented as a sequence of sections.

\paragraph{Goal statement.}
In the \emph{Goal} section (Figure~\ref{fig:symbiotik-goal}), the user specifies the objective of the adaptation: the type of goal, the optimization operator (e.g., minimize selection time, maximize accuracy), the measure to optimize, and the visual element(s) to be targeted. This stage links the adaptation to measurable objectives and explicit visual targets.

\paragraph{Initiative.}
In the \emph{Initiative} section (Figure~\ref{fig:symbiotik-initiative}), the user chooses which agent should trigger the adaptation (e.g., the user, the intelligent driver, an RL module, or a mixed-initiative combination), as well as the temporal and contextual conditions that should lead to an adaptation. This stage configures the initiative dimension of mixed-initiative interfaces~\cite{Horvitz:1999} along the adaptability--adaptivity continuum.

\paragraph{Specification.}
In the \emph{Specification} section (Figure~\ref{fig:symbiotik-specification}), the user defines what will be changed in the UI: which adaptation operations will be applied, on which dashboard elements, and with which parameters. Operations can be selected from a catalog, suggested by the adaptation engine based on the scoring function, or specified manually. This stage connects the goal to a concrete set of lateral transformations.

\paragraph{Application.}
In the \emph{Application} section (Figure~\ref{fig:symbiotik-application}), the user validates the set of operations and configures application details such as timing, execution mode, and rollback strategies. The Extra-UI thus supports both one-shot and repeatable adaptations, and makes explicit how to recover if an adaptation degrades performance or usability.

\paragraph{Transition.}
In the \emph{Transition} section (Figure~\ref{fig:symbiotik-transition}), the user chooses how the transition from the original UI state to the adapted state should be visualized. Options include immediate changes, animated transitions~\cite{Dessart:2012,Schlienger:2007}, or progressive disclosure. This stage is crucial to preserve users’ mental models and to mitigate change blindness.

\paragraph{Interpretation.}
In the \emph{Interpretation} section (Figure~\ref{fig:symbiotik-interpretation}), the Extra-UI displays summary statistics and visual feedback on how the chosen measures evolved after adaptation. This may include performance metrics, error rates, or cognitive load indicators gathered from sensing devices~\cite{GasparFigueiredo:2023}. The user can inspect these outcomes in relation to the original goal.

\paragraph{Evaluation.}
Finally, in the \emph{Evaluation} section (Figure~\ref{fig:symbiotik-evaluation}), the user can rate the adaptation using scales or questionnaires, compare it to baselines, and decide whether to accept it, reject it, or trigger a new cycle with modified parameters. The system can also compute automatic evaluations based on the optimization objective.

Not all stages need to involve the user at runtime. When all stages are driven by the user, the adaptation corresponds to full adaptability. When all stages are executed automatically based on predefined policies and learned models, the adaptation corresponds to full adaptivity. Intermediate levels can be defined by selectively involving the user in some stages while performing others transparently.

\section{Evolution of the Adaptation Engine Through Three Implementations}
\label{sec:chap4-evolution}

\subsection{First implementation: architecture and model grounding}
\label{subsec:chap4-first-impl}

The first implementation of the adaptation engine served to operationalize the seven-stage adaptation life cycle and the associated models in a proof-of-concept architecture. It focused on:
\begin{itemize}
    \item defining the software architecture of the adaptation engine and its integration with AVT;
    \item modeling the \texttt{User} and \texttt{Adaptation} classes and their relationships;
    \item implementing a preliminary set of adaptation operations and their execution on dashboard configurations.
\end{itemize}

This implementation validated the feasibility of using a model-driven approach to connect user profiles, adaptation goals, and concrete operations on dashboards. It also provided early feedback on the usability of the Extra-UI for editing complex adaptation models.

\subsection{Second implementation: scoring function and recommendation engine}
\label{subsec:chap4-second-impl}

The second implementation focused on the selection of adaptation operations through a scoring function and a k-NN recommendation engine. It built upon earlier work on adaptive forms and menus and was instantiated in a form-based adaptive system for a car rental case study.

In this case study, users could adapt form widgets (such as text fields and text areas) based on their preferences and tasks. User actions were logged and fed into the scoring function, and a k-NN algorithm suggested operations that similar users had found beneficial. A user study with 55 participants showed that over time, users’ adaptations converged toward a set of preferred widgets for each field, indicating that the combination of user-driven and engine-driven adaptation could stabilize on configurations that were broadly acceptable.

From an Extra-UI perspective, this implementation demonstrated that:
\begin{itemize}
    \item exposing recommendations as ranked options within an Extra-UI can support progressive tailoring without forcing users to accept opaque adaptations;
    \item logging adaptation choices and outcomes creates a feedback loop that can improve future recommendations;
    \item scoring factors and similarity metrics can be tuned to balance expert-designed operations with end-user preferences.
\end{itemize}

\subsection{Third implementation: user study engine and RL-integrated engine}
\label{subsec:chap4-third-impl}

The third implementation, which is the current version in SYMBIOTIK, consists of two closely related variants of the adaptation engine. Both use the same core architecture (React front-end, Node.js/Express or NestJS back-end, MongoDB, Kafka, Socket.IO, and AVT), but differ in how adaptation operations are triggered and selected.

\paragraph{Adaptation engine for user study.}
The first variant was designed to support a controlled user study (Experiment \#5 in SYMBIOTIK Deliverable D3.3). The Extra-UI allows experimenters to configure study conditions:
\begin{itemize}
    \item \emph{Frequency} of adaptations (e.g., every 5, 10, 20, 30, or 60 seconds).
    \item \emph{Adaptation mode} (no adaptations, one adaptation at a time, or applying all available adaptations).
    \item \emph{Automation} (randomly generated adaptations versus RL-suggested adaptations).
    \item \emph{Task} (the question dataset that participants must answer on the AVT dashboard).
    \item \emph{User and organization} (to retrieve the Keycloak ID and log events per user).
\end{itemize}

The back-end manages trials by:
\begin{itemize}
    \item establishing Kafka consumers for mouse-tracking and dialog events;
    \item scheduling adaptation operations according to the configured frequency and mode;
    \item applying operations from an adaptation catalog to the active graph (timeline, network, or distribution chart);
    \item logging all interactions and adaptations to CSV files with timestamps, event types, and operation details.
\end{itemize}

This variant instantiates the Extra-UI not only as a tool for end users, but also as a tool for experimenters to prototype and study different adaptation policies in controlled conditions.

\paragraph{RL-integrated adaptation engine.}
The second variant integrates the adaptation engine with an RL module that suggests adaptation strategies and target graphs. The engine:
\begin{itemize}
    \item listens to RL events on a dedicated Kafka topic;
    \item maps RL actions to adaptation strategies (e.g., color-based, shape-based, thickness-based, proximity-based, or combinations thereof);
    \item maps RL representations to target graphs (network, timeline, distribution);
    \item retrieves the current AVT configuration from a dedicated endpoint, rather than maintaining local state;
    \item selects and applies adaptation operations consistent with the RL strategy and graph;
    \item logs RL-driven adaptations for subsequent analysis or training.
\end{itemize}

In this configuration, the Extra-UI remains the primary place where users can observe which strategies and operations are being applied, inspect the current adapted state, and potentially override or evaluate adaptations. It thus acts as a boundary object between the RL engine, the adaptation engine, and the human analyst.

\section{Discussion: Design Lessons for Extra-UIs}
\label{sec:chap4-discussion}

\subsection{Support for user control across the adaptation life cycle}
\label{subsec:chap4-discussion-control}

The SYMBIOTIK Extra-UI demonstrates that user control can be supported at multiple points in the adaptation life cycle:
\begin{itemize}
    \item \emph{Before} adaptation, through explicit configuration of goals, initiatives, and specifications.
    \item \emph{During} adaptation, through control over timing, execution mode, and transitions.
    \item \emph{After} adaptation, through interpretation and evaluation stages, logging, and rollback options.
\end{itemize}

By associating each stage with a dedicated section in the Extra-UI, the design space of Extra-UIs is instantiated in a granular manner: users can decide where they want control and where they are comfortable delegating to the engine. This is consistent with principles of mixed-initiative interaction~\cite{Horvitz:1999} and with the need to avoid over-automation~\cite{Parasuraman:1997}.

\subsection{Transparency, traceability, and explanation}
\label{subsec:chap4-discussion-transparency}

The implementations also highlight the role of Extra-UIs in making adaptations transparent and traceable:
\begin{itemize}
    \item The adaptation panel exposes which goals, strategies, and operations are in effect.
    \item Real-time feedback shows which graph is currently active and what operation was just applied.
    \item Logs and CSV files provide a persistent record of how the system behaved over time.
\end{itemize}

These features make it possible to explain why a certain adaptation occurred and to reconstruct its effects on performance and behavior. Extra-UIs thus serve as a key infrastructure for explainable adaptive systems and for later evaluation of symbiotic relationships between humans and adaptive systems~\cite{Zeng:2023}.

\subsection{Coverage of the design space}
\label{subsec:chap4-discussion-designspace}

The SYMBIOTIK use case covers several regions of the design space introduced in Chapter~\ref{chap:design}:
\begin{itemize}
    \item On the \emph{object} axis, the Extra-UI manipulates user models, adaptation models, and catalogues of adaptation operations, as well as dashboard widgets and measures.
    \item On the \emph{power} axis, it offers services ranging from simple configuration and inspection to complex orchestration of RL-based strategies, and it is extensible through the addition of new operations and strategies.
    \item On the \emph{qualities} axis, it provides a high level of categorization (separate models and stages) and a tunable level of user control (through preferences, initiative configuration, and evaluation tools).
\end{itemize}

This coverage shows that the design space is not purely theoretical: it can guide the design and evaluation of real Extra-UIs that support both practical project constraints and research goals.

\subsection{Limitations and future directions}
\label{subsec:chap4-limitations}

The current implementations also reveal several limitations:
\begin{itemize}
    \item The configuration interfaces can become complex for non-expert users, especially when multiple measures, devices, and strategies are available.
    \item The scoring and recommendation mechanisms rely on sufficient data to be effective and can be sensitive to the choice of similarity metrics and weighting schemes.
    \item RL integration introduces new challenges in terms of stability, exploration/exploitation trade-offs, and alignment with users’ preferences.
\end{itemize}

Future work could explore:
\begin{itemize}
    \item adaptive complexity of the Extra-UI itself, simplifying views for novices and exposing more advanced controls for experts;
    \item richer forms of explanation and what-if analysis, allowing users to simulate the effects of potential adaptations before committing;
    \item explicit modeling and measurement of cognitive, conative, and computational costs of adaptation, in order to assess the symbiosis between the user and the adaptive system~\cite{Zeng:2023}.
\end{itemize}

\section{Conclusion}
\label{sec:chap4-conclusion}

This chapter has operationalized the concept of Extra-User Interfaces introduced in earlier chapters by presenting a model-driven method, architecture, and concrete use cases in the context of the SYMBIOTIK project. The SYMBIOTIK adaptation panel exemplifies how an Extra-UI can be shipped with an interactive system to govern a full adaptation life cycle, while preserving user control, transparency, and traceability.

The next chapter will ...................
.......................

