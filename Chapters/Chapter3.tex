\chapter{A Design Space for Extra-User Interfaces}
\label{chap:design}
%%#comment should i keep this ? 
Initially, an extra-UI covers the set of functions and their respective UI required to enable the end user to configure, control, and evaluate the state of the current UI in the domain, such as in ambient interactive~\cite{Coutaz:2007}. Like any digital service, an extra-UI:
\begin{enumerate}
\item allows objects to be manipulated,
\item offers functions whose power is intended to cover the required utility,
and
\item has qualities in response to usability requirements.
\end{enumerate}

Figure~\ref{fig:circle} sets out these three aspects with the object, power and quality dials, which we refine into six non-oriented axes:
- For the manipulated objects dial: nature of objects and type of object visualization,
- For the power quadrant: services offered and extensibility of the interaction language,
- For the qualities quadrant: level of categorization and level of user control.
In the following sections, we present each of these axes in detail.
 
\section{A Problem Space for Engineering User Interface Adaptation}
\label{sec:operation}

In the context of UI adaptation, there are many design spaces \cite{Alvarez:2009,Bouzit:2017:PDA,Vanderdonckt:2020}, taxonomies \cite{Kuhme:1992,Dieterich:1994}, and frameworks \cite{Calvary:2002,Lopez:2007,Nivethika:2013,Dubiel:2022}, each shedding light on different perspectives based on the authors' concerns. Here, we adopt a comprehensive approach for the specification of UI adaptation, while also considering the quality of adaptation. We propose a dual problem space model called \textsc{Hemispheres} \cite{Calvary:2007}, designed for the engineering of ``plastic`` interactive systems \cite{Sottet:2007,Sottet:2008,Vanderdonckt:2008,Vanderdonckt:2008:multi}, i.e. systems whose adaptation preserves quality in use over the evolution of the context of use. This problem space separates concerns into two areas:

\begin{itemize}
    \item The specification of the adaptation operation (\eg ``If the battery gets low, then migrate the UI to the nearest platform``);
    \item The specification of the adaptation implementation (\eg including when and how to adapt).
\end{itemize}

The right hemisphere (Adaptation operation) and left hemisphere (Life cycle) are responsible for these concerns.

\begin{figure}
    \centering
%    \vspace{-8pt}
    \includegraphics[width=.4\textwidth]{Images/Circle.pdf}
    \caption{The four quadrants of an adaptation operation: \textsc{condition}, \textsc{event}, \textsc{action}, and \textsc{value}.}
    \label{fig:circle}
%    \vspace{-32pt}
\end{figure}
\subsection{The Right Hemisphere for the Specification of Adaptation Operations}
The right hemisphere examines the specification of \textit{adaptation operations}, which under condition associate a response to a change in the context of use, aiming to preserve a certain set of values, such as quality factors, altogether contributing to the system's worth. The overall format of an adaptation operation is: \textsc{On Event, If Condition, Then Action for Value(s)}. For example, in the adaptation operation ``If the laptop battery gets low, when a smartphone becomes available around, then propose migrating the UI from the laptop to the smartphone``:
\begin{itemize}
    \item The \textsc{Condition} concerns the battery state of the laptop: ``If the battery gets low``;
    \item The \textsc{Event} occurs when the context of use changes with the arrival of the smartphone: ``when a smartphone becomes available around``;
    \item The \textsc{Action} proposed in response to this contextual change response is a UI migration: ``then propose migrating the UI from the laptop to the smartphone");
    \item The \textsc{Value} is implicit: the interaction continuity should be preserved.
\end{itemize}
Note that UI migration~\cite{Grolaux:2004,Frosini:2014} consists of transferring a particular UI from one platform to another while preserving its interaction state.
The key concepts of \textsc{Condition}, \textsc{Event}, \textsc{Action}, and \textsc{Value} (\autoref{fig:circle}) are classical \textsc{Event-Condition-Action} (ECA) rules enhanced with the notion of \textsc{Value}. We will now further examine, define, and develop the four quadrants of \autoref{fig:circle}.
\begin{figure}
    \centering
    \includegraphics[width=\textwidth]{Images/Condition.pdf}
    \caption{The \textsc{Condition} quadrant of our design space for adaptation.}
    \label{fig:condition}
%    \vspace{-12pt}
\end{figure}
\subsubsection{Condition}
The \textsc{Condition} relates to any observable aspect of the interactive ecosystem, which encompasses any variable of the context of use (\eg, user and task, platform/device, and environment)~\cite{Calvary:2002,Calvary:2003}. If the condition is not met, the adaptation operation cannot be triggered. The \textsc{Condition} is expressed as either \textit{atomic} when it consists of only one term or \textit{compound} when it is composed of several terms (\autoref{fig:condition}). Examples of atomic conditions include: ``if the user is color-blind'', ``if there is no ongoing task'', ``if the PC battery gets low'', ``if there is a mobile device near the user'', ``if the computing platform is a public display``, ``if the physical environment is shaky'', ``if the organisational environment is hierarchical''.


An atomic condition concerns an observable (its subject) that can belong to the interactive system in its business domain of application (\eg adapt the UI on business processes~\cite{Sousa:2008}) or adaptation parts, the context of use with its components \textsc{User}, \textsc{Platform}, and \textsc{Environment}, or the deployment of the interactive system in its context of use. The condition is expressed as a formula of the first-order logic \textsc{Predicate} evaluated with potential arguments and their names, such as \textsc{isLow(battery)} or \textsc{isNear(user)}. The expression can involve quantifiers (\eg ``For all'', ``There exists'') that may be positive or negative.

A \textsc{Composite Condition} is a combination of conditions (atomic or composite) using operators. For example, ``If the user's smartphone is switched on AND there is no ongoing task on it'', ``If the laptop battery has dropped AND the user turns on his/her mobile phone OR PDA''. These examples implement two classic types of operators in task modeling~\cite{Paterno:2009}: logical and temporal. Temporal operators allow reasoning about the interaction history. Other types of operator could be imagined.

The condition can be decorated with \textsc{properties} which can also be reasoned about (\eg by property-based reasoning~\cite{Blouin:2011}), such as ``If condition C is frequent'', ``If menu item I has a high probability of selection''. Relevant properties include: 
the actor who evaluated the condition,
the time of condition evaluation (Time),
the confidence factor in the condition's evaluation, and
the frequency of the condition being met. This detailed structure is crucial for creating effective adaptation operations that can be effectively and efficiently implemented to ensure AUIs under varying conditions.

\begin{figure}
    \centering
    \includegraphics[width=\textwidth]{Images/Event.pdf}
    \vspace{-8pt}
    \caption{The \textsc{Event} quadrant of our design space for adaptation.}
    \label{fig:event}
\end{figure}
\subsubsection{Event}
An \textsc{Event} identifies what has changed in the context of use and justifies an action in response to this change. Key elements include (\autoref{fig:event}):

\begin{itemize}
    \item The \textsc{variable} of change: the change can concern the user, the platform, or the environment. For example, if the user turns on a smartphone, the variable is the platform, which is now enriched with this device.
    \item The \textsc{nature} of the variation: in this case, it is the arrival (Addition) of a new platform.
    \item The \textsc{initiator} of the variation: here, it is the user.
\end{itemize}

The \textsc{Event} can also have properties such as the confidence level in perception, the actor of perception, or the frequency of the event. Reasoning can be applied to these properties. For example, the adaptation operation ``If the user is in front of the PC and the confidence factor for this event is 100\%, then...``.
The variables identify what has changed in the context of use. According to the ontology of Crowley \etal (Crowley:2002), four types of changes can occur depending on whether they affect \textsc{entities}, \textsc{roles}, \textsc{relations}, or \textsc{associations} between entities, roles, and relations:
\begin{figure}
        \centering
        \includegraphics[width=0.6\linewidth]{Images/HandMenu.png}
        \caption{Selection of menu item adapted to the hand~\cite{Antoniac:2002}.}
        \label{fig:hand}
%        \vspace{-16pt}
    \end{figure}
\begin{itemize}
    \item \textsc{Entities:} they pertain to the living or inanimate physical world. An entity is a grouping of observables. By measuring observables, the application can detect the presence of physical world entities, recognize them (\eg by gesture recognition through surfaces~\cite{Sluyters:2024}), track them, and determine the value of their attributes and relationships. We generalize this to all observable components of the context of use: user, platform, and environment (\textsc{Type}).
    \item \textsc{Roles:} they are the functions held by these entities. The roles of input and output devices are predominant. In the past, these roles were assigned to the traditional screen with the keyboard and mouse trio. Today, they are replaced by any physical entity that has the right properties and is observable by the system. For example, a hand, traditionally used as a pointing device, becomes a display surface \cite{Antoniac:2002}. Its proximity to the user and the elongated shape of the fingers make it suitable for menu display (\autoref{fig:hand}). A single entity can have multiple roles. In contrast, a role can be held by several entities simultaneously.
    
    \item \textsc{Relations:} they pertain to a set of entities. They can be spatial (\eg ``the user is near the platform''), temporal (\eg ``the device will be available during 5 minutes''), functional (\eg ``the sun illuminates the plant''. Identifying the variables means locating the change in terms of entities (\ie state, roles, or relations) as well as the set of roles and relations held.
        
    \item \textsc{Variation:} it characterizes the type of change. According to Crowley's ontology \cite{Crowley:2002}, the notions of context of use (or usage context) and situation are distinguished. A \textsc{context of use} is defined by a set of roles and relations held by a set of entities. A context change occurs when a role or relation appears (\textsc{Create}), is modified (\textsc{Modify}), or disappears (\textsc{Delete}). A \textsc{situation} is characterized by a mapping between entities, roles, and relations. A situation change occurs when these mappings change (\textsc{Create} or \textsc{Delete}  associations between entities and roles/relations) or when entities appear (\textsc{Create}) or disappear (\textsc{Delete}). \textsc{Modify} is planned to integrate changes in the state of entities into the reflection if necessary.
             This ontology allows the establishment of a graph of contexts and situations. An event is implicitly defined as an arc of this graph. There are two types of events depending on whether they are \textit{intra-context} (\ie between situations within the same context) or \textit{inter-contexts} (\ie between situations of different contexts).
        
\end{itemize}
  
Schmidt \cite{Schmidt:2000} defines the notion of an event: it is not necessarily the conjunction of an exit from one node (context or situation) and an entry into another node. An event can be more atomic, an exit from a node, an entry into a node, or even the presence in a node. Thus, we distinguish atomic events from compound events. Similarly to compound conditions, compound events use operators, particularly logical operators.

The \textsc{Initiator} of the change identifies the primary responsible agent for the variation. This information can be relevant to avoid conflicting with user actions. For example, if the user has moved the UI to a less visible area of the screen, it would be inappropriate to recentralize the UI in response. The initiator can be the interactive system or its usage context. In the interactive system, we distinguish between the business parts and the adaptation parts. For example, the interactive system \textsc{Diffie} initiates the content adaptation and highlights the results since the last visit of the user (\autoref{fig:diffie}). For the context of use, we refer to its three components: user, platform, and environment \cite{Calvary:2003}. By identifying these changes and their properties, the adaptation can better respond to the dynamic conditions in which an AUI operates, ensuring continued usability and satisfaction.
\begin{figure}
    \centering
    \includegraphics[width=\textwidth]{Images/Diffie.png}
    \caption{\textsc{Diffie} adapts the content and highlights the changes since the last visit~\cite{Teevan:2009}.}
    \label{fig:diffie}
\end{figure}




\begin{figure}
    \centering
    \includegraphics[width=\textwidth]{Images/Action.pdf}
    \caption{The \textsc{Action} quadrant of our design space for adaptation.}
    \label{fig:action}
\end{figure}

\subsubsection{Action}

The \textsc{Action} specifies the reaction to implement following a change in the context of use (\autoref{fig:action}). Examples of atomic actions include: ``Migrate the UI to the nearest platform'' \cite{Frosini:2014}, ``Remove infrequent tasks'' \cite{Florins:2004}, ``Execute a specific UI'' \cite{Calvary:2002}, ``Prioritize remodeling'' \cite{Vanderdonckt:2008:multi}, ``Eliminate automatic distribution'' \cite{Grolaux:2004}, ``Declare that the user is tired'', ``Add a specific adaptation operation on top of existing ones'', ``Execute the first applicable operation possible''.

Compound actions can also be imagined by combining actions (atomic or compound) through operators (\eg logical or temporal). Actions, whether atomic or compound, can be decorated with \textsc{properties}. Relevant properties include specifying the actor responsible for their execution and the optimization of their execution (\textsc{Time}). An atomic action is issued with a certain \textsc{force}: the action is either \textsc{proposed} or \textsc{imposed}. Actions are governed by control policies, such as ``Execute all imposed actions''. \textsc{Create}, \textsc{Delete}, and \textsc{Modify} are action types for control policies. These are \textsc{Statement} actions as opposed to \textsc{Strategies}.

\textsc{Strategies} manipulate \textsc{statements}: they filter or weight them. Four types of statements are defined: \textsc{policies}, \textsc{facts}, \textsc{guidelines}, and \textsc{target interventions}. \textsc{Target interventions} act on the context of use (\eg ``turning on the light'') or on the interactive system. For the interactive system, we functionally refine the target into either the business functions (i.e., related to the application domain) or the adaptation itself. In turn, the functional \textsc{coverage} needs to be specified in terms of the components being modified (\ie UI, functional core, and/or dialog control). For the \textsc{functional core}, we simply nuance whether the adaptation concerns the intrinsic functional core or its deployment in the context of use.

Interventions in the UI can be explored from the perspective of software architecture (\textsc{Scope}): which elements (\ie functions, components, processes) and allocations between elements (\ie functions to components, components to processes, processes to physical resources) are concerned. The considered elements and allocations depend on the adaptation level \cite{Vanderdonckt:2008:multi}: \textsc{remodeling} acts on a constant distribution state of the UI on interaction resources, unlike \textsc{redistribution}. Remodeling and redistribution can be specified in terms of \textsc{plan} or \textsc{goal}. The goal sets the objective to be achieved (\eg ``migrate the UI'') without imposing any conceptual or implementation solution, unlike the plan, where all degrees of freedom are fixed.
\begin{figure}
    \centering
    \includegraphics[width=\linewidth]{Images/Home.PNG}
    \caption{Example of a task model for home automation and its related final UI \cite{Sottet:2007}.}
    \label{fig:home}
\end{figure}

Research in plasticity~\cite{Calvary:2002} began studying remodeling on various abstraction levels such as the task level (user tasks \cite{Mezhoudi:2021} and concepts~\cite{Vanderdonckt:2008}), the abstract UI level, the concrete UI level, and the final UI level.

The \textsc{task level} describes the user's task and domain concepts independently of any representation and implementation. ConcurTaskTree and its language \cite{Paterno:2009}, UML class diagrams of OWL ontologies, are usually considered for task and domain modeling, respectively. They can be decorated with properties such as frequency, iteration, and optionality. For example, \autoref{fig:home} ~ illustrates task modeling for a home automation application. The user controls the home temperature (``Manage home temperature'') by iteratively (\ie decoration ``*'') handling the different rooms of the house. Handling a room involves specifying a command (``Specify command'') and optionally checking the feedback (``Check feedback'' with its optionality decoration). Specifying a command involves specifying the room of interest (``Specify room'') and the action to apply: either check its temperature (``Check temperature'') or change its temperature (``Set temperature''). User Interface Description Languages (UIDLs), such as UsiXML \cite{Limbourg:2004} or MariaXML~\cite{Paterno:2009} capture these aspects in a Domain Specific Language.
 


The \textsc{Final UI level} corresponds to the particular implementation of the UI for a given platform in any programming or markup language. For example, \autoref{fig:plasticlock} performs adaptation at the final UI level by adapting the layout and its widgets depending on the length and height of the window, but also depending on the different time zones when the user needs to fly between two areas.

\begin{figure}
    \centering
    \includegraphics[width=\linewidth]{Images/PlastiClock.png}
    \caption{\textsc{PlastiClock}: the final UI changes its layout and widgets depending on the window dimensions and time zone~\cite{Calvary:2004}.}
    \label{fig:plasticlock}
%    \vspace{-8pt}
\end{figure}


\subsubsection{Value}


In economics, \textsc{Value} measures the ratio between a benefit and a cost, which is suitable for adaptation, which always induces some benefits and drawbacks for a cost~\cite{Lavie:2010}. In this vein, we consider two elements: the application benefit (which we generalize into any positive or negative effect) and the implementation cost related to the execution of the adaptation operation. The cost is practical, as opposed to the effect, which is theoretical. The effect captures the expected outcomes of the adaptation operation by formulating properties expressed in a given reference frame:

\begin{itemize}
    \item \textsc{Ensured} by the application of the adaptation operation: they were not ensured before the application and they become so after the adaptation operation is completed. For example, task continuity is ensured after migration.
    \item \textsc{Preserved}: they were initially satisfied and remained so. For example, a domain concept is observable \cite{Cockton:1987} and is still observable after adaptation.
    \item \textsc{Improved}: they were partially satisfied and are now improved. For example, the cognitive load in terms of informational density is improved by moving to a large screen or multiple displays~\cite{Sluyters:2021}.
    \item \textsc{Degraded}: they were partially satisfied and are now deteriorated. For example, graceful degradation~\cite{Florins:2004} to a small screen increases navigation among tasks and increases workload in terms of physical actions.
    \item \textsc{Removed}: they were initially satisfied and they are no longer satisfied. For example, the removal of user guidance due to a lack of display surface will no longer satisfy this property.
\end{itemize}
\begin{figure}
    \centering
    \includegraphics[width=\textwidth]{Images/Value.pdf}
%    \vspace{-8pt}
    \caption{The \textsc{Value} quadrant of our design space for adaptation.}
    \label{fig:value}
%    \vspace{-8pt}
\end{figure}
The \textsc{effect} can be expressed more globally by applying the adaptation operation: does the operation pertain to the survival of the interactive system or the user's comfort? If it is about user comfort, is it functional, \ie related to the system's utility, or extra-functional related to the usability of the adapted UI and/or the meta-UI~\cite{Coutaz:2006} (defined as the UI governing the UI adaptation, later renamed into extra-UI~\cite{Melchior:2012}).

Whether the adaptation operation pertains to the survival of the interactive system or user comfort, it can be applied \textsc{proactively} (\ie in anticipation of a change of context), or \textsc{reactively} (\ie to face a reality). We call this dimension the \textsc{term}.

The \textsc{theoretical value} can be decorated with \textsc{properties} that are useful for reasoning about the adaptation. We identify as relevant the \textsc{validity} of the adaptation operation (\ie expiration date or duration), the \textsc{trust} placed in the operation, its \textsc{actor}, and its \textsc{date of issue} (\eg a recent form adaptation is more trusted than an old one~\cite{Eloi:2024}). Such values can also incorporate software quality factors \cite{iso25010}.

The \textsc{practical value} considers the cost of applying the operation. This cost is measured from a system perspective (\eg by estimating digital and physical resources required for calculation, communication, and interaction) but also from a human (\eg perceptual, cognitive, and motor load) for \textsc{perception} (condition and event) and \textsc{action}. Estimating the cost is a challenging aspect of UI adaptation. The practical value can also be decorated with properties that are evaluated by practice: the frequency of triggering the adaptation, its application and cancellation frequencies \cite{Eloi:2024}, its determinism, which integrates the level of abstraction of the operation. For example, icons in a toolbar (\autoref{fig:promotion}) can be promoted (made larger) or demoted (made smaller) depending on their probability of being selected \cite{Bouzit:2019}. This probability can be computed by recency, frequency, recurrence, importance, or any combination of them \cite{Vanderdonckt:2018, Vanderdonckt:2020}.
The adaptation operations being defined, the next section studies their life cycle.



\begin{figure}
    \centering
    \includegraphics[width=\textwidth]{Images/Promotion.pdf}
    \caption{Promotion or demotion of icons depending on their probability of use \cite{Bouzit:2019}.}
    \label{fig:promotion}
\end{figure}

\subsection{Categories of Adaptation Operations}
While any adaptation operation can be defined according to the terms defined in Section~\ref{sec:operation}, some frequent categories of adaptation operations emerge according to the effect types they produce~\cite{Paterno:2014,Florins:2004}.
We detail them in the next sub-sections.

 \subsubsection{Property-Changing Operations in Adaptation}

These operations adjust the properties of the interface without replacing or splitting elements, allowing for smooth transitions between different states of a UI based on real-time conditions. Property-changing operations are essential for dynamically modifying the presentation and behavior of UI elements in response to changes in context or user interactions. A common example of a property-changing operation can be observed in the calculator interface on a smartphone, which dynamically adjusts based on the device's orientation property.

\textbf{Example: smartphone calculator.}
When a smartphone is held vertically in portrait mode, the calculator default configuration displays only simple arithmetic operations (\autoref{fig:portrait}). When the smartphone is turned horizontally in landscape mode, it provides more space to display sophisticated scientific functions, such as trigonometric functions (\autoref{fig:landscape}). The underlying components remain the same, but the orientation affects how they are displayed and arranged. This modification is a representative example of an adaptation operation that modifies a property. In contrast, \textsc{MiniAba}~\cite{Schlee:2004} enables the end users to (un)select any subset of arithmetic, trigonometric, or advanced functions they want and to recompile the project to get an adapted version.

\begin{figure}[H]
    \centering
    \includegraphics[width=\textwidth]{Images/Calculator.png}
    \caption{Comparison of smartphone calculator in different modes: (a) portrait and (b) landscape (Source: \url{https://dribbble.com/shots/6851610-Calculator-light-mode}.}
    \label{fig:calculatorModes}
\end{figure}
%https://dribbble.com/shots/6851610-Calculator-light-mode

This adaptation operation can be represented as follows:

\textsc{On Change(Orientation)} 

\textsc{If Screen.Orientation=vertical}

\textsc{Then} 
\textsc{(Screen.orientation=horizontal And Display(FullCalculator))} 

\textsc{For Supporting expert mode}.

\begin{enumerate}
\item \textsc{Event:} the user rotates the smartphone from portrait to landscape mode, where the rotation sensor in the device detects that the phone has been turned.
\item \textsc{Condition:} the device orientation changes from vertical (portrait) to horizontal (landscape). The rotation must be 90° to transition the screen from portrait to landscape.
\item \textsc{Action:} the calculator UI reveals advanced functions (scientific mode), modifying the layout and visibility of the interface elements and dynamically adjusting its layout to display more advanced functions like trigonometric operators, exponential functions, and memory functions. In this case, the UI structure remains the same, but its presentation properties (such as visible elements and screen space allocation) are modified to accommodate the additional functions. 
\item \textsc{Value:} to enhance user interaction by dynamically displaying advanced functions, improving efficiency for expert users who need scientific calculations.
\end{enumerate}

%\textbf{Additional Examples of Property-Changing Operations:}

%\begin{enumerate} \item \textbf{Event:} The brightness of the environment changes (detected by a light sensor). \begin{itemize} \item \textbf{Condition:} The ambient light level is below a defined threshold (e.g., low light condition). \item \textbf{Action:} The UI increases the font size and contrast for better readability. \end{itemize} \item \textbf{Event:} The user activates a mobile app in a different time zone. \begin{itemize} \item \textbf{Condition:} The device detects a significant change in geographical location. \item \textbf{Action:} The app updates the displayed time and date based on the new time zone without modifying the core UI elements. \end{itemize} \end{enumerate}

\subsubsection{Splitting Operations}
Splitting operations divide a UI into smaller, manageable components, elements, or interaction spaces to adapt to devices with different display capabilities. These operations ensure the usability and consistency~\cite{Aquino:2010} across multiple platforms \cite{Florins:2006}:

\begin{enumerate}
    \item \textit{Class 1: Split in sequential operators}: if some tasks are arranged in a sequence (\eg Step 1 $\rightarrow$ Step 2 $\rightarrow$ Step 3), the UI before each sequential task can be split to create separate a screen or a page for each step, such as in a wizard widget.
    \item \textit{Class 2: Split before optional tasks}: if a sequence includes optional tasks, the UI can be split before the optional task to avoid users see unnecessary options unless needed.
    \item \textit{Class 3: Use concurrent operators when sequential splitting is not possible}: if sequential splitting is not adequate or possible, the UI can be split using concurrent operators, which indicate tasks that can occur in any order.
    \item \textit{Class 4: Split at the highest level}: when multiple levels of tasks can be split, choose the highest level. This keeps semantically related tasks together.
    \item \textit{Class 5: Distribution of tasks}:
    when splitting occurs under a higher-priority operator, ensure that necessary tasks, like a ``Cancel'' option, are available in all resulting interaction spaces.
\end{enumerate}

\begin{figure}
    \centering
    \begin{subfigure}[b]{0.9\textwidth} % Adjust the width as needed
        \centering
        \includegraphics[width=\textwidth]{Images/fig16a.jpg}
        \caption{A hotel room in three separate windows.}
        \label{fig:splitting_rules16a}
    \end{subfigure}
    \hfill
    \begin{subfigure}[b]{0.9\textwidth} % Adjust the width as needed
        \centering
        \includegraphics[width=\textwidth]{Images/Tabbed.jpg}
        \caption{A hotel room in a tabbed dialog box.}
        \label{fig:splitting_rules16b}
    \end{subfigure}
    \caption{Splitting rules applied to a hotel room booking form~\cite{Florins:2006}.}
    \label{fig:combined_splitting_rules}
\end{figure}
\textbf{Example: Book a hotel room.}
A form used to book a hotel room usually consists of specifying three parts: the \textit{location} with the arrival and departure dates and the number of guests, the selection of the hotel \textit{category}, and a selection of hotel \textit{preferences}. This form can be split into three independent windows (Class 3) as illustrated in  \autoref{fig:splitting_rules16a} or collected in a tabbed dialog box (Class 4)) as reproduced in \autoref{fig:splitting_rules16b}.
When the first task related to user information (\eg user identification) is completed, the subsequent task of specifying booking details, as represented above, and payment information can be represented as follows:

\textsc{On Change(Screen.Size)}

\textsc{If Screen.Size = Small And Task.Operator = Sequential} (\eg User Information $\rightarrow$ Booking Details $\rightarrow$ Payment Information)

\textsc{Then  Split Interaction.Space Before Task "Booking Details"}

\textsc{And Create Interaction.Space "Booking Details"}

\textsc{For Consistency}

The task model could also comprise an optional task where the guest can specify any special request, such as a non-smoking room with wi-fi access. This optional task is therefore subject to the following splitting rule (Class 2):

\textsc{On Change(Screen.Size)}

\textsc{If Screen.Size = Small and Task.Type ("Special Requests") = Optional}

\textsc{Then Split Interaction.Space Before Task="Special Requests"}

\textsc{Create Interaction.Space ("Special Requests")}

\textsc{For Optimization}

This adaptation operation ensures that the optional task "Special Requests" is displayed only when necessary to optimize the screen real estate when the screen resolution is constrained. This is part of the overall UI transformation shown in \autoref{fig:splitting_rules16b}, where navigation is adapted:

\textsc{On Change(Screen.Size)}

\textsc{If Screen.Size = Small}
\textsc{And Task.operator = Concurrent} (\eg User Information $||$ Payment Information)

\textsc{Then Create Interaction.Space ("User Information", "Payment Information") and Create Navigation.Link ("User Information", "Payment Information")}

\textsc{For Reducing(CognitiveLoad)}.

\subsubsection{Replacement Operations}
To address the needs of users with disabilities, replacement operations entail switching out certain types of UI elements with others. By offering alternate modes of interaction for the same task or action, the expected value concerns accessibility. For example, motor or visually impaired users who need to interact with push buttons, perform text input, and select options in a list, could experience trouble. Replacing these elements or enriching them with voice commands or audio feedback should support a better user experience for them \cite{Minon:2016}.
\autoref{fig:Fig13} shows the hierarchy of UI elements for entering personal data through such buttons, text fields, and radio buttons to support graphical interaction. \autoref{fig:Fig14} shows the adapted UI hierarchy where graphical elements have been replaced by speech recognition, vocal textual input and selection with audio prompts to support vocal interaction. This adaptation operation can be represented as follows:

\textsc{On Detect(User.Disability)=true} |
This event triggers the operation whenever a user’s disability is detected, such as when the user logs in or the context changes.

\textsc{If UI.Elements=graphical} |
The condition checks if the current UI elements are graphical. This means the user interface contains some elements like buttons, text fields, or icons that may not be accessible to users with certain disabilities.

\textsc{Then Replace(UI.Elements, graphical, vocal)}
\textsc{And Create(AudioCommands)} |
This action replaces the graphical UI elements with vocal elements and audio commands. The user can now interact with the interface using voice commands and receive audio feedback instead of visual prompts.

\textsc{For Accessibility}
 
\begin{figure}%[H]
    \centering
    \begin{subfigure}[b]{0.45\textwidth}
        \centering
        \includegraphics[width=\textwidth]{Images/replacement_rule_13.png}
        \caption{For graphical interaction.}
        \label{fig:Fig13}
    \end{subfigure}
    \hfill
    \begin{subfigure}[b]{0.47\textwidth}
        \centering
        \includegraphics[width=\textwidth]{Images/Replacement_rule_14.png}
        \caption{For vocal interaction.}
        \label{fig:Fig14}
    \end{subfigure}
    \caption{Hierarchy of UI elements to input personal data ~\cite{Paterno:2011}: (a) before adaptation, (b) after adaptation.}
    \label{fig:combinedFigure}
\end{figure}


\subsubsection{Removal Operations}
Removal operations remove UI elements at run-time for several reasons: they are irrelevant, inappropriate, unnecessary for particular user groups or use cases, or consuming too many resources, such as screen space. Removal operations could physically remove these UI elements, momentarily hide them, or disable them. In contrast to replacement operations, which replace UI elements with alternatives, removal operations simply delete them, either momentarily or permanently. For example, \textsc{Supple}~\cite{Gajos:2010} dynamically adapts UIs based on user behavior and needs. In \autoref{fig:removal_example15}, a print dialog box is adapted to prioritize frequently used options and remove less relevant components. 
\begin{figure}[t]
    \centering
    \includegraphics[width=\textwidth]{Images/supple.pdf}
    \caption{(a) Original print dialog box with multiple steps to access printing orientation. (b) Adapted dialog box with simplified access to printing orientation, removing unnecessary steps \cite{Gajos:2010}.}
    \label{fig:removal_example15}
\end{figure}

\autoref{fig:removal_example15} demonstrates how the UI is adapted to reduce complexity. In the original print dialog box (\autoref{fig:removal_example15}a), changing the print orientation from portrait to landscape requires navigating through multiple steps and options, which can be difficult for users with motor impairments or limited cognitive abilities. This complexity is not only time-consuming, but also increases the risk of errors during interaction. The adapted interface (\autoref{fig:removal_example15}b) simplifies this process by removing unnecessary steps and presenting the landscape printing option directly on the main screen, which reduces the need for complex navigation and makes the UI more intuitive and accessible.

By focusing on the most commonly used functions and removing less relevant options, \textsc{Supple} tailors the UI to the user's needs, resulting in a streamlined interaction experience. This adaptation operation can be represented as follows:

\textsc{On User.Access("LandscapeOption") And Frequency("LandscapeOption") $>$ Threshold}

\textsc{If UI.Contains("IntermediateSteps")}

\textsc{Then UI.Remove("IntermediateSteps") and            UI.Display("LandscapeOption", "MainScreen")}
       
\textsc{On User.Preference("SimplifiedUI")}

\textsc{If UI.Contains("RarelyUsedOptions")}

\textsc{Then UI.Remove("RarelyUsedOptions")}

\textsc{For Reducing(TaskTime)}
\begin{itemize}
    \item \textsc{Event:} the system detects frequent access to the landscape printing function by comparing it to a reference threshold.
    \item \textsc{Condition:} the ``Print'' dialog box includes multiple steps for accessing this function.
    \item \textsc{Action:} the system removes intermediate steps and presents the landscape printing option by default, simplifying the interaction.
    \item \textsc{Value:} the adaptation aims at simplifying the UI by reducing the task completion time.
\end{itemize}

This operation dynamically simplifies the user interface based on the user's capabilities, ensuring a more accessible and user-friendly experience.


\subsection{The Left Hemisphere for the Specification of the Adaptation Implementation}
The left hemisphere identifies four stages in the life cycle of an adaptation operation (\autoref{fig:lifecycle}): the definition of an adaptation operation, its execution, its evaluation and, finally, the consolidation of experience gained with this adaptation. For each stage, we are encouraged to answer the same questions of the Quintilian hexameter~\cite{Motti:2013}, as outlined in \autoref{sec:introduction}.

\begin{figure}
    \centering
    \includegraphics[width=.65\textwidth]{Images/LifeCycle.pdf}
    \vspace{-6pt}
    \caption{Left hemisphere: stages of adaptation life cycle.}
    \label{fig:lifecycle}
\end{figure}

\subsubsection{Definition of the Adaptation Operation}
The \textit{definition} of the adaptation operation is hereby referred to as any specification, possibly by code generation or interpretation of the specification, of the precise instructions to be performed by an adaptation operation. This specification is aimed at enabling the concrete execution of an operation in the next stage. This should not be confused with the specification of adaptation (\autoref{fig:lifecycle}), which specifies which entity is responsible for deciding an adaptation, whereas this definition is aimed at building the adaptation operation itself. This encompasses the research, the design, and the coding of an operation.
In regard to \textit{What?}, we distinguish the specification of the overall goals of the operation (\eg maximize the cognitive load) and the specification of the operation itself (\eg ``Remove optional widgets''). By overall goals, we may consider the specification of the \textsc{Value} to be guaranteed and the domains or zones of plasticity~\cite{Collignon:2008} to be guaranteed for this value \cite{Calvary:2002}. 
Specifications can be written by (\textit{Who?}) experts in human factors or adaptation, the designer, the end user, the interactive system, any third party, or any combination of them. For example, \textsc{Scaler}~\cite{Eloi:2024} enables the definition of an adaptation operation by configuring and balancing weights (\autoref{fig:configuration}) specified by a designer, a developer, and the end-user, either in isolation one by one or from a crowd, as suggested by Nichols \etal~\cite{Nichols:2013}.  

\begin{figure}
    \centering
    \includegraphics[width=\textwidth]{Images/Configuration.png}
    \caption{Configuration of parameterization for adaptivity by unsupervised learning \cite{Eloi:2024}.}
    \label{fig:configuration}
\end{figure}

The specification tool (\textit{How?}) depends, of course, on the actor in charge of specification. Adaptation operations can be defined according to a wide range of tools, spanning from hard-coded environments (\eg the graceful degradation plug-in offers three families of adaptation operations that are all predefined and coded in Java~\cite{Florins:2004}) and knowledge bases (\eg operations for adapting a graphical UI to a mobile device in \cite{Eisenstein:2000} are stored in a knowledge base that is processed by an inference engine) to specification environments (\eg operations are defined in a domain specific language) and model-driven architectures~\cite{Sottet:2007, Sottet:2008}.


Depending on the tools (\textit{How?}), the actor (\textit{Who?}), the definition can be achieved at different times (\textit{When?}): design time, linking time, compilation time, installation or deployment of the interactive system, its execution and inter-session, requiring the interactive system to be stopped and then restarted.
Operations can be stored (\textit{Where?}) within the interactive system itself
(internal) or externally, in configuration files for example.


\subsubsection{Execution of the Adaptation Operation}
Adaptation is a seven-stage process (\autoref{fig:lifecycle}) involving (\textit{What?}) detecting the change in the context of use to take the initiative for adaptation (stage 2), to specify which adaptation (stage 3) and apply it (stage 4, which corresponds to the actual \textit{execution} of the adaptation operation). These functions require
mechanisms that are (\textit{Where?}) internal or external to the interactive system and its AUI. 
These functions can be carried out by four types of actors (\textit{Who?}): the end-user,
another user (\eg a controller), the interactive system, or another system. We identify five key moments for their execution (\textit{When?}): either during execution (\eg at any time or at precise moments), the granularity of the stage can range from the physical action to the session, via the elementary task and the compound task. Depending on the support tool, past actions may be lost or retained.

Support tools (\textit{How?}) can be generic and reusable, or integrated into the interactive system, and consequently non-reusable. They can be equipped with an extra-UI~\cite{Coutaz:2006} (formerly known as meta-UI), which is itself a UI that ensures observability and control of the adaptation~\cite{Bouzit:2017:PDA}. This extra-UI may require
resources (\eg UI components, libraries, perception services, agents, inference engines) that are either prefabricated or generated at runtime~\cite{Blumendorf:2010}. Their location is internal or external to the interactive system and their availability can be perennial (\eg beyond adaptation) or temporary (\eg erased after adaptation).

\subsubsection{Evaluation of the Adaptation Operation}
The \textit{evaluation} is a process that critically examines to what extent the adaptation operation has been effective and efficient, which involves collecting and analyzing data and outcomes resulting from its execution and outcomes. Its purpose is to assess the overall quality of an adaptation operation to improve its effectiveness and to inform future executions.
We identify five actors (\textit{Who?}) likely to be involved in evaluation: an evaluation specialist, a designer, the end-user, and the interactive system or any third party, provided that they hold the computational capabilities to perform such an evaluation.
The evaluation can cover the four stages: the definition of an adaptation operation, its execution, its evaluation (which would then be a recursive process: how to evaluate the evaluation) and its consolidation of experience. For example, \autoref{fig:evaluation} shows how an adaptation operation can be evaluated by the system itself (\eg by performing some automatic evaluation) or by the end user qualitatively (\eg using a rating bar) or quantitatively (\eg using a rating scale). In the past, automatic UI evaluation has been explored, such as by guideline review~\cite{Beirekdar:2005} or agent analysis \cite{Lopez:2009}. More recently, eye tracking and emotion recognition represent attractive candidates to determine to what extent the end-user is happy or not with an adaptation~\cite{Haddad:2024}.

Evaluation can be carried out in the very true context of use or controlled in a specific environment, such as in a usability laboratory. It can be carried out on the fly or off-line (\textit{When?}). For example, captured on-the-fly, the first impression that an end user produces is indicative of the graphical UI being good or bad~\cite{Haddad:2024}. Mechanisms for showing how adaptivity has been performed (\eg \autoref{fig:transition} shows adaptivity at run-time using an animated transition that preserves the context of use) and for explaining it are welcome.

\begin{figure}
    \centering
    \includegraphics[width=\textwidth]{Images/Transition.png}
    \caption{Animated transition showing the various stages of adaptivity \cite{Dessart:2011}.}
    \label{fig:transition}
\end{figure}

\subsubsection{Consolidation of Experience}
Consolidation of the experience gained with the adaptation operation is a new dimension that is required for future use by learning what happened (\eg by keeping an appropriate operation and discarding an inappropriate one~\cite{Zouhaier:2021}). Five players are likely to be involved (\textit{Who?}): an expert in knowledge management or computational approaches \cite{Jiang:2022}, an experienced designer, the end user, the interactive system and its AUI or any other party. Consolidation can be general, covering the value of operations (weak or strong), or limited to values only (\textit{What?}).
Consolidation requires software mechanisms (\textit{How?}), \eg ML/DL/RL tools~\cite{Bouzit:2017:PDA,Gaspar:2024,Todi:2021}, which can be external or embedded in the interactive system itself. The support offered can range from simple memorization of experiences to the deduction of new adaptation operations. Results can be organized or not, justified or not, stored internally or externally to the interactive system (\textit{Where?}), and consolidated on-the-fly or off-line (\textit{When?}).




\section{Practical Lessons from Experience}
Existing methods for designing UIs of interactive systems (\eg \cite{Cockton:1987,Gulliksen:2003}) still apply to AUIs, but with the need to pay attention to the context of use and the quality of the use. For both of them, it must be understood that they may result either from user-centered requirements elicited during the \textsf{Analysis} phase (see the \textsf{Problem} part of the design process in \autoref{fig:Gulliksen}) or from design choices made by the practitioner (see the \textsf{Solution} part of the design process in \autoref{fig:Gulliksen}). Several lessons can be drawn from experience. \textbf{Lesson n°1} recommends considering explicitly both the context of use and the quality in use, together with their rationale. User-centered requirements ``must`` be satisfied, while practitioner's decisions ``should`` be preserved for interaction continuity.
\begin{figure}[b]
    \centering
    \vspace{-16pt}
    \includegraphics[width=.95\textwidth]{Images/UCSD.pdf}
    \vspace{-8pt}
    \caption{Gulliksen's method for designing UIs, adapted from \cite{Gulliksen:2003} (Illustration by courtesy of Jan Gulliksen and Bengt Göransson).}
    \label{fig:Gulliksen}
\end{figure}
Once the context of use is determined and the quality in use is decided accordingly, \textbf{lesson from experience n°2} recommends reasoning for the transitions, considering the change from one context of use to another. The \textbf{lesson from experience n°3} recommends designing a UI first for the most constrained context of use, \ie the one combining the most constrained user, platform, and environment. This ``reference UI`` could then be considered for designing UIs for other contexts of use by progressively relaxing, which is referred to as \textit{progressive enhancement}, the inverse of \textit{graceful degradation}~\cite{Florins:2004}. Serna \etal ~\cite{Serna:2010} demonstrates that this improves the quality of use of the resulting UIs. Consequently, \textbf{lesson from experience n°4} consists of using the AUIs as a means for quality, \ie to consider very constrained contexts of use, even if they are not targeted, just for quality.

\section{Conclusion and Perspectives}
\label{sec:conclusion}
This paper provides an overview of UI adaptation since its inception until recent days, by focusing mainly on adaptive UIs, where adaptation is ensured by the interactive system or application itself, as opposed to adaptable UIs (where the end-user is responsible for adapting the UI) and mixed-initiative UIs (where the system and the end-user collaborate to ensure the adaptation).

For this purpose, we discuss the traditional questions of adaptation (\ie what to adapt, why to adapt, how to adapt, with regard to what, who controls the adaptation, when to adapt, and where to adapt) that need to be answered when adaptation should be implemented in a UI. We then presented a generic UI adaptation life cycle, which can be decomposed into seven stages (\eg ranging from UI adaptation goals to evaluation of adaptation). A targeted literature review resulted in a discussion of existing and future challenges encountered in adaptation and in recent advances in the domain, such as those induced by computational approaches. We then presented a comprehensive problem space for UI adaptation made up of two hemispheres: one for the specification of the adaptation operation (which is the cornerstone of adaptivity) and one for the specification of the implementation of adaptation. The first hemisphere is further decomposed into four quadrants: condition, event, action, and value. The second hemisphere is further divided into four stages: definition, execution, evaluation, and consolidation. From experience, we believe that these structuring elements inform and guide the engineering of adaptive UIs.

The greatest challenge posed by adaptation, now and in the future, is that of the evolution of adaptation. Adaptivity is inevitably an evolutionary process for both the end-user and the interactive system. Users evolve in their experience, the way they perform interactive tasks, and in their preferences. The environment in which they evolve also evolves inexorably. The interactive system cannot remain unaffected by these changes. To date, numerous methods have been applied to compensate for this lack of evolution, mainly by adding new adaptation rules, modifying and improving their application strategies, or taking better account of user parameters. This compensation remains insufficient to draw relevant and useful conclusions about the future of the interactive system.

That is why machine learning, in general, \cite{Mahdavinejad:2018} and reinforcement learning in particular offer many suitable advantages, enabling us to take into account how the UI evolves in its context of use. The most recent advances \cite{Gaspar:2024,Langerak:2024,Todi:2021,Zouhaier:2021,Mezhoudi:2021,Eloi:2024} are heading in this direction, and are already taking constructive steps towards overcoming this lack of consideration for evolution. A second potential avenue is to determine to what extent adaptation can be fragmented in time and space (\eg Todi \etal \cite{Todi:2021} gradually transform graphical adaptive menus, Bouzit \etal \cite{Bouzit:2016} suggest step-by-step progress and adaptation in mobile menus) to deliver an adaptation experience that would reduce adverse effects~\cite{Lavie:2010}, such as disruption~\cite{Hui:2009}.

We hope that the future will allow us to determine the best methods, techniques, and algorithms to ensure the best possible adaptivity. Meanwhile, lessons from experience are reported to reconcile generic know-how in engineering HCI and specific advances in AUIs.

 